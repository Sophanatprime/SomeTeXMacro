\pagenumbering{roman}
\setlength{\columnsep}{3em}
\addcontentsline{toc}{chapter}{目录}
\tableofcontents


\cleardoublepage
\pagenumbering{arabic}
\chapter{甄士隐梦幻识通灵\ttlbreak 贾雨村风尘怀闺秀}


此开卷第一回也。作者自云:因曾历过一番梦幻之后,故将真事隐去,而借"通灵"之说,撰此《石头记》一书也。故曰"甄士隐"云云。但书中所记何事何人?自又云:“今风尘碌碌,一事无成,忽念及当日所有之女子,一一细考较去,觉其行止见识,皆出于我之上。何我堂堂须眉,诚不若彼裙钗哉?实愧则有余,悔又无益之大无可如何之日也!当此,则自欲将已往所赖天恩祖德,锦衣纨绔之时,饫甘餍肥之日,背父兄教育之恩,负师友规谈之德,以至今日一技无成,半生潦倒之罪,编述一集,以告天下人:我之罪固不免,然闺阁中本自历历有人,万不可因我之不肖,自护己短,一并使其泯灭也。虽今日之茅椽蓬牖,瓦灶绳床,其晨夕风露,阶柳庭花,亦未有妨我之襟怀笔墨者。虽我未学,下笔无文,又何妨用假语村言,敷演出一段故事来,亦可使闺阁昭传,复可悦世之目,破人愁闷,不亦宜乎?"故曰"贾雨村"云云。

此回中凡用“梦”用“幻”等字,是提醒阅者眼目,亦是此书立意本旨。

列位看官:你道此书从何而来?说起根由虽近荒唐,细按则深有趣味。待在下将此来历注明,方使阅者了然不惑。

原来女娲氏炼石补天之时,于大荒山无稽崖练成高经十二丈,方经二十四丈顽石三万六千五百零一块。娲皇氏只用了三万六千五百块,只单单剩了一块未用,便弃在此山青埂峰下。谁知此石自经煅炼之后,灵性已通,因见众石俱得补天,独自己无材不堪入选,遂自怨自叹,日夜悲号惭愧。

一日,正当嗟悼之际,俄见一僧一道远远而来,生得骨骼不凡,丰神迥异,说说笑笑来至峰下,坐于石边高谈快论。先是说些云山雾海神仙玄幻之事,后便说到红尘中荣华富贵。此石听了,不觉打动凡心,也想要到人间去享一享这荣华富贵,但自恨粗蠢,不得已,便口吐人言,向那僧道说道:“大师,弟子蠢物,不能见礼了。适闻二位谈那人世间荣耀繁华,心切慕之。弟子质虽粗蠢,性却稍通,况见二师仙形道体,定非凡品,必有补天济世之材,利物济人之德。如蒙发一点慈心,携带弟子得入红尘,在那富贵场中,温柔乡里受享几年,自当永佩洪恩,万劫不忘也。”二仙师听毕,齐憨笑道:“善哉,善哉!那红尘中有却有些乐事,但不能永远依恃,况又有`美中不足,好事多魔-八个字紧相连属,瞬息间则又乐极悲生,人非物换,究竟是到头一梦,万境归空,倒不如不去的好。”这石凡心已炽,那里听得进这话去,乃复苦求再四。二仙知不可强制,乃叹道:“此亦静极怂级*,无中生有之数也。既如此,我们便携你去受享受享,只是到不得意时,切莫后悔。”石道:“自然,自然。”那僧又道:“若说你性灵,却又如此质蠢,并更无奇贵之处。如此也只好踮脚而已。也罢,我如今大施佛法助你助,待劫终之日,复还本质,以了此案。你道好否?"石头听了,感谢不尽。那僧便念咒书符,大展幻术,将一块大石登时变成一块鲜明莹洁的美玉,且又缩成扇坠大小的可佩可拿。那僧托于掌上,笑道:“形体倒也是个宝物了!还只没有,实在的好处,须得再镌上数字,使人一见便知是奇物方妙。然后携你到那昌明隆盛之邦,诗礼簪缨之族,花柳繁华地,温柔富贵乡去安身乐业。”石头听了,喜不能禁,乃问:“不知赐了弟子那几件奇处,又不知携了弟子到何地方?望乞明示,使弟子不惑。”那僧笑道:“你且莫问,日后自然明白的。”说着,便袖了这石,同那道人飘然而去,竟不知投奔何方何舍。

后来,又不知过了几世几劫,因有个空空道人访道求仙,忽从这大荒山无稽崖青埂峰下经过,忽见一大块石上字迹分明,编述历历。空空道人乃从头一看,原来就是无材补天,幻形入世,蒙茫茫大士,渺渺真人携入红尘,历尽离合悲欢炎凉世态的一段故事。后面又有一首偈云:

无材可去补苍天,枉入红尘若许年。

此系身前身后事,倩谁记去作奇传?诗后便是此石坠落之乡,投胎之处,亲自经历的一段陈迹故事。其中家庭闺阁琐事,以及闲情诗词倒还全备,或可适趣解闷,然朝代年纪,地舆邦国,却反失落无考。

空空道人遂向石头说道:“石兄,你这一段故事,据你自己说有些趣味,故编写在此,意欲问世传奇。据我看来,第一件,无朝代年纪可考,第二件,并无大贤大忠理朝廷治风俗的善政,其中只不过几个异样女子,或情或痴,或小才微善,亦无班姑,蔡女之德能。我纵抄去,恐世人不爱看呢。”石头笑答道:“我师何太痴耶!若云无朝代可考,今我师竟假借汉唐等年纪添缀,又有何难?但我想,历来野史,皆蹈一辙,莫如我这不借此套者,反倒新奇别致,不过只取其事体情理罢了,又何必拘拘于朝代年纪哉!再者,市井俗人喜看理治之书者甚少,爱适趣闲文者特多。历来野史,或讪谤君相,或贬人妻女,奸滢凶恶,不可胜数。更有一种风月笔墨,其滢秽污臭,屠毒笔墨,坏人子弟,又不可胜数。至若佳人才子等书,则又千部共出一套,且其中终不能不涉于滢滥,以致满纸潘安,子建,西子,文君,不过作者要写出自己的那两首情诗艳赋来,故假拟出男女二人名姓,又必旁出一小人其间拨乱,亦如剧中之小丑然。且鬟婢开口即者也之乎,非文即理。故逐一看去,悉皆自相矛盾,大不近情理之话,竟不如我半世亲睹亲闻的这几个女子,虽不敢说强似前代书中所有之人,但事迹原委,亦可以消愁破闷,也有几首歪诗熟话,可以喷饭供酒。至若离合悲欢,兴衰际遇,则又追踪蹑迹,不敢稍加穿凿,徒为供人之目而反失其真传者。今之人,贫者日为衣食所累,富者又怀不足之心,纵然一时稍闲,又有贪滢恋色,好货寻愁之事,那里去有工夫看那理治之书?所以我这一段故事,也不愿世人称奇道妙,也不定要世人喜悦检读,只愿他们当那醉滢饱卧之时,或避世去愁之际,把此一玩,岂不省了些寿命筋力?就比那谋虚逐妄,却也省了口舌是非之害,腿脚奔忙之苦。再者,亦令世人换新眼目,不比那些胡牵乱扯,忽离忽遇,满纸才人淑女,子建文君红娘小玉等通共熟套之旧稿。我师意为何如?”

空空道人听如此说,思忖半晌,将《石头记》再检阅一遍,因见上面虽有些指奸责佞贬恶诛邪之语,亦非伤时骂世之旨,及至君仁臣良父慈子孝,凡轮常所关之处,皆是称功颂德,眷眷无穷,实非别书之可比。虽其中大旨谈情,亦不过实录其事,又非假拟妄称,一味滢邀艳约,私订偷盟之可比。因毫不干涉时世,方从头至尾抄录回来,问世传奇。从此空空道人因空见色,由色生情,传情入色,自色悟空,遂易名为情僧,改《石头记》为《情僧录》。东鲁孔梅溪则题曰《风月宝鉴》。后因曹雪芹于悼红轩中披阅十载,增删五次,纂成目录,分出章回,则题曰《金陵十二钗》。并题一绝云:

满纸荒唐言,一把辛酸泪!

都云作者痴,谁解其中味?

出则既明,且看石上是何故事。按那石上书云:

当日地陷东南,这东南一隅有处曰姑苏,有城曰阊门者,最是红尘中一二等富贵风流之地。这阊门外有个十里街,街内有个仁清巷,巷内有个古庙,因地方窄狭,人皆呼作葫芦庙。庙旁住着一家乡宦,姓甄,名费,字士隐。嫡妻封氏,情性贤淑,深明礼义。家中虽不甚富贵,然本地便也推他为望族了。因这甄士隐禀性恬淡,不以功名为念,每日只以观花修竹,酌酒吟诗为乐,倒是神仙一流人品。只是一件不足:如今年已半百,膝下无儿,只有一女,侞名唤作英莲,年方三岁。

一日,炎夏永昼,士隐于书房闲坐,至手倦抛书,伏几少憩,不觉朦胧睡去。梦至一处,不辨是何地方。忽见那厢来了一僧一道,且行且谈。只听道人问道:“你携了这蠢物,意欲何往?"那僧笑道:“你放心,如今现有一段风流公案正该了结,这一干风流冤家,尚未投胎入世。趁此机会,就将此蠢物夹带于中,使他去经历经历。”那道人道:“原来近日风流冤孽又将造劫历世去不成?但不知落于何方何处?"那僧笑道:“此事说来好笑,竟是千古未闻的罕事。只因西方灵河岸上三生石畔,有绛珠草一株,时有赤瑕宫神瑛侍者,日以甘露灌溉,这绛珠草始得久延岁月。后来既受天地精华,复得雨露滋养,遂得脱却草胎木质,得换人形,仅修成个女体,终日游于离恨天外,饥则食蜜青果为膳,渴则饮灌愁海水为汤。只因尚未酬报灌溉之德,故其五内便郁结着一段缠绵不尽之意。恰近日这神瑛侍者凡心偶炽,乘此昌明太平朝世,意欲下凡造历幻缘,已在警幻仙子案前挂了号。警幻亦曾问及,灌溉之情未偿,趁此倒可了结的。那绛珠仙子道:`他是甘露之惠,我并无此水可还。他既下世为人,我也去下世为人,但把我一生所有的眼泪还他,也偿还得过他了。-因此一事,就勾出多少风流冤家来,陪他们去了结此案。”那道人道:“果是罕闻。实未闻有还泪之说。想来这一段故事,比历来风月事故更加琐碎细腻了。”那僧道:“历来几个风流人物,不过传其大概以及诗词篇章而已,至家庭闺阁中一饮一食,总未述记。再者,大半风月故事,不过偷香窃玉,暗约私奔而已,并不曾将儿女之真情发泄一二。想这一干人入世,其情痴色鬼,贤愚不肖者,悉与前人传述不同矣。”那道人道:“趁此何不你我也去下世度脱几个,岂不是一场功德?"那僧道:“正合吾意,你且同我到警幻仙子宫中,将蠢物交割清楚,待这一干风流孽鬼下世已完,你我再去。如今虽已有一半落尘,然犹未全集。”道人道:“既如此,便随你去来。”

却说甄士隐俱听得明白,但不知所云"蠢物"系何东西。遂不禁上前施礼,笑问道:“二仙师请了。”那僧道也忙答礼相问。士隐因说道:“适闻仙师所谈因果,实人世罕闻者。但弟子愚浊,不能洞悉明白,若蒙大开痴顽,备细一闻,弟子则洗耳谛听,稍能警省,亦可免沉轮之苦。”二仙笑道:“此乃玄机不可预泄者。到那时不要忘我二人,便可跳出火坑矣。”士隐听了,不便再问。因笑道:“玄机不可预泄,但适云`蠢物-,不知为何,或可一见否?"那僧道:“若问此物,倒有一面之缘。”说着,取出递与士隐。士隐接了看时,原来是块鲜明美玉,上面字迹分明,镌着"通灵宝玉"四字,后面还有几行小字。正欲细看时,那僧便说已到幻境,便强从手中夺了去,与道人竟过一大石牌坊,上书四个大字,乃是"太虚幻境"。两边又有一幅对联,道是:

假作真时真亦假,无为有处有还无。士隐意欲也跟了过去,方举步时,忽听一声霹雳,有若山崩地陷。士隐大叫一声,定睛一看,只见烈日炎炎,芭蕉冉冉,所梦之事便忘了大半。又见奶母正抱了英莲走来。士隐见女儿越发生得粉妆玉琢,乖觉可喜,便伸手接来,抱在怀内,斗他顽耍一回,又带至街前,看那过会的热闹。方欲进来时,只见从那边来了一僧一道:那僧则癞头跣脚,那道则跛足蓬头,疯疯癫癫,挥霍谈笑而至。及至到了他门前,看见士隐抱着英莲,那僧便大哭起来,又向士隐道:“施主,你把这有命无运,累及爹娘之物,抱在怀内作甚?"士隐听了,知是疯话,也不去睬他。那僧还说:“舍我罢,舍我罢!"士隐不耐烦,便抱女儿撤身要进去,那僧乃指着他大笑,口内念了四句言词道:

惯养娇生笑你痴,菱花空对雪澌澌。

好防佳节元宵后,便是烟消火灭时。士隐听得明白,心下犹豫,意欲问他们来历。只听道人说道:“你我不必同行,就此分手,各干营生去罢。三劫后,我在北邙山等你,会齐了同往太虚幻境销号。”那僧道:“最妙,最妙!"说毕,二人一去,再不见个踪影了。士隐心中此时自忖:这两个人必有来历,该试一问,如今悔却晚也。

这士隐正痴想,忽见隔壁葫芦庙内寄居的一个穷儒-姓贾名化,表字时飞,别号雨村者走了出来。这贾雨村原系胡州人氏,也是诗书仕宦之族,因他生于末世,父母祖宗根基已尽,人口衰丧,只剩得他一身一口,在家乡无益,因进京求取功名,再整基业。自前岁来此,又淹蹇住了,暂寄庙中安身,每日卖字作文为生,故士隐常与他交接。当下雨村见了士隐,忙施礼陪笑道:“老先生倚门伫望,敢是街市上有甚新闻否?"士隐笑道:“非也。适因小女啼哭,引他出来作耍,正是无聊之甚,兄来得正妙,请入小斋一谈,彼此皆可消此永昼。”说着,便令人送女儿进去,自与雨村携手来至书房中。小童献茶。方谈得三五句话,忽家人飞报:“严老爷来拜。”士隐慌的忙起身谢罪道:“恕诳驾之罪,略坐,弟即来陪。”雨村忙起身亦让道:“老先生请便。晚生乃常造之客,稍候何妨。”说着,士隐已出前厅去了。

这里雨村且翻弄书籍解闷。忽听得窗外有女子嗽声,雨村遂起身往窗外一看,原来是一个丫鬟,在那里撷花,生得仪容不俗,眉目清明,虽无十分姿色,却亦有动人之处。雨村不觉看的呆了。那甄家丫鬟撷了花,方欲走时,猛抬头见窗内有人,敝巾旧服,虽是贫窘,然生得腰圆背厚,面阔口方,更兼剑眉星眼,直鼻权腮。这丫鬟忙转身回避,心下乃想:“这人生的这样雄壮,却又这样褴褛,想他定是我家主人常说的什么贾雨村了,每有意帮助周济,只是没甚机会。我家并无这样贫窘亲友,想定是此人无疑了。怪道又说他必非久困之人。”如此想来,不免又回头两次。雨村见他回了头,便自为这女子心中有意于他,便狂喜不尽,自为此女子必是个巨眼英雄,风尘中之知己也。一时小童进来,雨村打听得前面留饭,不可久待,遂从夹道中自便出门去了。士隐待客既散,知雨村自便,也不去再邀。

一日,早又中秋佳节。士隐家宴已毕,乃又另具一席于书房,却自己步月至庙中来邀雨村。原来雨村自那日见了甄家之婢曾回顾他两次,自为是个知己,便时刻放在心上。今又正值中秋,不免对月有怀,因而口占五言一律云:

未卜三生愿,频添一段愁。

闷来时敛额,行去几回头。

自顾风前影,谁堪月下俦?

蟾光如有意,先上玉人楼。雨村吟罢,因又思及平生抱负,苦未逢时,乃又搔首对天长叹,复高吟一联曰:

玉在中求善价,钗于奁内待时飞。恰值士隐走来听见,笑道:“雨村兄真抱负不浅也!"雨村忙笑道:“不过偶吟前人之句,何敢狂诞至此。”因问:“老先生何兴至此?"士隐笑道:“今夜中秋,俗谓`团圆之节-,想尊兄旅寄僧房,不无寂寥之感,故特具小酌,邀兄到敝斋一饮,不知可纳芹意否?"雨村听了,并不推辞,便笑道:“既蒙厚爱,何敢拂此盛情。”说着,便同士隐复过这边书院中来。须臾茶毕,早已设下杯盘,那美酒佳肴自不必说。二人归坐,先是款斟漫饮,次渐谈至兴浓,不觉飞觥限起来。当时街坊上家家箫管,户户弦歌,当头一轮明月,飞彩凝辉,二人愈添豪兴,酒到杯干。雨村此时已有七八分酒意,狂兴不禁,乃对月寓怀,口号一绝云:

时逢三五便团圆,满把晴光护玉栏。

天上一轮才捧出,人间万姓仰头看。士隐听了,大叫:“妙哉!吾每谓兄必非久居人下者,今所吟之句,飞腾之兆已见,不日可接履于云霓之上矣。可贺,可贺!"乃亲斟一斗为贺。雨村因干过,叹道:“非晚生酒后狂言,若论时尚之学,晚生也或可去充数沽名,只是目今行囊路费一概无措,神京路远,非赖卖字撰文即能到者。”士隐不待说完,便道:“兄何不早言。愚每有此心,但每遇兄时,兄并未谈及,愚故未敢唐突。今既及此,愚虽不才,`义利-二字却还识得。且喜明岁正当大比,兄宜作速入都,春闱一战,方不负兄之所学也。其盘费余事,弟自代为处置,亦不枉兄之谬识矣!"当下即命小童进去,速封五十两白银,并两套冬衣。又云:“十九日乃黄道之期,兄可即买舟西上,待雄飞高举,明冬再晤,岂非大快之事耶!"雨村收了银衣,不过略谢一语,并不介意,仍是吃酒谈笑。那天已交了三更,二人方散。士隐送雨村去后,回房一觉,直至红日三竿方醒。因思昨夜之事,意欲再写两封荐书与雨村带至神都,使雨村投谒个仕宦之家为寄足之地。因使人过去请时,那家人去了回来说:“和尚说,贾爷今日五鼓已进京去了,也曾留下话与和尚转达老爷,说`读书人不在黄道黑道,总以事理为要,不及面辞了。-"士隐听了,也只得罢了。真是闲处光陰易过,倏忽又是元霄佳节矣。士隐命家人霍启抱了英莲去看社火花灯,半夜中,霍启因要小解,便将英莲放在一家门槛上坐着。待他小解完了来抱时,那有英莲的踪影?急得霍启直寻了半夜,至天明不见,那霍启也就不敢回来见主人,便逃往他乡去了。那士隐夫妇,见女儿一夜不归,便知有些不妥,再使几人去寻找,回来皆云连音响皆无。夫妻二人,半世只生此女,一旦失落,岂不思想,因此昼夜啼哭,几乎不曾寻死。看看的一月,士隐先就得了一病,当时封氏孺人也因思女构疾,日日请医疗治。

不想这日三月十五,葫芦庙中炸供,那些和尚不加小心,致使油锅火逸,便烧着窗纸。此方人家多用竹篱木壁者,大抵也因劫数,于是接二连三,牵五挂四,将一条街烧得如火焰山一般。彼时虽有军民来救,那火已成了势,如何救得下?直烧了一夜,方渐渐的熄去,也不知烧了几家。只可怜甄家在隔壁,早已烧成一片瓦砾场了。只有他夫妇并几个家人的性命不曾伤了。急得士隐惟跌足长叹而已。只得与妻子商议,且到田庄上去安身。偏值近年水旱不收,鼠盗蜂起,无非抢田夺地,鼠窃狗偷,民不安生,因此官兵剿捕,难以安身。士隐只得将田庄都折变了,便携了妻子与两个丫鬟投他岳丈家去。

他岳丈名唤封肃,本贯大如州人氏,虽是务农,家中都还殷实。今见女婿这等狼狈而来,心中便有些不乐。幸而士隐还有折变田地的银子未曾用完,拿出来托他随分就价薄置些须房地,为后日衣食之计。那封肃便半哄半赚,些须与他些薄田朽屋。士隐乃读书之人,不惯生理稼穑等事,勉强支持了一二年,越觉穷了下去。封肃每见面时,便说些现成话,且人前人后又怨他们不善过活,只一味好吃懒作等语。士隐知投人不着,心中未免悔恨,再兼上年惊唬,急忿怨痛,已有积伤,暮年之人,贫病交攻,竟渐渐的露出那下世的光景来。

可巧这日拄了拐杖挣挫到街前散散心时,忽见那边来了一个跛足道人,疯癫落脱,麻屣鹑衣,口内念着几句言词,道是:

世人都晓神仙好,惟有功名忘不了!

古今将相在何方?荒冢一堆草没了。

世人都晓神仙好,只有金银忘不了!

终朝只恨聚无多,及到多时眼闭了。

世人都晓神仙好,只有姣妻忘不了!

君生日日说恩情,君死又随人去了。

世人都晓神仙好,只有儿孙忘不了!

痴心父母古来多,孝顺儿孙谁见了?士隐听了,便迎上来道:“你满口说些什么?只听见些`好-`了-`好-`了-。那道人笑道:“你若果听见`好-`了-二字,还算你明白。可知世上万般,好便是了,了便是好。若不了,便不好,若要好,须是了。我这歌儿,便名《好了歌》"士隐本是有宿慧的,一闻此言,心中早已彻悟。因笑道:“且住!待我将你这《好了歌》解注出来何如?"道人笑道:“你解,你解。”士隐乃说道:

陋室空堂,当年笏满床,衰草枯杨,曾为歌舞场。蛛丝儿结满雕梁,绿纱今又糊在蓬窗上。说什么脂正浓,粉正香,如何两鬓又成霜?昨日黄土陇头送白骨,今宵红灯帐底卧鸳鸯。金满箱,银满箱,展眼乞丐人皆谤。正叹他人命不长,那知自己归来丧!训有方,保不定日后作强梁。择膏粱,谁承望流落在烟花巷!因嫌纱帽小,致使锁枷杠,昨怜破袄寒,今嫌紫蟒长:乱烘烘你方唱罢我登场,反认他乡是故乡。甚荒唐,到头来都是为他人作嫁衣裳!那疯跛道人听了,拍掌笑道:“解得切,解得切!"士隐便说一声"走罢!"将道人肩上褡裢抢了过来背着,竟不回家,同了疯道人飘飘而去。当下烘动街坊,众人当作一件新闻传说。封氏闻得此信,哭个死去活来,只得与父亲商议,遣人各处访寻,那讨音信?无奈何,少不得依靠着他父母度日。幸而身边还有两个旧日的丫鬟伏侍,主仆三人,日夜作些针线发卖,帮着父亲用度。那封肃虽然日日抱怨,也无可奈何了。

这日,那甄家大丫鬟在门前买线,忽听街上喝道之声,众人都说新太爷到任。丫鬟于是隐在门内看时,只见军牢快手,一对一对的过去,俄而大轿抬着一个乌帽猩袍的官府过去。丫鬟倒发了个怔,自思这官好面善,倒象在那里见过的。于是进入房中,也就丢过不在心上。至晚间,正待歇息之时,忽听一片声打的门响,许多人乱嚷,说:“本府太爷差人来传人问话。”封肃听了,唬得目瞪口呆,不知有何祸事。

\chapter{贾夫人仙逝扬州城\ttlbreak 冷子兴演说荣国府}

诗云

一局输赢料不真,香销茶尽尚逡巡。欲知目下兴衰兆,须问旁观冷眼人。

却说封肃因听见公差传唤,忙出来陪笑启问。那些人只嚷:“快请出甄爷来!"封肃忙陪笑道:“小人姓封,并不姓甄。只有当日小婿姓甄,今已出家一二年了,不知可是问他?"那些公人道:“我们也不知什么`真-`假-,因奉太爷之命来问,他既是你女婿,便带了你去亲见太爷面禀,省得乱跑。”说着,不容封肃多言,大家推拥他去了。封家人个个都惊慌,不知何兆。

那天约二更时,只见封肃方回来,欢天喜地。众人忙问端的。他乃说道:“原来本府新升的太爷姓贾名化,本贯胡州人氏,曾与女婿旧日相交。方才在咱门前过去,因见娇杏那丫头买线,所以他只当女婿移住于此。我一一将原故回明,那太爷倒伤感叹息了一回,又问外孙女儿,我说看灯丢了。太爷说:`不妨,我自使番役务必探访回来。-说了一回话,临走倒送了我二两银子。”甄家娘子听了,不免心中伤感。一宿无话。至次日,早有雨村遣人送了两封银子,四匹锦缎,答谢甄家娘子,又寄一封密书与封肃,转托问甄家娘子要那娇杏作二房。封肃喜的屁滚尿流,巴不得去奉承,便在女儿前一力撺掇成了,乘夜只用一乘小轿,便把娇杏送进去了。雨村欢喜,自不必说,乃封百金赠封肃,外谢甄家娘子许多物事,令其好生养赡,以待寻访女儿下落。封肃回家无话。

却说娇杏这丫鬟,便是那年回顾雨村者。因偶然一顾,便弄出这段事来,亦是自己意料不到之奇缘。谁想他命运两济,不承望自到雨村身边,只一年便生了一子,又半载,雨村嫡妻忽染疾下世,雨村便将他扶侧作正室夫人了。正是:

偶因一着错,便为人上人。原来,雨村因那年士隐赠银之后,他于十六日便起身入都,至大比之期,不料他十分得意,已会了进士,选入外班,今已升了本府知府。虽才干优长,未免有些贪酷之弊,且又恃才侮上,那些官员皆侧目而视。不上一年,便被上司寻了个空隙,作成一本,参他生情狡猾,擅纂礼仪,大怒,即批革职。该部文书一到,本府官员无不喜悦。那雨村心中虽十分惭恨,却面上全无一点怨色,仍是嘻笑自若,交代过公事,将历年做官积的些资本并家小人属送至原籍,安排妥协,却是自己担风袖月,游览天下胜迹。

那日,偶又游至维扬地面,因闻得今岁鹾政点的是林如海。这林如海姓林名海,表字如海,乃是前科的探花,今已升至兰台寺大夫,本贯姑苏人氏,今钦点出为巡盐御史,到任方一月有余。原来这林如海之祖,曾袭过列侯,今到如海,业经五世。起初时,只封袭三世,因当今隆恩盛德,远迈前代,额外加恩,至如海之父,又袭了一代;至如海,便从科第出身。虽系钟鼎之家,却亦是书香之族。只可惜这林家支庶不盛,子孙有限,虽有几门,却与如海俱是堂族而已,没甚亲支嫡派的。今如海年已四十,只有一个三岁之子,偏又于去岁死了。虽有几房姬妾,奈他命中无子,亦无可如何之事。今只有嫡妻贾氏,生得一女,侞名黛玉,年方五岁。夫妻无子,故爱如珍宝,且又见他聪明清秀,便也欲使他读书识得几个字,不过假充养子之意,聊解膝下荒凉之叹。

雨村正值偶感风寒,病在旅店,将一月光景方渐愈。一因身体劳倦,二因盘费不继,也正欲寻个合式之处,暂且歇下。幸有两个旧友,亦在此境居住,因闻得鹾政欲聘一西宾,雨村便相托友力,谋了进去,且作安身之计。妙在只一个女学生,并两个伴读丫鬟,这女学生年又小,身体又极怯弱,工课不限多寡,故十分省力。堪堪又是一载的光陰,谁知女学生之母贾氏夫人一疾而终。女学生侍汤奉药,守丧尽哀,遂又将辞馆别图。林如海意欲令女守制读书,故又将他留下。近因女学生哀痛过伤,本自怯弱多病的,触犯旧症,遂连日不曾上学。雨村闲居无聊,每当风日晴和,饭后便出来闲步。

这日,偶至郭外,意欲赏鉴那村野风光。忽信步至一山环水旋,茂林深竹之处,隐隐的有座庙宇,门巷倾颓,墙垣朽败,门前有额,题着"智通寺"三字,门旁又有一副旧破的对联,曰

身后有余忘缩手,眼前无路想回头。雨村看了,因想到:“这两句话,文虽浅近,其意则深。我也曾游过些名山大刹,倒不曾见过这话头,其中想必有个翻过筋斗来的亦未可知,何不进去试试。”想着走入,只有一个龙钟老僧在那里煮粥。雨村见了,便不在意。及至问他两句话,那老僧既聋且昏,齿落舌钝,所答非所问。

雨村不耐烦,便仍出来,意欲到那村肆中沽饮三杯,以助野趣,于是款步行来。将入肆门,只见座上吃酒之客有一人起身大笑,接了出来,口内说:“奇遇,奇遇。”雨村忙看时,此人是都中在古董行中贸易的号冷子兴者,旧日在都相识。雨村最赞这冷子兴是个有作为大本领的人,这子兴又借雨村斯文之名,故二人说话投机,最相契合。雨村忙笑问道:“老兄何日到此?弟竟不知。今日偶遇,真奇缘也。”子兴道:“去年岁底到家,今因还要入都,从此顺路找个敝友说一句话,承他之情,留我多住两日。我也无紧事,且盘桓两日,待月半时也就起身了。今日敝友有事,我因闲步至此,且歇歇脚,不期这样巧遇!"一面说,一面让雨村同席坐了,另整上酒肴来。二人闲谈漫饮,叙些别后之事。

雨村因问:“近日都中可有新闻没有?"子兴道:“倒没有什么新闻,倒是老先生你贵同宗家,出了一件小小的异事。”雨村笑道:“弟族中无人在都,何谈及此?"子兴笑道:“你们同姓,岂非同宗一族?"雨村问是谁家。子兴道:“荣国府贾府中,可也玷辱了先生的门楣么?"雨村笑道:“原来是他家。若论起来,寒族人丁却不少,自东汉贾复以来,支派繁盛,各省皆有,谁逐细考查得来?若论荣国一支,却是同谱。但他那等荣耀,我们不便去攀扯,至今故越发生疏难认了。”子兴叹道:“老先生休如此说。如今的这宁荣两门,也都萧疏了,不比先时的光景。”雨村道:“当日宁荣两宅的人口也极多,如何就萧疏了?"冷子兴道:“正是,说来也话长。”雨村道:“去岁我到金陵地界,因欲游览六朝遗迹,那日进了石头城,从他老宅门前经过。街东是宁国府,街西是荣国府,二宅相连,竟将大半条街占了。大门前虽冷落无人,隔着围墙一望,里面厅殿楼阁,也还都峥嵘轩峻,就是后一带花园子里面树木山石,也还都有蓊蔚洇润之气,那里象个衰败之家?"冷子兴笑道:“亏你是进士出身,原来不通!古人有云:`百足之虫,死而不僵。-如今虽说不及先年那样兴盛,较之平常仕宦之家,到底气象不同。如今生齿日繁,事务日盛,主仆上下,安富尊荣者尽多,运筹谋画者无一,其日用排场费用,又不能将就省俭,如今外面的架子虽未甚倒,内囊却也尽上来了。这还是小事。更有一件大事:谁知这样钟鸣鼎食之家,翰墨诗书之族,如今的儿孙,竟一代不如一代了!"雨村听说,也纳罕道:“这样诗礼之家,岂有不善教育之理?别门不知,只说这宁,荣二宅,是最教子有方的。”

子兴叹道:“正说的是这两门呢。待我告诉你:当日宁国公与荣国公是一母同胞弟兄两个。宁公居长,生了四个儿子。宁公死后,贾代化袭了官,也养了两个儿子:长名贾敷,至八九岁上便死了,只剩了次子贾敬袭了官,如今一味好道,只爱烧丹炼汞,余者一概不在心上。幸而早年留下一子,名唤贾珍,因他父亲一心想作神仙,把官倒让他袭了。他父亲又不肯回原籍来,只在都中城外和道士们胡羼。这位珍爷倒生了一个儿子,今年才十六岁,名叫贾蓉。如今敬老爹一概不管。这珍爷那里肯读书,只一味高乐不了,把宁国府竟翻了过来,也没有人敢来管他。再说荣府你听,方才所说异事,就出在这里。自荣公死后,长子贾代善袭了官,娶的也是金陵世勋史侯家的小姐为妻,生了两个儿子:长子贾赦,次子贾政。如今代善早已去世,太夫人尚在,长子贾赦袭着官,次子贾政,自幼酷喜捕潦*,祖父最疼,原欲以科甲出身的,不料代善临终时遗本一上,皇上因恤先臣,即时令长子袭官外,问还有几子,立刻引见,遂额外赐了这政老爹一个主事之衔,令其入部习学,如今现已升了员外郎了。这政老爹的夫人王氏,头胎生的公子,名唤贾珠,十四岁进学,不到二十岁就娶了妻生了子,一病死了。第二胎生了一位小姐,生在大年初一,这就奇了,不想后来又生一位公子,说来更奇,一落胎胞,嘴里便衔下一块五彩晶莹的玉来,上面还有许多字迹,就取名叫作宝玉。你道是新奇异事不是?”

雨村笑道:“果然奇异。只怕这人来历不小。”子兴冷笑道:“万人皆如此说,因而乃祖母便先爱如珍宝。那年周岁时,政老爹便要试他将来的志向,便将那世上所有之物摆了无数,与他抓取。谁知他一概不取,伸手只把些脂粉钗环抓来。政老爹便大怒了,说:“`将来酒色之徒耳!-因此便大不喜悦。独那史老太君还是命根一样。说来又奇,如今长了七八岁,虽然淘气异常,但其聪明乖觉处,百个不及他一个。说起孩子话来也奇怪,他说:`女儿是水作的骨肉,男人是泥作的骨肉。我见了女儿,我便清爽,见了男子,便觉浊臭逼人。-你道好笑不好笑?将来色鬼无疑了!"雨村罕然厉色忙止道:“非也!可惜你们不知道这人来历。大约政老前辈也错以滢魔色鬼看待了。若非多读书识事,加以致知格物之功,悟道参玄之力,不能知也。”

子兴见他说得这样重大,忙请教其端。雨村道:“天地生人,除大仁大恶两种,余者皆无大异。若大仁者,则应运而生,大恶者,则应劫而生。运生世治,劫生世危。尧,舜,禹,汤,文,武,周,召,孔,孟,董,韩,周,程,张,朱,皆应运而生者。蚩尤,共工,桀,纣,始皇,王莽,曹躁,桓温,安禄山,秦桧等,皆应劫而生者。大仁者,修治天下,大恶者,挠乱天下。清明灵秀,天地之正气,仁者之所秉也,残忍乖僻,天地之邪气,恶者之所秉也。今当运隆祚永之朝,太平无为之世,清明灵秀之气所秉者,上至朝廷,下及草野,比比皆是。所余之秀气,漫无所归,遂为甘露,为和风,洽然溉及四海。彼残忍乖僻之邪气,不能荡溢于光天化日之中,遂凝结充塞于深沟大壑之内,偶因风荡,或被云催,略有摇动感发之意,一丝半缕误而泄出者,偶值灵秀之气适过,正不容邪,邪复妒正,两不相下,亦如风水雷电,地中既遇,既不能消,又不能让,必至搏击掀发后始尽。故其气亦必赋人,发泄一尽始散。使男女偶秉此气而生者,在上则不能成仁人君子,下亦不能为大凶大恶。置之于万万人中,其聪俊灵秀之气,则在万万人之上,其乖僻邪谬不近人情之态,又在万万人之下。若生于公侯富贵之家,则为情痴情种,若生于诗书清贫之族,则为逸士高人,纵再偶生于薄祚寒门,断不能为走卒健仆,甘遭庸人驱制驾驭,必为奇优名倡。如前代之许由,陶潜,阮籍,嵇康,刘伶,王谢二族,顾虎头,陈后主,唐明皇,宋徽宗,刘庭芝,温飞卿,米南宫,石曼卿,柳耆卿,秦少游,近日之倪云林,唐伯虎,祝枝山,再如李龟年,黄幡绰,敬新磨,卓文君,红拂,薛涛,崔莺,朝云之流,此皆易地则同之人也。”

子兴道:“依你说,`成则王侯败则贼了。-"雨村道:“正是这意。你还不知,我自革职以来,这两年遍游各省,也曾遇见两个异样孩子。所以,方才你一说这宝玉,我就猜着了八九亦是这一派人物。不用远说,只金陵城内,钦差金陵省体仁院总裁甄家,你可知么?"子兴道:“谁人不知!这甄府和贾府就是老亲,又系世交。两家来往,极其亲热的。便在下也和他家来往非止一日了。”

雨村笑道:“去岁我在金陵,也曾有人荐我到甄府处馆。我进去看其光景,谁知他家那等显贵,却是个富而好礼之家,倒是个难得之馆。但这一个学生,虽是启蒙,却比一个举业的还劳神。说起来更可笑,他说:`必得两个女儿伴着我读书,我方能认得字,心里也明白,不然我自己心里糊涂。-又常对跟他的小厮们说:`这女儿两个字,极尊贵,极清净的,比那阿弥陀佛,元始天尊的这两个宝号还更尊荣无对的呢!你们这浊口臭舌,万不可唐突了这两个字,要紧。但凡要说时,必须先用清水香茶漱了口才可,设若失错,便要凿牙穿腮等事。-其暴虐浮躁,顽劣憨痴,种种异常。只一放了学,进去见了那些女儿们,其温厚和平,聪敏文雅,竟又变了一个。因此,他令尊也曾下死笞楚过几次,无奈竟不能改。每打的吃疼不过时,他便`姐姐-`妹妹-乱叫起来。后来听得里面女儿们拿他取笑:`因何打急了只管叫姐妹做甚?莫不是求姐妹去说情讨饶?你岂不愧些!-他回答的最妙。他说:`急疼之时,只叫`姐姐-妹妹-字样,或可解疼也未可知,因叫了一声,便果觉不疼了,遂得了秘法:每疼痛之极,便连叫姐妹起来了。-你说可笑不可笑?也因祖母溺爱不明,每因孙辱师责子,因此我就辞了馆出来。如今在这巡盐御史林家做馆了。你看,这等子弟,必不能守祖父之根基,从师长之规谏的。只可惜他家几个姊妹都是少有的。”

子兴道:“便是贾府中,现有的三个也不错。政老爹的长女,名元春,现因贤孝才德,选入宫作女史去了。二小姐乃赦老爹之妾所出,名迎春,三小姐乃政老爹之庶出,名探春,四小姐乃宁府珍爷之胞妹,名唤惜春。因史老夫人极爱孙女,都跟在祖母这边一处读书,听得个个不错。雨村道:“更妙在甄家的风俗,女儿之名,亦皆从男子之名命字,不似别家另外用这些`春-`红-`香-`玉-等艳字的。何得贾府亦乐此俗套?"子兴道:“不然。只因现今大小姐是正月初一日所生,故名元春,余者方从了`春-字。上一辈的,却也是从兄弟而来的。现有对证:目今你贵东家林公之夫人,即荣府中赦,政二公之胞妹,在家时名唤贾敏。不信时,你回去细访可知。”雨村拍案笑道:“怪道这女学生读至凡书中有`敏-字,皆念作`密-字,每每如是,写字遇着`敏-字,又减一二笔,我心中就有些疑惑。今听你说的,是为此无疑矣。怪道我这女学生言语举止另是一样,不与近日女子相同,度其母必不凡,方得其女,今知为荣府之孙,又不足罕矣,可伤上月竟亡故了。”子兴叹道:“老姊妹四个,这一个是极小的,又没了。长一辈的姊妹,一个也没了。只看这小一辈的,将来之东床如何呢。”

雨村道:“正是。方才说这政公,已有衔玉之儿,又有长子所遗一个弱孙。这赦老竟无一个不成?"子兴道:“政公既有玉儿之后,其妾又生了一个,倒不知其好歹。只眼前现有二子一孙,却不知将来如何。若问那赦公,也有二子,长名贾琏,今已二十来往了,亲上作亲,娶的就是政老爹夫人王氏之内侄女,今已娶了二年。这位琏爷身上现捐的是个同知,也是不肯读书,于世路上好机变,言谈去的,所以如今只在乃叔政老爷家住着,帮着料理些家务。谁知自娶了他令夫人之后,倒上下无一人不称颂他夫人的,琏爷倒退了一射之地:说模样又极标致,言谈又爽利,心机又极深细,竟是个男人万不及一的。”

雨村听了,笑道:“可知我前言不谬。你我方才所说的这几个人,都只怕是那正邪两赋而来一路之人,未可知也。”子兴道:“邪也罢,正也罢,只顾算别人家的帐,你也吃一杯酒才好。”雨村道:“正是,只顾说话,竟多吃了几杯。”子兴笑道:“说着别人家的闲话,正好下酒,即多吃几杯何妨。”雨村向窗外看道:“天也晚了,仔细关了城。我们慢慢的进城再谈,未为不可。”于是,二人起身,算还酒帐。方欲走时,又听得后面有人叫道:“雨村兄,恭喜了!特来报个喜信的。”雨村忙回头看时-

\chapter{贾雨村夤缘复旧职\ttlbreak 林黛玉抛父进京都}

却说雨村忙回头看时,不是别人,乃是当日同僚一案参革的号张如圭者。他本系此地人,革后家居,今打听得都中奏准起复旧员之信,他便四下里寻情找门路,忽遇见雨村,故忙道喜。二人见了礼,张如圭便将此信告诉雨村,雨村自是欢喜,忙忙的叙了两句,遂作别各自回家。冷子兴听得此言,便忙献计,令雨村央烦林如海,转向都中去央烦贾政。雨村领其意,作别回至馆中,忙寻邸报看真确了。

次日,面谋之如海。如海道:“天缘凑巧,因贱荆去世,都中家岳母念及小女无人依傍教育,前已遣了男女船只来接,因小女未曾大痊,故未及行。此刻正思向蒙训教之恩未经酬报,遇此机会,岂有不尽心图报之理。但请放心。弟已预为筹画至此,已修下荐书一封,转托内兄务为周全协佐,方可稍尽弟之鄙诚,即有所费用之例,弟于内兄信中已注明白,亦不劳尊兄多虑矣。”雨村一面打恭,谢不释口,一面又问:“不知令亲大人现居何职?只怕晚生草率,不敢骤然入都干渎。”如海笑道:“若论舍亲,与尊兄犹系同谱,乃荣公之孙:大内兄现袭一等将军,名赦,字恩侯,二内兄名政,字存周,现任工部员外郎,其为人谦恭厚道,大有祖父遗风,非膏粱轻薄仕宦之流,故弟方致书烦托。否则不但有污尊兄之清躁,即弟亦不屑为矣。”雨村听了,心下方信了昨日子兴之言,于是又谢了林如海。如海乃说:“已择了出月初二日小女入都,尊兄即同路而往,岂不两便?"雨村唯唯听命,心中十分得意。如海遂打点礼物并饯行之事,雨村一一领了。

那女学生黛玉,身体方愈,原不忍弃父而往,无奈他外祖母致意务去,且兼如海说:“汝父年将半百,再无续室之意,且汝多病,年又极小,上无亲母教养,下无姊妹兄弟扶持,今依傍外祖母及舅氏姊妹去,正好减我顾盼之忧,何反云不往?"黛玉听了,方洒泪拜别,随了奶娘及荣府几个老妇人登舟而去。雨村另有一只船,带两个小童,依附黛玉而行。

有日到了都中,进入神京,雨村先整了衣冠,带了小童,拿着宗侄的名帖,至荣府的门前投了。彼时贾政已看了妹丈之书,即忙请入相会。见雨村相貌魁伟,言语不俗,且这贾政最喜读书人,礼贤下士,济弱扶危,大有祖风,况又系妹丈致意,因此优待雨村,更又不同,便竭力内中协助,题奏之日,轻轻谋了一个复职候缺,不上两个月,金陵应天府缺出,便谋补了此缺,拜辞了贾政,择日上任去了。不在话下。

且说黛玉自那日弃舟登岸时,便有荣国府打发了轿子并拉行李的车辆久候了。这林黛玉常听得母亲说过,他外祖母家与别家不同。他近日所见的这几个三等仆妇,吃穿用度,已是不凡了,何况今至其家。因此步步留心,时时在意,不肯轻易多说一句话,多行一步路,惟恐被人耻笑了他去。自上了轿,进入城中从纱窗向外瞧了一瞧,其街市之繁华,人烟之阜盛,自与别处不同。又行了半日,忽见街北蹲着两个大石狮子,三间兽头大门,门前列坐着十来个华冠丽服之人。正门却不开,只有东西两角门有人出入。正门之上有一匾,匾上大书"敕造宁国府"五个大字。黛玉想道:这必是外祖之长房了。想着,又往西行,不多远,照样也是三间大门,方是荣国府了。却不进正门,只进了西边角门。那轿夫抬进去,走了一射之地,将转弯时,便歇下退出去了。后面的婆子们已都下了轿,赶上前来。另换了三四个衣帽周全十七八岁的小厮上来,复抬起轿子。众婆子步下围随至一垂花门前落下。众小厮退出,众婆子上来打起轿帘,扶黛玉下轿。林黛玉扶着婆子的手,进了垂花门,两边是抄手游廊,当中是穿堂,当地放着一个紫檀架子大理石的大插屏。转过插屏,小小的三间厅,厅后就是后面的正房大院。正面五间上房,皆雕梁画栋,两边穿山游廊厢房,挂着各色鹦鹉,画眉等鸟雀。台矶之上,坐着几个穿红着绿的丫头,一见他们来了,便忙都笑迎上来,说:“刚才老太太还念呢,可巧就来了。”于是三四人争着打起帘笼,一面听得人回话:“林姑娘到了。”

黛玉方进入房时,只见两个人搀着一位鬓发如银的老母迎上来,黛玉便知是他外祖母。方欲拜见时,早被他外祖母一把搂入怀中,心肝儿肉叫着大哭起来。当下地下侍立之人,无不掩面涕泣,黛玉也哭个不住。一时众人慢慢解劝住了,黛玉方拜见了外祖母。——此即冷子兴所云之史氏太君,贾赦贾政之母也。当下贾母一一指与黛玉:“这是你大舅母,这是你二舅母,这是你先珠大哥的媳妇珠大嫂子。”黛玉一一拜见过。贾母又说:“请姑娘们来。今日远客才来,可以不必上学去了。”众人答应了一声,便去了两个。

不一时,只见三个奶嬷嬷并五六个丫鬟,簇拥着三个姊妹来了。第一个肌肤微丰,合中身材,腮凝新荔,鼻腻鹅脂,温柔沉默,观之可亲。第二个削肩细腰,长挑身材,鸭蛋脸面,俊眼修眉,顾盼神飞,文彩精华,见之忘俗。第三个身量未足,形容尚小。其钗环裙袄,三人皆是一样的妆饰。黛玉忙起身迎上来见礼,互相厮认过,大家归了坐。丫鬟们斟上茶来。不过说些黛玉之母如何得病,如何请医服药,如何送死发丧。不免贾母又伤感起来,因说:“我这些儿女,所疼者独有你母,今日一旦先舍我而去,连面也不能一见,今见了你,我怎不伤心!"说着,搂了黛玉在怀,又呜咽起来。众人忙都宽慰解释,方略略止住。

众人见黛玉年貌虽小,其举止言谈不俗,身体面庞虽怯弱不胜,却有一段自然的风流态度,便知他有不足之症。因问:“常服何药,如何不急为疗治?"黛玉道:“我自来是如此,从会吃饮食时便吃药,到今日未断,请了多少名医修方配药,皆不见效。那一年我三岁时,听得说来了一个癞头和尚,说要化我去出家,我父母固是不从。他又说:既舍不得他,只怕他的病一生也不能好的了。若要好时,除非从此以后总不许见哭声,除父母之外,凡有外姓亲友之人,一概不见,方可平安了此一世。-疯疯癫癫,说了这些不经之谈,也没人理他。如今还是吃人参养荣丸。”贾母道:“正好,我这里正配丸药呢。叫他们多配一料就是了。

一语未了,只听后院中有人笑声,说:“我来迟了,不曾迎接远客!"黛玉纳罕道:“这些人个个皆敛声屏气,恭肃严整如此,这来者系谁,这样放诞无礼?"心下想时,只见一群媳妇丫鬟围拥着一个人从后房门进来。这个人打扮与众姑娘不同,彩绣辉煌,恍若神妃仙子:头上戴着金丝八宝攒珠髻,绾着朝阳五凤挂珠钗,项上戴着赤金盘螭璎珞圈,裙边系着豆绿宫绦,双衡比目玫瑰佩,身上穿着缕金百蝶穿花大红洋缎窄Ё袄,外罩五彩刻丝石青银鼠褂,下着翡翠撒花洋绉裙。一双丹凤三角眼,两弯柳叶吊梢眉,身量苗条,体格风蚤,粉面含春威不露,丹唇未起笑先闻。黛玉连忙起身接见。贾母笑道,"你不认得他,他是我们这里有名的一个泼皮破落户儿,南省俗谓作`辣子-,你只叫他`凤辣子-就是了。”黛玉正不知以何称呼,只见众姊妹都忙告诉他道:“这是琏嫂子。”黛玉虽不识,也曾听见母亲说过,大舅贾赦之子贾琏,娶的就是二舅母王氏之内侄女,自幼假充男儿教养的,学名王熙凤。黛玉忙陪笑见礼,以"嫂"呼之。这熙凤携着黛玉的手,上下细细打谅了一回,仍送至贾母身边坐下,因笑道:“天下真有这样标致的人物,我今儿才算见了!况且这通身的气派,竟不象老祖宗的外孙女儿,竟是个嫡亲的孙女,怨不得老祖宗天天口头心头一时不忘。只可怜我这妹妹这样命苦,怎么姑妈偏就去世了!"说着,便用帕拭泪。贾母笑道:“我才好了,你倒来招我。你妹妹远路才来,身子又弱,也才劝住了,快再休提前话。”这熙凤听了,忙转悲为喜道:“正是呢!我一见了妹妹,一心都在他身上了,又是喜欢,又是伤心,竟忘记了老祖宗。该打,该打!"又忙携黛玉之手,问:“妹妹几岁了?可也上过学?现吃什么药?在这里不要想家,想要什么吃的,什么玩的,只管告诉我,丫头老婆们不好了,也只管告诉我。”一面又问婆子们:“林姑娘的行李东西可搬进来了?带了几个人来?你们赶早打扫两间下房,让他们去歇歇。”

说话时,已摆了茶果上来。熙凤亲为捧茶捧果。又见二舅母问他:“月钱放过了不曾?"熙凤道:“月钱已放完了。才刚带着人到后楼上找缎子,找了这半日,也并没有见昨日太太说的那样的,想是太太记错了?"王夫人道:“有没有,什么要紧。”因又说道:“该随手拿出两个来给你这妹妹去裁衣裳的,等晚上想着叫人再去拿罢,可别忘了。”熙凤道:“这倒是我先料着了,知道妹妹不过这两日到的,我已预备下了,等太太回去过了目好送来。”王夫人一笑,点头不语。

当下茶果已撤,贾母命两个老嬷嬷带了黛玉去见两个母舅。时贾赦之妻邢氏忙亦起身,笑回道:“我带了外甥女过去,倒也便宜。”贾母笑道:“正是呢,你也去罢,不必过来了。”邢夫人答应了一声"是"字,遂带了黛玉与王夫人作辞,大家送至穿堂前。出了垂花门,早有众小厮们拉过一辆翠幄青н车*,邢夫人携了黛玉,坐在上面,众婆子们放下车帘,方命小厮们抬起,拉至宽处,方驾上驯骡,亦出了西角门,往东过荣府正门,便入一黑油大门中,至仪门前方下来。众小厮退出,方打起车帘,邢夫人搀着黛玉的手,进入院中。黛玉度其房屋院宇,必是荣府中花园隔断过来的。进入三层仪门,果见正房厢庑游廊,悉皆小巧别致,不似方才那边轩峻壮丽,且院中随处之树木山石皆在。一时进入正室,早有许多盛妆丽服之姬妾丫鬟迎着,邢夫人让黛玉坐了,一面命人到外面书房去请贾赦。一时人来回话说:“老爷说了:~连日身上不好,见了姑娘彼此倒伤心,暂且不忍相见。劝姑娘不要伤心想家,跟着老太太和舅母,即同家里一样。姊妹们虽拙,大家一处伴着,亦可以解些烦闷。或有委屈之处,只管说得,不要外道才是。-"黛玉忙站起来,一一听了。再坐一刻,便告辞。邢夫人苦留吃过晚饭去,黛玉笑回道:“舅母爱惜赐饭,原不应辞,只是还要过去拜见二舅舅,恐领了赐去不恭,异日再领,未为不可。望舅母容谅。”邢夫人听说,笑道:“这倒是了。”遂令两三个嬷嬷用方才的车好生送了姑娘过去,于是黛玉告辞。邢夫人送至仪门前,又嘱咐了众人几句,眼看着车去了方回来。

一时黛玉进了荣府,下了车。众嬷嬷引着,便往东转弯,穿过一个东西的穿堂,向南大厅之后,仪门内大院落,上面五间大正房,两边厢房鹿顶耳房钻山,四通八达,轩昂壮丽,比贾母处不同。黛玉便知这方是正经正内室,一条大甬路,直接出大门的。进入堂屋中,抬头迎面先看见一个赤金九龙青地大匾,匾上写着斗大的三个大字,是"荣禧堂",后有一行小字:“某年月日,书赐荣国公贾源",又有"万几宸翰之宝"。大紫檀雕螭案上,设着三尺来高青绿古铜鼎,悬着待漏随朝墨龙大画,一边是金ы彝,一边是玻璃ニ。地下两溜十六张楠木交椅,又有一副对联,乃乌木联牌,镶着錾银的字迹,道是:

座上珠玑昭日月,堂前黼黻焕烟霞。下面一行小字,道是:“同乡世教弟勋袭东安郡王穆莳拜手书"。

原来王夫人时常居坐宴息,亦不在这正室,只在这正室东边的三间耳房内。于是老嬷嬷引黛玉进东房门来。临窗大炕上铺着猩红洋や,正面设着大红金钱蟒靠背,石青金钱蟒引枕,秋香色金钱蟒大条褥。两边设一对梅花式洋漆小几。左边几上文王鼎匙箸香盒,右边几上汝窑美人觚——觚内插着时鲜花卉,并茗碗痰盒等物。地下面西一溜四张椅上,都搭着银红撒花椅搭,底下四副脚踏。椅之两边,也有一对高几,几上茗碗瓶花俱备。其余陈设,自不必细说。老嬷嬷们让黛玉炕上坐,炕沿上却有两个锦褥对设,黛玉度其位次,便不上炕,只向东边椅子上坐了。本房内的丫鬟忙捧上茶来。黛玉一面吃茶,一面打谅这些丫鬟们,妆饰衣裙,举止行动,果亦与别家不同。

茶未吃了,只见一个穿红绫袄青缎掐牙背心的丫鬟走来笑说道:“太太说,请林姑娘到那边坐罢。”老嬷嬷听了,于是又引黛玉出来,到了东廊三间小正房内。正房炕上横设一张炕桌,桌上磊着书籍茶具,靠东壁面西设着半旧的青缎靠背引枕。王夫人却坐在西边下首,亦是半旧的青缎靠背坐褥。见黛玉来了,便往东让。黛玉心中料定这是贾政之位。因见挨炕一溜三张椅子上,也搭着半旧的弹墨椅袱,黛玉便向椅上坐了。王夫人再四携他上炕,他方挨王夫人坐了。王夫人因说:“你舅舅今日斋戒去了,再见罢。只是有一句话嘱咐你:你三个姊妹倒都极好,以后一处念书认字学针线,或是偶一顽笑,都有尽让的。但我不放心的最是一件:我有一个孽根祸胎,是家里的`混世魔王-,今日因庙里还愿去了,尚未回来,晚间你看见便知了。你只以后不要睬他,你这些姊妹都不敢沾惹他的。”

黛玉亦常听得母亲说过,二舅母生的有个表兄,乃衔玉而诞,顽劣异常,极恶读书,最喜在内帏厮混,外祖母又极溺爱,无人敢管。今见王夫人如此说,便知说的是这表兄了。因陪笑道:“舅母说的,可是衔玉所生的这位哥哥?在家时亦曾听见母亲常说,这位哥哥比我大一岁,小名就唤宝玉,虽极憨顽,说在姊妹情中极好的。况我来了,自然只和姊妹同处,兄弟们自是别院另室的,岂得去沾惹之理?"王夫人笑道:“你不知道原故:他与别人不同,自幼因老太太疼爱,原系同姊妹们一处娇养惯了的。若姊妹们有日不理他,他倒还安静些,纵然他没趣,不过出了二门,背地里拿着他两个小幺儿出气,咕唧一会子就完了。若这一日姊妹们和他多说一句话,他心里一乐,便生出多少事来。所以嘱咐你别睬他。他嘴里一时甜言蜜语,一时有天无日,一时又疯疯傻傻,只休信他。”

黛玉一一的都答应着。只见一个丫鬟来回:“老太太那里传晚饭了。”王夫人忙携黛玉从后房门由后廊往西,出了角门,是一条南北宽夹道。南边是倒座三间小小的抱厦厅,北边立着一个粉油大影壁,后有一半大门,小小一所房室。王夫人笑指向黛玉道:“这是你凤姐姐的屋子,回来你好往这里找他来,少什么东西,你只管和他说就是了。”这院门上也有四五个才总角的小厮,都垂手侍立。王夫人遂携黛玉穿过一个东西穿堂,便是贾母的后院了。于是,进入后房门,已有多人在此伺候,见王夫人来了,方安设桌椅。贾珠之妻李氏捧饭,熙凤安箸,王夫人进羹。贾母正面榻上独坐,两边四张空椅,熙凤忙拉了黛玉在左边第一张椅上坐了,黛玉十分推让。贾母笑道:“你舅母你嫂子们不在这里吃饭。你是客,原应如此坐的。”黛玉方告了座,坐了。贾母命王夫人坐了。迎春姊妹三个告了座方上来。迎春便坐右手第一,探春左第二,惜春右第二。旁边丫鬟执着拂尘,漱盂,巾帕。李,凤二人立于案旁布让。外间伺候之媳妇丫鬟虽多,却连一声咳嗽不闻。寂然饭毕,各有丫鬟用小茶盘捧上茶来。当日林如海教女以惜福养身,云饭后务待饭粒咽尽,过一时再吃茶,方不伤脾胃。今黛玉见了这里许多事情不合家中之式,不得不随的,少不得一一改过来,因而接了茶。早见人又捧过漱盂来,黛玉也照样漱了口。プ手毕,又捧上茶来,这方是吃的茶。贾母便说:“你们去罢,让我们自在说话儿。”王夫人听了,忙起身,又说了两句闲话,方引凤,李二人去了。贾母因问黛玉念何书。黛玉道:“只刚念了《四书》。”黛玉又问姊妹们读何书。贾母道:“读的是什么书,不过是认得两个字,不是睁眼的瞎子罢了!”

一语未了,只听外面一阵脚步响,丫鬟进来笑道:“宝玉来了!"黛玉心中正疑惑着:“这个宝玉,不知是怎生个惫懒人物,懵懂顽童?"——倒不见那蠢物也罢了。心中想着,忽见丫鬟话未报完,已进来了一位年轻的公子:头上戴着束发嵌宝紫金冠,齐眉勒着二龙抢珠金抹额,穿一件二色金百蝶穿花大红箭袖,束着五彩丝攒花结长穗宫绦,外罩石青起花八团倭锻排穗褂,登着青缎粉底小朝靴。面若中秋之月,色如春晓之花,鬓若刀裁,眉如墨画,面如桃瓣,目若秋波。虽怒时而若笑,即视而有情。项上金螭璎珞,又有一根五色丝绦,系着一块美玉。黛玉一见,便吃一大惊,心下想道:“好生奇怪,倒象在那里见过一般,何等眼熟到如此!"只见这宝玉向贾母请了安,贾母便命:“去见你娘来。”宝玉即转身去了。一时回来,再看,已换了冠带:头上周围一转的短发,都结成小辫,红丝结束,共攒至顶中胎发,总编一根大辫,黑亮如漆,从顶至梢,一串四颗大珠,用金八宝坠角,身上穿着银红撒花半旧大袄,仍旧带着项圈,宝玉,寄名锁,护身符等物,下面半露松花撒花绫裤腿,锦边弹墨袜,厚底大红鞋。越显得面如敷粉,唇若施脂,转盼多情,语言常笑。天然一段风蚤,全在眉梢,平生万种情思,悉堆眼角。看其外貌最是极好,却难知其底细。后人有《西江月》二词,批宝玉极恰,其词曰:

无故寻愁觅恨,有时似傻如狂。纵然生得好皮囊,腹内

原来草莽。潦倒不通世务,愚顽怕读文章。行为偏僻

性乖张,那管世人诽谤!

富贵不知乐业,贫穷难耐凄凉。可怜辜负好韶光,于国于家无望。天下无能第一,古今不肖无双。寄言纨э

与膏粱:莫效此儿形状!

贾母因笑道:“外客未见,就脱了衣裳,还不去见你妹妹!"宝玉早已看见多了一个姊妹,便料定是林姑妈之女,忙来作揖。厮见毕归坐,细看形容,与众各别:两弯似蹙非蹙ズ烟眉,一双似喜非喜含情目。态生两ь之愁,娇袭一身之病。泪光点点,娇喘微微。闲静时如姣花照水,行动处似弱柳扶风。心较比干多一窍,病如西子胜三分。宝玉看罢,因笑道:“这个妹妹我曾见过的。”贾母笑道:“可又是胡说,你又何曾见过他?"宝玉笑道:“虽然未曾见过他,然我看着面善,心里就算是旧相识,今日只作远别重逢,亦未为不可。”贾母笑道:“更好,更好,若如此,更相和睦了。”宝玉便走近黛玉身边坐下,又细细打量一番,因问:“妹妹可曾读书?"黛玉道:“不曾读,只上了一年学,些须认得几个字。”宝玉又道:“妹妹尊名是那两个字?"黛玉便说了名。宝玉又问表字。黛玉道:“无字。”宝玉笑道:“我送妹妹一妙字,莫若`颦颦-二字极妙。”探春便问何出。宝玉道:“《古今人物通考》上说:`西方有石名黛,可代画眉之墨。-况这林妹妹眉尖若蹙,用取这两个字,岂不两妙!"探春笑道:“只恐又是你的杜撰。”宝玉笑道:“除《四书》外,杜撰的太多,偏只我是杜撰不成?"又问黛玉:“可也有玉没有?"众人不解其语,黛玉便忖度着因他有玉,故问我有也无,因答道:“我没有那个。想来那玉是一件罕物,岂能人人有的。”宝玉听了,登时发作起痴狂病来,摘下那玉,就狠命摔去,骂道:“什么罕物,连人之高低不择,还说`通灵-不`通灵-呢!我也不要这劳什子了!"吓的众人一拥争去拾玉。贾母急的搂了宝玉道:“孽障!你生气,要打骂人容易,何苦摔那命根子!"宝玉满面泪痕泣道:“家里姐姐妹妹都没有,单我有,我说没趣,如今来了这们一个神仙似的妹妹也没有,可知这不是个好东西。”贾母忙哄他道:“你这妹妹原有这个来的,因你姑妈去世时,舍不得你妹妹,无法处,遂将他的玉带了去了:一则全殉葬之礼,尽你妹妹之孝心,二则你姑妈之灵,亦可权作见了女儿之意。因此他只说没有这个,不便自己夸张之意。你如今怎比得他?还不好生慎重带上,仔细你娘知道了。”说着,便向丫鬟手中接来,亲与他带上。宝玉听如此说,想一想大有情理,也就不生别论了。

当下,奶娘来请问黛玉之房舍。贾母说:“今将宝玉挪出来,同我在套间暖阁儿里,把你林姑娘暂安置碧纱橱里。等过了残冬,春天再与他们收拾房屋,另作一番安置罢。”宝玉道:“好祖宗,我就在碧纱橱外的床上很妥当,何必又出来闹的老祖宗不得安静。”贾母想了一想说:“也罢了。”每人一个奶娘并一个丫头照管,余者在外间上夜听唤。一面早有熙凤命人送了一顶藕合色花帐,并几件锦被缎褥之类。

黛玉只带了两个人来:一个是自幼奶娘王嬷嬷,一个是十岁的小丫头,亦是自幼随身的,名唤作雪雁。贾母见雪雁甚小,一团孩气,王嬷嬷又极老,料黛玉皆不遂心省力的,便将自己身边的一个二等丫头,名唤鹦哥者与了黛玉。外亦如迎春等例,每人除自幼侞母外,另有四个教引嬷嬷,除贴身掌管钗钏プ沐两个丫鬟外,另有五六个洒扫房屋来往使役的小丫鬟。当下,王嬷嬷与鹦哥陪侍黛玉在碧纱橱内。宝玉之侞母李嬷嬷,并大丫鬟名唤袭人者,陪侍在外面大床上。

原来这袭人亦是贾母之婢,本名珍珠。贾母因溺爱宝玉,生恐宝玉之婢无竭力尽忠之人,素喜袭人心地纯良,克尽职任,遂与了宝玉。宝玉因知他本姓花,又曾见旧人诗句上有"花气袭人"之句,遂回明贾母,更名袭人。这袭人亦有些痴处:伏侍贾母时,心中眼中只有一个贾母,如今服侍宝玉,心中眼中又只有一个宝玉。只因宝玉性情乖僻,每每规谏宝玉,心中着实忧郁。

是晚,宝玉李嬷嬷已睡了,他见里面黛玉和鹦哥犹未安息,他自卸了妆,悄悄进来,笑问:“姑娘怎么还不安息?"黛玉忙让:“姐姐请坐。”袭人在床沿上坐了。鹦哥笑道:“林姑娘正在这里伤心,自己淌眼抹泪的说:`今儿才来,就惹出你家哥儿的狂病,倘或摔坏了那玉,岂不是因我之过!-因此便伤心,我好容易劝好了"。袭人道:“姑娘快休如此,将来只怕比这个更奇怪的笑话儿还有呢!若为他这种行止,你多心伤感,只怕你伤感不了呢。快别多心!"黛玉道:“姐姐们说的,我记着就是了。究竟那玉不知是怎么个来历?上面还有字迹?"袭人道:“连一家子也不知来历,上头还有现成的眼儿,听得说,落草时是从他口里掏出来的。等我拿来你看便知。”黛玉忙止道:“罢了,此刻夜深,明日再看也不迟。”大家又叙了一回,方才安歇。

次日起来,省过贾母,因往王夫人处来,正值王夫人与熙凤在一处拆金陵来的书信看,又有王夫人之兄嫂处遣了两个媳妇来说话的。黛玉虽不知原委,探春等却都晓得是议论金陵城中所居的薛家姨母之子姨表兄薛蟠,倚财仗势,打死人命,现在应天府案下审理。如今母舅王子腾得了信息,故遣他家内的人来告诉这边,意欲唤取进京之意。

\chapter{薄命女偏逢薄命郎\ttlbreak 葫芦僧乱判葫芦案}

却说黛玉同姊妹们至王夫人处,见王夫人与兄嫂处的来使计议家务,又说姨母家遭人命官司等语。因见王夫人事情冗杂,姊妹们遂出来,至寡嫂李氏房中来了。

原来这李氏即贾珠之妻。珠虽夭亡,幸存一子,取名贾兰,今方五岁,已入学攻书。这李氏亦系金陵名宦之女,父名李守中,曾为国子监祭酒,族中男女无有不诵诗读书者。至李守中继承以来,便说"女子无才便有德",故生了李氏时,便不十分令其读书,只不过将些《女四书》,《列女传》,《贤媛集》等三四种书,使他认得几个字,记得前朝这几个贤女便罢了,却只以纺绩井臼为要,因取名为李纨,字宫裁。因此这李纨虽青春丧偶,居家处膏粱锦绣之中,竟如槁木死灰一般,一概无见无闻,唯知侍亲养子,外则陪侍小姑等针黹诵读而已。今黛玉虽客寄于斯,日有这般姐妹相伴,除老父外,余者也都无庸虑及了。

如今且说雨村,因补授了应天府,一下马就有一件人命官司详至案下,乃是两家争买一婢,各不相让,以至殴伤人命。彼时雨村即传原告之人来审。那原告道:“被殴死者乃小人之主人。因那日买了一个丫头,不想是拐子拐来卖的。这拐子先已得了我家的银子,我家小爷原说第三日方是好日子,再接入门。这拐子便又悄悄的卖与薛家,被我们知道了,去找拿卖主,夺取丫头。无奈薛家原系金陵一霸,倚财仗势,众豪奴将我小主人竟打死了。凶身主仆已皆逃走,无影无踪,只剩了几个局外之人。小人告了一年的状,竟无人作主。望大老爷拘拿凶犯,剪恶除凶,以救孤寡,死者感戴天恩不尽!”

雨村听了大怒道:“岂有这样放屁的事!打死人命就白白的走了,再拿不来的!"因发签差公人立刻将凶犯族中人拿来拷问,令他们实供藏在何处,一面再动海捕文书。正要发签时,只见案边立的一个门子使眼色儿,——不令他发签之意。雨村心下甚为疑怪,只得停了手,即时退堂,至密室,侍从皆退去,只留门子服侍。这门子忙上来请安,笑问:“老爷一向加官进禄,八九年来就忘了我了?"雨村道:“却十分面善得紧,只是一时想不起来。”那门子笑道:“老爷真是贵人多忘事,把出身之地竟忘了,不记当年葫芦庙里之事?"雨村听了,如雷震一惊,方想起往事。原来这门子本是葫芦庙内一个小沙弥,因被火之后,无处安身,欲投别庙去修行,又耐不得清凉景况,因想这件生意倒还轻省热闹,遂趁年纪蓄了发,充了门子。雨村那里料得是他,便忙携手笑道:“原来是故人。”又让坐了好谈。这门子不敢坐。雨村笑道:“贫贱之交不可忘。你我故人也,二则此系私室,既欲长谈,岂有不坐之理?"这门子听说,方告了座,斜签着坐了。

雨村因问方才何故有不令发签之意。这门子道:“老爷既荣任到这一省,难道就没抄一张本省`护官符-来不成?"雨村忙问:“何为`护官符-?我竟不知。”门子道:“这还了得!连这个不知,怎能作得长远!如今凡作地方官者,皆有一个私单,上面写的是本省最有权有势,极富极贵的大乡绅名姓,各省皆然,倘若不知,一时触犯了这样的人家,不但官爵,只怕连性命还保不成呢!所以绰号叫作`护官符-。方才所说的这薛家,老爷如何惹他!他这件官司并无难断之处,皆因都碍着情分面上,所以如此。”一面说,一面从顺袋中取出一张抄写的`护官符-来,递与雨村,看时,上面皆是本地大族名宦之家的谚俗口碑。其口碑排写得明白,下面所注的皆是自始祖官爵并房次。石头亦曾抄写了一张,今据石上所抄云:

贾不假,白玉为堂金作马。(宁国荣国二公之后,共二十房分,宁荣亲派八房在都外,现原籍住者十二房。)

阿房宫,三百里,住不下金陵一个史。(保龄侯尚书令史公之后,房分共十八,都中现住者十房,原籍现居八房。)

东海缺少白玉床,龙王来请金陵王。(都太尉统制县伯王公之后,共十二房,都中二房,余在籍。)

丰年好大雪,珍珠如土金如铁。(紫薇舍人薛公之后,现领内府帑银行商,共八房分。)

雨村犹未看完,忽听传点,人报:“王老爷来拜。”雨村听说,忙具衣冠出去迎接。有顿饭工夫,方回来细问。这门子道:“这四家皆连络有亲,一损皆损,一荣皆荣,扶持遮饰,俱有照应的。今告打死人之薛,就系丰年大雪之`雪-也。也不单靠这三家,他的世交亲友在都在外者,本亦不少。老爷如今拿谁去?"雨村听如此说,便笑问门子道:“如你这样说来,却怎么了结此案?你大约也深知这凶犯躲的方向了?”

门子笑道:“不瞒老爷说,不但这凶犯的方向我知道,一并这拐卖之人我也知道,死鬼买主也深知道。待我细说与老爷听:这个被打之死鬼,乃是本地一个小乡绅之子,名唤冯渊,自幼父母早亡,又无兄弟,只他一个人守着些薄产过日子。长到十八九岁上,酷爱男风,最厌女子。这也是前生冤孽,可巧遇见这拐子卖丫头,他便一眼看上了这丫头,立意买来作妾,立誓再不交结男子,也不再娶第二个了,所以三日后方过门。谁晓这拐子又偷卖与薛家,他意欲卷了两家的银子,再逃往他省。谁知又不曾走脱,两家拿住,打了个臭死,都不肯收银,只要领人。那薛家公子岂是让人的,便喝着手下人一打,将冯公子打了个稀烂,抬回家去三日死了。这薛公子原是早已择定日子上京去的,头起身两日前,就偶然遇见这丫头,意欲买了就进京的,谁知闹出这事来。既打了冯公子,夺了丫头,他便没事人一般,只管带了家眷走他的路。他这里自有兄弟奴仆在此料理,也并非为此些些小事值得他一逃走的。这且别说,老爷你当被卖之丫头是谁?"雨村笑道:“我如何得知。”门子冷笑道:“这人算来还是老爷的大恩人呢!他就是葫芦庙旁住的甄老爷的小姐,名唤英莲的。”雨村罕然道:“原来就是他!闻得养至五岁被人拐去,却如今才来卖呢?”

门子道:“这一种拐子单管偷拐五六岁的儿女,养在一个僻静之处,到十一二岁,度其容貌,带至他乡转卖。当日这英莲,我们天天哄他顽耍,虽隔了七八年,如今十二三岁的光景,其模样虽然出脱得齐整好些,然大概相貌,自是不改,熟人易认。况且他眉心中原有米粒大小的一点胭脂т,从胎里带来的,所以我却认得。偏生这拐子又租了我的房舍居住,那日拐子不在家,我也曾问他。他是被拐子打怕了的,万不敢说,只说拐子系他亲爹,因无钱偿债,故卖他。我又哄之再四,他又哭了,只说`我不记得小时之事!-这可无疑了。那日冯公子相看了,兑了银子,拐子醉了,他自叹道:`我今日罪孽可满了!-后又听见冯公子令三日之后过门,他又转有忧愁之态。我又不忍其形景,等拐子出去,又命内人去解释他:`这冯公子必待好日期来接,可知必不以丫鬟相看。况他是个绝风流人品,家里颇过得,素习又最厌恶堂客,今竟破价买你,后事不言可知。只耐得三两日,何必忧闷!-他听如此说,方才略解忧闷,自为从此得所。谁料天下竟有这等不如意事,第二日,他偏又卖与薛家。若卖与第二个人还好,这薛公子的混名人称`呆霸王-,最是天下第一个弄性尚气的人,而且使钱如土,遂打了个落花流水,生拖死拽,把个英莲拖去,如今也不知死活。这冯公子空喜一场,一念未遂,反花了钱,送了命,岂不可叹!”

雨村听了,亦叹道:“这也是他们的孽障遭遇,亦非偶然。不然这冯渊如何偏只看准了这英莲?这英莲受了拐子这几年折磨,才得了个头路,且又是个多情的,若能聚合了,倒是件美事,偏又生出这段事来。这薛家纵比冯家富贵,想其为人,自然姬妾众多,滢佚无度,未必及冯渊定情于一人者。这正是梦幻情缘,恰遇一对薄命儿女。且不要议论他,只目今这官司,如何剖断才好?"门子笑道:“老爷当年何其明决,今日何反成了个没主意的人了!小的闻得老爷补升此任,亦系贾府王府之力,此薛蟠即贾府之亲,老爷何不顺水行舟,作个整人情,将此案了结,日后也好去见贾府王府。”雨村道:“你说的何尝不是。但事关人命,蒙皇上隆恩,起复委用,实是重生再造,正当殚心竭力图报之时,岂可因私而废法?是我实不能忍为者。”门子听了,冷笑道:“老爷说的何尝不是大道理,但只是如今世上是行不去的。岂不闻古人有云:`大丈夫相时而动-,又曰`趋吉避凶者为君子-。依老爷这一说,不但不能报效朝廷,亦且自身不保,还要三思为妥。”

雨村低了半日头,方说道:“依你怎么样?"门子道:“小人已想了一个极好的主意在此:老爷明日坐堂,只管虚张声势,动文书发签拿人。原凶自然是拿不来的,原告固是定要将薛家族中及奴仆人等拿几个来拷问。小的在暗中调停,令他们报个暴病身亡,令族中及地方上共递一张保呈,老爷只说善能扶鸾请仙,堂上设下乩坛,令军民人等只管来看。老爷就说:`乩仙批了,死者冯渊与薛蟠原因夙孽相逢,今狭路既遇,原应了结。薛蟠今已得了无名之病,被冯魂追索已死。其祸皆因拐子某人而起,拐之人原系某乡某姓人氏,按法处治,余不略及-等语。小人暗中嘱托拐子,令其实招。众人见乩仙批语与拐子相符,余者自然也都不虚了。薛家有的是钱,老爷断一千也可,五百也可,与冯家作烧埋之费。那冯家也无甚要紧的人,不过为的是钱,见有了这个银子,想来也就无话了。老爷细想此计如何?"雨村笑道:“不妥,不妥。等我再斟酌斟酌,或可压服口声。”二人计议,天色已晚,别无话说。

至次日坐堂,勾取一应有名人犯,雨村详加审问,果见冯家人口稀疏,不过赖此欲多得些烧埋之费,薛家仗势倚情,偏不相让,故致颠倒未决。雨村便徇情枉法,胡乱判断了此案。冯家得了许多烧埋银子,也就无甚话说了。雨村断了此案,急忙作书信二封,与贾政并京营节度使王子腾,不过说"令甥之事已完,不必过虑"等语。此事皆由葫芦庙内之沙弥新门子所出,雨村又恐他对人说出当日贫贱时的事来,因此心中大不乐业,后来到底寻了个不是,远远的充发了他才罢。

当下言不着雨村。且说那买了英莲打死冯渊的薛公子,亦系金陵人氏,本是书香继世之家。只是如今这薛公子幼年丧父,寡母又怜他是个独根孤种,未免溺爱纵容,遂至老大无成,且家中有百万之富,现领着内帑钱粮,采办杂料。这薛公子学名薛蟠,表字文起,五岁上就性情奢侈,言语傲慢。虽也上过学,不过略识几字,终日惟有斗鸡走马,游山玩水而已。虽是皇商,一应经济世事,全然不知,不过赖祖父之旧情分,户部挂虚名,支领钱粮,其余事体,自有伙计老家人等措办。寡母王氏乃现任京营节度使王子腾之妹,与荣国府贾政的夫人王氏,是一母所生的姊妹,今年方四十上下年纪,只有薛蟠一子。还有一女,比薛蟠小两岁,侞名宝钗,生得肌骨莹润,举止娴雅。当日有他父亲在日,酷爱此女,令其读书识字,较之乃兄竟高过十倍。自父亲死后,见哥哥不能依贴母怀,他便不以书字为事,只留心针黹家计等事,好为母亲分忧解劳。近因今上崇诗尚礼,征采才能,降不世出之隆恩,除聘选妃嫔外,凡仕宦名家之女,皆亲名达部,以备选为公主郡主入学陪侍,充为才人赞善之职。二则自薛蟠父亲死后,各省中所有的买卖承局,总管,伙计人等,见薛蟠年轻不谙世事,便趁时拐骗起来,京都中几处生意,渐亦消耗。薛蟠素闻得都中乃第一繁华之地,正思一游,便趁此机会,一为送妹待选,二为望亲,三因亲自入部销算旧帐,再计新支,-其实则为游览上国风光之意。因此早已打点下行装细软,以及馈送亲友各色土物人情等类,正择日一定起身,不想偏遇见了拐子重卖英莲。薛蟠见英莲生得不俗,立意买他,又遇冯家来夺人,因恃强喝令手下豪奴将冯渊打死。他便将家中事务一一的嘱托了族中人并几个老家人,他便带了母妹竟自起身长行去了。人命官司一事,他竟视为儿戏,自为花上几个臭钱,没有不了的。

在路不记其日。那日已将入都时,却又闻得母舅王子腾升了九省统制,奉旨出都查边。薛蟠心中暗喜道:“我正愁进京去有个嫡亲的母舅管辖着,不能任意挥霍挥霍,偏如今又升出去了,可知天从人愿。”因和母亲商议道:“咱们京中虽有几处房舍,只是这十来年没人进京居住,那看守的人未免偷着租赁与人,须得先着几个人去打扫收拾才好。”他母亲道:“何必如此招摇!咱们这一进京,原该先拜望亲友,或是在你舅舅家,或是你姨爹家。他两家的房舍极是便宜的,咱们先能着住下,再慢慢的着人去收拾,岂不消停些。”薛蟠道:“如今舅舅正升了外省去,家里自然忙乱起身,咱们这工夫一窝一拖的奔了去,岂不没眼色。”他母亲道:“你舅舅家虽升了去,还有你姨爹家。况这几年来,你舅舅姨娘两处,每每带信捎书,接咱们来。如今既来了,你舅舅虽忙着起身,你贾家姨娘未必不苦留我们。咱们且忙忙收拾房屋,岂不使人见怪?你的意思我却知道,守着舅舅姨爹住着,未免拘紧了你,不如你各自住着,好任意施为。你既如此,你自去挑所宅子去住,我和你姨娘,姊妹们别了这几年,却要厮守几日,我带了你妹子投你姨娘家去,你道好不好?"薛蟠见母亲如此说,情知扭不过的,只得吩咐人夫一路奔荣国府来。

那时王夫人已知薛蟠官司一事,亏贾雨村维持了结,才放了心。又见哥哥升了边缺,正愁又少了娘家的亲戚来往,略加寂寞。过了几日,忽家人传报:“姨太太带了哥儿姐儿,合家进京,正在门外下车。”喜的王夫人忙带了女媳人等,接出大厅,将薛姨妈等接了进去。姊妹们暮年相会,自不必说悲喜交集,泣笑叙阔一番。忙又引了拜见贾母,将人情土物各种酬献了。合家俱厮见过,忙又治席接风。

薛蟠已拜见过贾政,贾琏又引着拜见了贾赦,贾珍等。贾政便使人上来对王夫人说:“姨太太已有了春秋,外甥年轻不知世路,在外住着恐有人生事。咱们东北角上梨香院一所十来间房,白空闲着,打扫了,请姨太太和姐儿哥儿住了甚好。”王夫人未及留,贾母也就遣人来说:“请姨太太就在这里住下,大家亲密些"等语。薛姨妈正要同居一处,方可拘紧些儿子,若另住在外,又恐他纵性惹祸,遂忙道谢应允。又私与王夫人说明:“一应日费供给一概免却,方是处常之法。”王夫人知他家不难于此,遂亦从其愿。从此后薛家母子就在梨香院住了。

原来这梨香院即当日荣公暮年养静之所,小小巧巧,约有十余间房屋,前厅后舍俱全。另有一门通街,薛蟠家人就走此门出入。西南有一角门,通一夹道,出夹道便是王夫人正房的东边了。每日或饭后,或晚间,薛姨妈便过来,或与贾母闲谈,或与王夫人相叙。宝钗日与黛玉迎春姊妹等一处,或看书下棋,或作针黹,倒也十分乐业。只是薛蟠起初之心,原不欲在贾宅居住者,但恐姨父管约拘禁,料必不自在的,无奈母亲执意在此,且宅中又十分殷勤苦留,只得暂且住下,一面使人打扫出自己的房屋,再移居过去的。谁知自从在此住了不上一月的光景,贾宅族中凡有的子侄,俱已认熟了一半,凡是那些纨э气习者,莫不喜与他来往,今日会酒,明日观花,甚至聚赌嫖娼,渐渐无所不至,引诱的薛蟠比当日更坏了十倍。虽然贾政训子有方,治家有法,一则族大人多,照管不到这些,二则现任族长乃是贾珍,彼乃宁府长孙,又现袭职,凡族中事,自有他掌管,三则公私冗杂,且素性潇洒,不以俗务为要,每公暇之时,不过看书着棋而已,余事多不介意。况且这梨香院相隔两层房舍,又有街门另开,任意可以出入,所以这些子弟们竟可以放意畅怀的,因此遂将移居之念渐渐打灭了。

\chapter{贾宝玉神游太虚境\ttlbreak 警幻仙曲演红楼梦}

第四回中既将薛家母子在荣府内寄居等事略已表明,此回则暂不能写矣。

如今且说林黛玉自在荣府以来,贾母万般怜爱,寝食起居,一如宝玉,迎春,探春,惜春三个亲孙女倒且靠后,便是宝玉和黛玉二人之亲密友爱处,亦自较别个不同,日则同行同坐,夜则同息同止,真是言和意顺,略无参商。不想如今忽然来了一个薛宝钗,年岁虽大不多,然品格端方,容貌丰美,人多谓黛玉所不及。而且宝钗行为豁达,随分从时,不比黛玉孤高自许,目无下尘,故比黛玉大得下人之心。便是那些小丫头子们,亦多喜与宝钗去顽。因此黛玉心中便有些悒郁不忿之意,宝钗却浑然不觉。那宝玉亦在孩提之间,况自天性所禀来的一片愚拙偏僻,视姊妹弟兄皆出一意,并无亲疏远近之别。其中因与黛玉同随贾母一处坐卧,故略比别个姊妹熟惯些。既熟惯,则更觉亲密,既亲密,则不免一时有求全之毁,不虞之隙。这日不知为何,他二人言语有些不合起来,黛玉又气的独在房中垂泪,宝玉又自悔言语冒撞,前去俯就,那黛玉方渐渐的回转来。因东边宁府中花园内梅花盛开,贾珍之妻尤氏乃治酒,请贾母,邢夫人,王夫人等赏花。是日先携了贾蓉之妻,二人来面请。贾母等于早饭后过来,就在会芳园游顽,先茶后酒,不过皆是宁荣二府女眷家宴小集,并无别样新文趣事可记。

一时宝玉倦怠,欲睡中觉,贾母命人好生哄着,歇一回再来。贾蓉之妻秦氏便忙笑回道:“我们这里有给宝叔收拾下的屋子,老祖宗放心,只管交与我就是了。”又向宝玉的奶娘丫鬟等道:“嬷嬷,姐姐们,请宝叔随我这里来。”贾母素知秦氏是个极妥当的人,生的袅娜纤巧,行事又温柔和平,乃重孙媳中第一个得意之人,见他去安置宝玉,自是安稳的。

当下秦氏引了一簇人来至上房内间。宝玉抬头看见一幅画贴在上面,画的人物固好,其故事乃是《燃藜图》,也不看系何人所画,心中便有些不快。又有一幅对联,写的是:

世事洞明皆学问,人情练达即文章。及看了这两句,纵然室宇精美,铺陈华丽,亦断断不肯在这里了,忙说:“快出去!快出去!"秦氏听了笑道:“这里还不好,可往那里去呢?不然往我屋里去吧。”宝玉点头微笑。有一个嬷嬷说道:“那里有个叔叔往侄儿房里睡觉的理?"秦氏笑道:“嗳哟哟,不怕他恼。他能多大呢,就忌讳这些个!上月你没看见我那个兄弟来了,虽然与宝叔同年,两个人若站在一处,只怕那个还高些呢。”宝玉道:“我怎么没见过?你带他来我瞧瞧。”众人笑道:“隔着二三十里,往那里带去,见的日子有呢。”说着大家来至秦氏房中。刚至房门,便有一股细细的甜香袭人而来。宝玉觉得眼饧骨软,连说"好香!"入房向壁上看时,有唐伯虎画的《海棠春睡图》,两边有宋学士秦太虚写的一副对联,其联云:

嫩寒锁梦因春冷,芳气笼人是酒香。案上设着武则天当日镜室中设的宝镜,一边摆着飞燕立着舞过的金盘,盘内盛着安禄山掷过伤了太真侞的木瓜。上面设着寿昌公主于含章殿下卧的榻,悬的是同昌公主制的联珠帐。宝玉含笑连说:“这里好!"秦氏笑道:“我这屋子大约神仙也可以住得了。”说着亲自展开了西子浣过的纱衾,移了红娘抱过的鸳枕。于是众奶母伏侍宝玉卧好,款款散了,只留袭人,媚人,晴雯,麝月四个丫鬟为伴。秦氏便分咐小丫鬟们,好生在廊檐下看着猫儿狗儿打架。

那宝玉刚合上眼,便惚惚的睡去,犹似秦氏在前,遂悠悠荡荡,随了秦氏,至一所在。但见朱栏白石,绿树清溪,真是人迹希逢,飞尘不到。宝玉在梦中欢喜,想道:“这个去处有趣,我就在这里过一生,纵然失了家也愿意,强如天天被父母师傅打呢。”正胡思之间,忽听山后有人作歌曰:

春梦随云散,飞花逐水流,

寄言众儿女,何必觅闲愁。宝玉听了是女子的声音。歌声未息,早见那边走出一个人来,蹁跹袅娜,端的与人不同。有赋为证:

方离柳坞,乍出花房。但行处,鸟惊庭树,将到时,

影度回廊。仙袂乍飘兮,闻麝兰之馥郁,荷衣欲动兮,

听环佩之铿锵。靥笑春桃兮,云堆翠髻,唇绽樱颗兮,榴

齿含香。纤腰之楚楚兮,回风舞雪,珠翠之辉辉兮,满

额鹅黄。出没花间兮,宜嗔宜喜,徘徊池上兮,若飞若扬。

蛾眉颦笑兮,将言而未语,莲步乍移兮,待止而欲行。羡彼

之良质兮,冰清玉润,羡彼之华服兮,闪灼文章。爱彼之貌

容兮,香培玉琢,美彼之态度兮,凤翥龙翔。其素若何,

春梅绽雪。其洁若何,秋菊被霜。其静若何,松生空谷。

其艳若何,霞映澄塘。其文若何,龙游曲沼。其神若何,月

射寒江。应惭西子,实愧王嫱。奇矣哉,生于孰地,来自

何方,信矣乎,瑶池不二,紫府无双。果何人哉?如斯之

美也!

宝玉见是一个仙姑,喜的忙来作揖问道:“神仙姐姐不知从那里来,如今要往那里去?也不知这是何处,望乞携带携带。”那仙姑笑道:“吾居离恨天之上,灌愁海之中,乃放春山遣香洞太虚幻境警幻仙姑是也:司人间之风情月债,掌尘世之女怨男痴。因近来风流冤孽,缠绵于此处,是以前来访察机会,布散相思。今忽与尔相逢,亦非偶然。此离吾境不远,别无他物,仅有自采仙茗一盏,亲酿美酒一瓮,素练魔舞歌姬数人,新填《红楼梦》仙曲十二支,试随吾一游否?"宝玉听说,便忘了秦氏在何处,竟随了仙姑,至一所在,有石牌横建,上书"太虚幻境"四个大字,两边一副对联,乃是:

假作真时真亦假,无为有处有还无。转过牌坊,便是一座宫门,上面横书四个大字,道是:“孽海情天"。又有一副对联,大书云:

厚地高天,堪叹古今情不尽,

痴男怨女,可怜风月债难偿。

宝玉看了,心下自思道:“原来如此。但不知何为`古今之情-,何为`风月之债-?从今倒要领略领略。”宝玉只顾如此一想,不料早把些邪魔招入膏肓了。当下随了仙姑进入二层门内,至两边配殿,皆有匾额对联,一时看不尽许多,惟见有几处写的是:“痴情司","结怨司","朝啼司","夜怨司","春感司","秋悲司"。看了,因向仙姑道:“敢烦仙姑引我到那各司中游玩游玩,不知可使得?"仙姑道:“此各司中皆贮的是普天之下所有的女子过去未来的簿册,尔凡眼尘躯,未便先知的。”宝玉听了,那里肯依,复央之再四。仙姑无奈,说:“也罢,就在此司内略随喜随喜罢了。”宝玉喜不自胜,抬头看这司的匾上,乃是"薄命司"三字,两边对联写的是:

春恨秋悲皆自惹,花容月貌为谁妍。

宝玉看了,便知感叹。进入门来,只见有十数个大厨,皆用封条封着。看那封条上,皆是各省的地名。宝玉一心只拣自己的家乡封条看,遂无心看别省的了。只见那边厨上封条上大书七字云:“金陵十二钗正册"。宝玉问道:“何为`金陵十二钗正册-?"警幻道:“即贵省中十二冠首女子之册,故为`正册。”宝玉道:“常听人说,金陵极大,怎么只十二个女子?如今单我家里,上上下下,就有几百女孩子呢。”警幻冷笑道:“贵省女子固多,不过择其紧要者录之。下边二厨则又次之。余者庸常之辈,则无册可录矣。”宝玉听说,再看下首二厨上,果然写着"金陵十二钗副册",又一个写着"金陵十二钗又副册"。宝玉便伸手先将"又副册"厨开了,拿出一本册来,揭开一看,只见这首页上画着一幅画,又非人物,也无山水,不过是水墨ч染的满纸乌云浊雾而已。后有几行字迹,写的是:

霁月难逢,彩云易散。心比天高,身为下贱。风流灵巧

招人怨。寿夭多因毁谤生,多情公子空牵念。

宝玉看了,又见后面画着一簇鲜花,一床破席,也有几句言词,写道是:

枉自温柔和顺,空云似桂如兰,

堪羡优伶有福,谁知公子无缘。宝玉看了不解。遂掷下这个,又去开了副册厨门,拿起一本册来,揭开看时,只见画着一株桂花,下面有一池沼,其中水涸泥干,莲枯藕败,后面书云:

根并荷花一茎香,平生遭际实堪伤。

自从两地生孤木,致使香魂返故乡。宝玉看了仍不解。便又掷了,再去取"正册"看,只见头一页上便画着两株枯木,木上悬着一围玉带,又有一堆雪,雪下一股金簪。也有四句言词,道是:

可叹停机德,堪怜咏絮才。

玉带林中挂,金簪雪里埋。宝玉看了仍不解。待要问时,情知他必不肯泄漏,待要丢下,又不舍。遂又往后看时,只见画着一张弓,弓上挂着香橼。也有一首歌词云:

二十年来辨是非,榴花开处照宫闱。

三春争及初春景,虎兕相逢大梦归。后面又画着两人放风筝,一片大海,一只大船,船中有一女子掩面泣涕之状。也有四句写云:

才自精明志自高,生于末世运偏消。

清明涕送江边望,千里东风一梦遥。后面又画几缕飞云,一湾逝水。其词曰:

富贵又何为,襁褓之间父母违。

展眼吊斜晖,湘江水逝楚云飞。后面又画着一块美玉,落在泥垢之中。其断语云:

欲洁何曾洁,云空未必空。

可怜金玉质,终陷淖泥中。后面忽见画着个恶狼,追扑一美女,欲啖之意。其书云:

子系中山狼,得志便猖狂。

金闺花柳质,一载赴黄粱。后面便是一所古庙,里面有一美人在内看经独坐。其判云:

勘破三春景不长,缁衣顿改昔年妆。

可怜绣户侯门女,独卧青灯古佛旁。后面便是一片冰山,上面有一只雌凤。其判曰:

凡鸟偏从末世来,都知爱慕此生才。

一从二令三人木,哭向金陵事更哀。后面又是一座荒村野店,有一美人在那里纺绩。其判云:

势败休云贵,家亡莫论亲。

偶因济刘氏,巧得遇恩人。后面又画着一盆茂兰,旁有一位凤冠霞帔的美人。也有判云:

桃李春风结子完,到头谁似一盆兰。

如冰水好空相妒,枉与他人作笑谈。后面又画着高楼大厦,有一美人悬梁自缢。其判云:

情天情海幻情身,情既相逢必主滢。

漫言不肖皆荣出,造衅开端实在宁。

宝玉还欲看时,那仙姑知他天分高明,性情颖慧,恐把仙机泄漏,遂掩了卷册,笑向宝玉道:“且随我去游玩奇景,何必在此打这闷葫芦!”

宝玉恍恍惚惚,不觉弃了卷册,又随了警幻来至后面。但见珠帘绣幕,画栋雕檐,说不尽那光摇朱户金铺地,雪照琼窗玉作宫。更见仙花馥郁,异草芬芳,真好个所在。又听警幻笑道:“你们快出来迎接贵客!"一语未了,只见房中又走出几个仙子来,皆是荷袂蹁跹,羽衣飘舞,姣若春花,媚如秋月。一见了宝玉,都怨谤警幻道:“我们不知系何`贵客-,忙的接了出来!姐姐曾说今日今时必有绛珠妹子的生魂前来游玩,故我等久待。何故反引这浊物来污染这清净女儿之境?”

宝玉听如此说,便吓得欲退不能退,果觉自形污秽不堪。警幻忙携住宝玉的手,向众姊妹道:“你等不知原委:今日原欲往荣府去接绛珠,适从宁府所过,偶遇宁荣二公之灵,嘱吾云:`吾家自国朝定鼎以来,功名奕世,富贵传流,虽历百年,奈运终数尽,不可挽回者。故遗之子孙虽多,竟无可以继业。其中惟嫡孙宝玉一人,禀性乖张,生性怪谲,虽聪明灵慧,略可望成,无奈吾家运数合终,恐无人规引入正。幸仙姑偶来,万望先以情欲声色z等事警其痴顽,或能使彼跳出迷人圈子,然后入于正路,亦吾兄弟之幸矣。-如此嘱吾,故发慈心,引彼至此。先以彼家上中下三等女子之终身册籍,令彼熟玩,尚未觉悟,故引彼再至此处,令其再历饮馔声色之幻,或冀将来一悟,亦未可知也。”

说毕,携了宝玉入室。但闻一缕幽香,竟不知其所焚何物。宝玉遂不禁相问。警幻冷笑道:“此香尘世中既无,尔何能知!此香乃系诸名山胜境内初生异卉之精,合各种宝林珠树之油所制,名`群芳髓。”宝玉听了,自是羡慕而已。大家入座,小丫鬟捧上茶来。宝玉自觉清香异味,纯美非常,因又问何名。警幻道:“此茶出在放春山遣香洞,又以仙花灵叶上所带之宿露而烹,此茶名曰`千红一窟。”宝玉听了,点头称赏。因看房内,瑶琴,宝鼎,古画,新诗,无所不有,更喜窗下亦有唾绒,奁间时渍粉污。壁上也见悬着一副对联,书云:

幽微灵秀地,无可奈何天。宝玉看毕,无不羡慕。因又请问众仙姑姓名:一名痴梦仙姑,一名钟情大士,一名引愁金女,一名度恨菩提,各各道号不一。少刻,有小丫鬟来调桌安椅,设摆酒馔。真是:琼浆满泛玻璃盏,玉液浓斟琥珀杯。更不用再说那肴馔之盛。宝玉因闻得此酒清香甘冽,异乎寻常,又不禁相问。警幻道:“此酒乃以百花之蕊,万木之汁,加以麟髓之醅,凤侞之ш酿成,因名为`万艳同杯。”宝玉称赏不迭。

饮酒间,又有十二个舞女上来,请问演何词曲。警幻道:“就将新制《红楼梦》十二支演上来。”舞女们答应了,便轻敲檀板,款按银筝,听他歌道是:

开辟鸿蒙……方歌了一句,警幻便说道:“此曲不比尘世中所填传奇之曲,必有生旦净末之则,又有南北九宫之限。此或咏叹一人,或感怀一事,偶成一曲,即可谱入管弦。若非个中人,不知其中之妙。料尔亦未必深明此调。若不先阅其稿,后听其歌,翻成嚼蜡矣。”说毕,回头命小丫鬟取了《红楼梦》原稿来,递与宝玉。宝玉接来,一面目视其文,一面耳聆其歌曰:

《红楼梦引子》开辟鸿蒙,谁为情种?都只为风月情浓。趁着这奈何天,伤怀日,寂寥时,试遣愚衷。因此上,

演出这怀金悼玉的《红楼梦》。

[终身误]都道是金玉良姻,俺只念木石前盟。空对着,山中高士晶莹雪,终不忘,世外仙姝寂寞林。叹人间,美

中不足今方信。纵然是齐眉举案,到底意难平。

[枉凝眉]一个是阆苑仙葩,一个是美玉无瑕。若说

没奇缘,今生偏又遇着他,若说有奇缘,如何心事终虚化?一个枉自嗟呀,一个空劳牵挂。一个是水中月,一个是镜中

花。想眼中能有多少泪珠儿,怎经得秋流到冬尽,春流到

夏!

宝玉听了此曲,散漫无稽,不见得好处,但其声韵凄惋,竟能销魂醉魄。因此也不察其原委,问其来历,就暂以此释闷而已。因又看下道:

[恨无常]喜荣华正好,恨无常又到。眼睁睁,把万事

全抛。荡悠悠,把芳魂消耗。望家乡,路远山高。故向爹娘

梦里相寻告:儿命已入黄泉,天轮呵,须要退步怞身早!

[分骨肉]一帆风雨路三千,把骨肉家园齐来抛闪。

恐哭损残年,告爹娘,休把儿悬念。自古穷通皆有定,

离合岂无缘?从今分两地,各自保平安。奴去也,莫牵

连。

[乐中悲]襁褓中,父母叹双亡。纵居那绮罗丛,谁知娇

养?幸生来,英豪阔大宽宏量,从未将儿女私情略萦心上。

好一似,霁月光风耀玉堂。厮配得才貌仙郎,博得个地久天

长,准折得幼年时坎坷形状。终久是云散高唐,水涸湘江。

这是尘寰中消长数应当,何必枉悲伤!

[世难容]气质美如兰,才华阜比仙。天生成孤癖人皆

罕。你道是啖肉食腥膻,视绮罗俗厌,却不知太高人愈妒,过洁世同嫌。可叹这,青灯古殿人将老,辜负了,红粉朱楼

春色阑。到头来,依旧是风尘肮脏违心愿。好一似,无瑕白

玉遭泥陷,又何须,王孙公子叹无缘。

[喜冤家]中山狼,无情兽,全不念当日根由。一味的

骄奢滢荡贪还构。觑着那,侯门艳质同蒲柳,作践的,公府

千金似下流。叹芳魂艳魄,一载荡悠悠。

[虚花悟]将那三春看破,桃红柳绿待如何?把这韶

华打灭,觅那清淡天和。说什么,天上夭桃盛,云中杏蕊多。

到头来,谁把秋捱过?则看那,白杨村里人呜咽,青枫林下

鬼吟哦。更兼着,连天衰草遮坟墓。这的是,昨贫今富人劳

碌,春荣秋谢花折磨。似这般,生关死劫谁能躲?闻说道,

西方宝树唤婆娑,上结着长生果。

[聪明累]机关算尽太聪明,反算了卿卿性命。生前心已碎,死后性空灵。家富人宁,终有个家亡人散各奔腾。枉费

了,意悬悬半世心,好一似,荡悠悠三更梦。忽喇喇似大厦倾,

昏惨惨似灯将尽。呀!一场欢喜忽悲辛。叹人世,终难定!

[留余庆]留余庆,留余庆,忽遇恩人,幸娘亲,幸娘

亲,积得陰功。劝人生,济困扶穷,休似俺那爱银钱忘骨肉的狠舅奸兄!正是乘除加减,上有苍穹。

[晚韶华]镜里恩情,更那堪梦里功名!那美韶华去之何迅!再休提锈帐鸳衾。只这带珠冠,披凤袄,也抵不了

无常性命。虽说是,人生莫受老来贫,也须要陰骘积儿孙。

气昂昂头戴簪缨,气昂昂头戴簪缨,光灿灿胸悬金印,威赫

赫爵禄高登,威赫赫爵禄高登,昏惨惨黄泉路近。问古来将

相可还存?也只是虚名儿与后人钦敬。

[好事终]画梁春尽落香尘。擅风情,秉月貌,便是败

家的根本。箕裘颓堕皆从敬,家事消亡首罪宁。宿孽总因

情。

[收尾。飞鸟各投林]为官的,家业凋零,富贵的,金

银散尽,有恩的,死里逃生,无情的,分明报应。欠命的,命已还,欠泪的,泪已尽。冤冤相报实非轻,分离聚合皆前定。

欲知命短问前生,老来富贵也真侥幸。看破的,遁入空门,痴

迷的,枉送了性命。好一似食尽鸟投林,落了片白茫茫大地真干净!

歌毕,还要歌副曲。警幻见宝玉甚无趣味,因叹:“痴儿竟尚未悟!"那宝玉忙止歌姬不必再唱,自觉朦胧恍惚,告醉求卧。警幻便命撤去残席,送宝玉至一香闺绣阁之中,其间铺陈之盛,乃素所未见之物。更可骇者,早有一位女子在内,其鲜艳妩媚,有似乎宝钗,风流袅娜,则又如黛玉。正不知何意,忽警幻道:“尘世中多少富贵之家,那些绿窗风月,绣阁烟霞,皆被滢污纨э与那些流荡女子悉皆玷辱。更可恨者,自古来多少轻薄浪子,皆以`好色不滢-为饰,又以`情而不滢-作案,此皆饰非掩丑之语也。好色即滢,知情更滢。是以巫山之会,云雨之欢,皆由既悦其色,复恋其情所致也。吾所爱汝者,乃天下古今第一滢人也”

宝玉听了,唬的忙答道:“仙姑差了。我因懒于读书,家父母尚每垂训饬,岂敢再冒`滢-字。况且年纪尚小,不知`滢-字为何物。”警幻道:“非也。滢虽一理,意则有别。如世之好滢者,不过悦容貌,喜歌舞,调笑无厌,云雨无时,恨不能尽天下之美女供我片时之趣兴,此皆皮肤滢滥之蠢物耳。如尔则天分中生成一段痴情,吾辈推之为`意滢-。`意滢-二字,惟心会而不可口传,可神通而不可语达。汝今独得此二字,在闺阁中,固可为良友,然于世道中未免迂阔怪诡,百口嘲谤,万目睚眦。今既遇令祖宁荣二公剖腹深嘱,吾不忍君独为我闺阁增光,见弃于世道,是以特引前来,醉以灵酒,沁以仙茗,警以妙曲,再将吾妹一人,侞名兼美字可卿者,许配于汝。今夕良时,即可成姻。不过令汝领略此仙闺幻境之风光尚如此,何况尘境之情景哉?而今后万万解释,改悟前情,留意于孔孟之间,委身于经济之道。”说毕便秘授以云雨之事,推宝玉入房,将门掩上自去。

那宝玉恍恍惚惚,依警幻所嘱之言,未免有儿女之事,难以尽述。至次日,便柔情缱绻,软语温存,与可卿难解难分。因二人携手出去游顽之时,忽至一个所在,但见荆榛遍地,狼虎同群,迎面一道黑溪阻路,并无桥梁可通。正在犹豫之间,忽见警幻后面追来,告道:“快休前进,作速回头要紧!"宝玉忙止步问道:“此系何处?"警幻道:“此即迷津也。深有万丈,遥亘千里,中无舟楫可通,只有一个木筏,乃木居士掌舵,灰侍者撑篙,不受金银之谢,但遇有缘者渡之。尔今偶游至此,设如堕落其中,则深负我从前谆谆警戒之语矣。”话犹未了,只听迷津内水响如雷,竟有许多夜叉海鬼将宝玉拖将下去。吓得宝玉汗下如雨,一面失声喊叫:“可卿救我!"吓得袭人辈众丫鬟忙上来搂住,叫:“宝玉别怕,我们在这里!”

却说秦氏正在房外嘱咐小丫头们好生看着猫儿狗儿打架,忽听宝玉在梦中唤他的小名,因纳闷道:“我的小名这里从没人知道的,他如何知道,在梦里叫出来?"正是:

一场幽梦同谁近,千古情人独我痴。

\chapter{贾宝玉初试云雨情\ttlbreak 刘姥姥一进荣国府}

却说秦氏因听见宝玉从梦中唤他的侞名,心中自是纳闷,又不好细问。彼时宝玉迷迷惑惑,若有所失。众人忙端上桂圆汤来,呷了两口,遂起身整衣。袭人伸手与他系裤带时,不觉伸手至大腿处,只觉冰凉一片沾湿,唬的忙退出手来,问是怎么了。宝玉红涨了脸,把他的手一捻。袭人本是个聪明女子,年纪本又比宝玉大两岁,近来也渐通人事,今见宝玉如此光景,心中便觉察一半了,不觉也羞的红涨了脸面,不敢再问。仍旧理好衣裳,遂至贾母处来,胡乱吃毕了晚饭,过这边来。

袭人忙趁众奶娘丫鬟不在旁时,另取出一件中衣来与宝玉换上。宝玉含羞央告道:“好姐姐,千万别告诉人。”袭人亦含羞笑问道:“你梦见什么故事了?是那里流出来的那些脏东西?"宝玉道:“一言难尽。”说着便把梦中之事细说与袭人听了。然后说至警幻所授云雨之情,羞的袭人掩面伏身而笑。宝玉亦素喜袭人柔媚娇俏,遂强袭人同领警幻所训云雨之事。袭人素知贾母已将自己与了宝玉的,今便如此,亦不为越礼,遂和宝玉偷试一番,幸得无人撞见。自此宝玉视袭人更比别个不同,袭人待宝玉更为尽心。暂且别无话说。

按荣府中一宅人合算起来,人口虽不多,从上至下也有三四百丁,虽事不多,一天也有一二十件,竟如乱麻一般,并无个头绪可作纲领。正寻思从那一件事自那一个人写起方妙,恰好忽从千里之外,芥щ之微,小小一个人家,因与荣府略有些瓜葛,这日正往荣府中来,因此便就此一家说来,倒还是头绪。你道这一家姓甚名谁,又与荣府有甚瓜葛?且听细讲。方才所说的这小小之家,乃本地人氏,姓王,祖上曾作过小小的一个京官,昔年与凤姐之祖王夫人之父认识。因贪王家的势利,便连了宗认作侄儿。那时只有王夫人之大兄凤姐之父与王夫人随在京中的,知有此一门连宗之族,余者皆不认识。目今其祖已故,只有一个儿子,名唤王成,因家业萧条,仍搬出城外原乡中住去了。王成新近亦因病故,只有其子,小名狗儿。狗儿亦生一子,小名板儿,嫡妻刘氏,又生一女,名唤青儿。一家四口,仍以务农为业。因狗儿白日间又作些生计,刘氏又躁井臼等事,青板姊妹两个无人看管,狗儿遂将岳母刘姥姥接来一处过活。这刘姥姥乃是个积年的老寡妇,膝下又无儿女,只靠两亩薄田度日。今者女婿接来养活,岂不愿意,遂一心一计,帮趁着女儿女婿过活起来。因这年秋尽冬初,天气冷将上来,家中冬事未办,狗儿未免心中烦虑,吃了几杯闷酒,在家闲寻气恼,刘氏也不敢顶撞。因此刘姥姥看不过,乃劝道:“姑爷,你别嗔着我多嘴。咱们村庄人,那一个不是老老诚诚的,守多大碗儿吃多大的饭。你皆因年小的时候,托着你那老家之福,吃喝惯了,如今所以把持不住。有了钱就顾头不顾尾,没了钱就瞎生气,成个什么男子汉大丈夫呢!如今咱们虽离城住着,终是天子脚下。这长安城中,遍地都是钱,只可惜没人会去拿去罢了。在家跳蹋会子也不中用。”狗儿听说,便急道:“你老只会炕头儿上混说,难道叫我打劫偷去不成?"刘姥姥道:“谁叫你偷去呢。也到底想法儿大家裁度,不然那银子钱自己跑到咱家来不成?"狗儿冷笑道:“有法儿还等到这会子呢。我又没有收税的亲戚,作官的朋友,有什么法子可想的?便有,也只怕他们未必来理我们呢!”

刘姥姥道:“这倒不然。谋事在人,成事在天。咱们谋到了,看菩萨的保佑,有些机会,也未可知。我倒替你们想出一个机会来。当日你们原是和金陵王家连过宗的,二十年前,他们看承你们还好,如今自然是你们拉硬屎,不肯去亲近他,故疏远起来。想当初我和女儿还去过一遭。他们家的二小姐着实响快,会待人,倒不拿大。如今现是荣国府贾二老爷的夫人。听得说,如今上了年纪,越发怜贫恤老,最爱斋僧敬道,舍米舍钱的。如今王府虽升了边任,只怕这二姑太太还认得咱们。你何不去走动走动,或者他念旧,有些好处,也未可知。要是他发一点好心,拔一根寒毛比咱们的腰还粗呢。”刘氏一旁接口道:“你老虽说的是,但只你我这样个嘴脸,怎样好到他门上去的。先不先,他们那些门上的人也未必肯去通信。没的去打嘴现世。”

谁知狗儿利名心最重,听如此一说,心下便有些活动起来。又听他妻子这话,便笑接道:“姥姥既如此说,况且当年你又见过这姑太太一次,何不你老人家明日就走一趟,先试试风头再说。”刘姥姥道:“嗳哟哟!可是说的,`侯门深似海-,我是个什么东西,他家人又不认得我,我去了也是白去的。”狗儿笑道:“不妨,我教你老人家一个法子:你竟带了外孙子板儿,先去找陪房周瑞,若见了他,就有些意思了。这周瑞先时曾和我父亲交过一件事,我们极好的。”刘姥姥道:“我也知道他的。只是许多时不走动,知道他如今是怎样。这也说不得了,你又是个男人,又这样个嘴脸,自然去不得,我们姑娘年轻媳妇子,也难卖头卖脚的,倒还是舍着我这付老脸去碰一碰。果然有些好处,大家都有益,便是没银子来,我也到那公府侯门见一见世面,也不枉我一生。”说毕,大家笑了一回。当晚计议已定。

次日天未明,刘姥姥便起来梳洗了,又将板儿教训了几句。那板儿才五六岁的孩子,一无所知,听见刘姥姥带他进城逛去,便喜的无不应承。于是刘姥姥带他进城,找至宁荣街。来至荣府大门石狮子前,只见簇簇轿马,刘姥姥便不敢过去,且掸了掸衣服,又教了板儿几句话,然后蹭到角门前。只见几个挺胸叠肚指手画脚的人,坐在大板凳上,说东谈西呢。刘姥姥只得蹭上来问:“太爷们纳福。”众人打量了他一会,便问"那里来的?"刘姥姥陪笑道:“我找太太的陪房周大爷的,烦那位太爷替我请他老出来。”那些人听了,都不瞅睬,半日方说道:“你远远的在那墙角下等着,一会子他们家有人就出来的。”内中有一老年人说道:“不要误他的事,何苦耍他。”因向刘姥姥道:“那周大爷已往南边去了。他在后一带住着,他娘子却在家。你要找时,从这边绕到后街上后门上去问就是了。”

刘姥姥听了谢过,遂携了板儿,绕到后门上。只见门前歇着些生意担子,也有卖吃的,也有卖顽耍物件的,闹吵吵三二十个小孩子在那里厮闹。刘姥姥便拉住一个道:“我问哥儿一声,有个周大娘可在家么?"孩子们道:“那个周大娘?我们这里周大娘有三个呢,还有两个周奶奶,不知是那一行当的?"刘姥姥道:“是太太的陪房周瑞。”孩子道:“这个容易,你跟我来。”说着,跳蹿蹿的引着刘姥姥进了后门,至一院墙边,指与刘姥姥道:“这就是他家。”又叫道:“周大娘,有个老奶奶来找你呢,我带了来了。”

周瑞家的在内听说,忙迎了出来,问:“是那位?"刘姥姥忙迎上来问道:“好呀,周嫂子!"周瑞家的认了半日,方笑道:“刘姥姥,你好呀!你说说,能几年,我就忘了。请家里来坐罢。”刘姥姥一壁里走着,一壁笑说道:“你老是贵人多忘事,那里还记得我们呢。”说着,来至房中。周瑞家的命雇的小丫头倒上茶来吃着。周瑞家的又问板儿道:“你都长这们大了!"又问些别后闲话。又问刘姥姥:“今日还是路过,还是特来的?"刘姥姥便说:“原是特来瞧瞧嫂子你,二则也请请姑太太的安。若可以领我见一见更好,若不能,便借重嫂子转致意罢了。”

周瑞家的听了,便已猜着几分来意。只因昔年他丈夫周瑞争买田地一事,其中多得狗儿之力,今见刘姥姥如此而来,心中难却其意,二则也要显弄自己的体面。听如此说,便笑说道:“姥姥你放心。大远的诚心诚意来了,岂有个不教你见个真佛去的呢。论理,人来客至回话,却不与我相干。我们这里都是各占一样儿:我们男的只管春秋两季地租子,闲时只带着小爷们出门子就完了,我只管跟太太奶奶们出门的事。皆因你原是太太的亲戚,又拿我当个人,投奔了我来,我就破个例,给你通个信去。但只一件,姥姥有所不知,我们这里又不比五年前了。如今太太竟不大管事*,都是琏二奶奶管家了。你道这琏二奶奶是谁?就是太太的内侄女,当日大舅老爷的女儿,小名凤哥的。”刘姥姥听了,罕问道:“原来是他!怪道呢,我当日就说他不错呢。这等说来,我今儿还得见他了。”周瑞家的道:“这自然的。如今太太事多心烦,有客来了,略可推得去的就推过去了,都是凤姑娘周旋迎待。今儿宁可不会太太,倒要见他一面,才不枉这里来一遭。”刘姥姥道:“阿弥陀佛!全仗嫂子方便了。”周瑞家的道:“说那里话。俗语说的:`与人方便,自己方便。-不过用我说一句话罢了,害着我什么。”说着,便叫小丫头到倒厅上悄悄的打听打听,老太太屋里摆了饭了没有。小丫头去了。这里二人又说些闲话。

刘姥姥因说:“这凤姑娘今年大还不过二十岁罢了,就这等有本事,当这样的家,可是难得的。”周瑞家的听了道:“我的姥姥,告诉不得你呢。这位凤姑娘年纪虽小,行事却比世人都大呢。如今出挑的美人一样的模样儿,少说些有一万个心眼子。再要赌口齿,十个会说话的男人也说他不过。回来你见了就信了。就只一件,待下人未免太严些个。”说着,只见小丫头回来说:“老太太屋里已摆完了饭了,二奶奶在太太屋里呢。”周瑞家的听了,连忙起身,催着刘姥姥说:“快走,快走。这一下来他吃饭是个空子,咱们先赶着去。若迟一步,回事的人也多了,难说话。再歇了中觉,越发没了时候了。”说着一齐下了炕,打扫打扫衣服,又教了板儿几句话,随着周瑞家的,逶迤往贾琏的住处来。先到了倒厅,周瑞家的将刘姥姥安插在那里略等一等。自己先过了影壁,进了院门,知凤姐未下来,先找着凤姐的一个心腹通房大丫头名唤平儿的。周瑞家的先将刘姥姥起初来历说明,又说:“今日大远的特来请安。当日太太是常会的,今日不可不见,所以我带了他进来了。等奶奶下来,我细细回明,奶奶想也不责备我莽撞的。”平儿听了,便作了主意:“叫他们进来,先在这里坐着就是了。”周瑞家的听了,方出去引他两个进入院来。上了正房台矶,小丫头打起猩红毡帘,才入堂屋,只闻一阵香扑了脸来,竟不辨是何气味,身子如在云端里一般。满屋中之物都耀眼争光的,使人头悬目眩。刘姥姥此时惟点头咂嘴念佛而已。于是来至东边这间屋内,乃是贾琏的女儿大姐儿睡觉之所。平儿站在炕沿边,打量了刘姥姥两眼,只得问个好让坐。刘姥姥见平儿遍身绫罗,插金带银,花容玉貌的,便当是凤姐儿了。才要称姑奶奶,忽见周瑞家的称他是平姑娘,又见平儿赶着周瑞家的称周大娘,方知不过是个有些体面的丫头了。于是让刘姥姥和板儿上了炕,平儿和周瑞家的对面坐在炕沿上,小丫头子斟了茶来吃茶。

刘姥姥只听见咯当咯当的响声,大有似乎打箩柜筛面的一般,不免东瞧西望的。忽见堂屋中柱子上挂着一个匣子,底下又坠着一个秤砣般一物,却不住的乱幌。刘姥姥心中想着:“这是什么爱物儿?有甚用呢?"正呆时,只听得当的一声,又若金钟铜磬一般,不防倒唬的一展眼。接着又是一连八九下。方欲问时,只见小丫头子们齐乱跑,说:“奶奶下来了。”周瑞家的与平儿忙起身,命刘姥姥"只管等着,是时候我们来请你。”说着,都迎出去了。

刘姥姥屏声侧耳默候。只听远远有人笑声,约有一二十妇人,衣裙ъл,渐入堂屋,往那边屋内去了。又见两三个妇人,都捧着大漆捧盒,进这边来等候。听得那边说了声"摆饭",渐渐的人才散出,只有伺候端菜的几个人。半日鸦雀不闻之后,忽见二人抬了一张炕桌来,放在这边炕上,桌上碗盘森列,仍是满满的鱼肉在内,不过略动了几样。板儿一见了,便吵着要肉吃,刘姥姥一巴掌打了他去。忽见周瑞家的笑嘻嘻走过来,招手儿叫他。刘姥姥会意,于是带了板儿下炕,至堂屋中,周瑞家的又和他唧咕了一会,方过这边屋里来。

只见门外錾铜钩上悬着大红撒花软帘,南窗下是炕,炕上大红毡条,靠东边板壁立着一个锁子锦靠背与一个引枕,铺着金心绿闪缎大坐褥,旁边有雕漆痰盒。那凤姐儿家常带着秋板貂鼠昭君套,围着攒珠勒子,穿着桃红撒花袄,石青刻丝灰鼠披风,大红洋绉银鼠皮裙,粉光脂艳,端端正正坐在那里,手内拿着小铜火箸儿拨手炉内的灰。平儿站在炕沿边,捧着小小的一个填漆茶盘,盘内一个小盖钟。凤姐也不接茶,也不抬头,只管拨手炉内的灰,慢慢的问道:“怎么还不请进来?"一面说,一面抬身要茶时,只见周瑞家的已带了两个人在地下站着呢。这才忙欲起身,犹未起身时,满面春风的问好,又嗔着周瑞家的怎么不早说。刘姥姥在地下已是拜了数拜,问姑奶奶安。凤姐忙说:“周姐姐,快搀起来,别拜罢,请坐。我年轻,不大认得,可也不知是什么辈数,不敢称呼。”周瑞家的忙回道:“这就是我才回的那姥姥了。”凤姐点头。刘姥姥已在炕沿上坐了。板儿便躲在背后,百般的哄他出来作揖,他死也不肯。

凤姐儿笑道:“亲戚们不大走动,都疏远了。知道的呢,说你们弃厌我们,不肯常来,不知道的那起小人,还只当我们眼里没人似的。”刘姥姥忙念佛道:“我们家道艰难,走不起,来了这里,没的给姑奶奶打嘴,就是管家爷们看着也不象。”凤姐儿笑道:“这话没的叫人恶心。不过借赖着祖父虚名,作了穷官儿,谁家有什么,不过是个旧日的空架子。俗语说,`朝廷还有三门子穷亲戚-呢,何况你我。”说着,又问周瑞家的回了太太了没有。周瑞家的道:“如今等奶奶的示下。”凤姐道:“你去瞧瞧,要是有人有事就罢,得闲儿呢就回,看怎么说。”周瑞家的答应着去了。

这里凤姐叫人抓些果子与板儿吃,刚问些闲话时,就有家下许多媳妇管事的来回话。平儿回了,凤姐道:“我这里陪客呢,晚上再来回。若有很要紧的,你就带进来现办。”平儿出去了,一会进来说:“我都问了,没什么紧事,我就叫他们散了。”凤姐点头。只见周瑞家的回来,向凤姐道:“太太说了,今日不得闲,二奶奶陪着便是一样。多谢费心想着。白来逛逛呢便罢,若有甚说的,只管告诉二奶奶,都是一样。”刘姥姥道:“也没甚说的,不过是来瞧瞧姑太太,姑奶奶,也是亲戚们的情分。”周瑞家的道:“没甚说的便罢,若有话,只管回二奶奶,是和太太一样的。”一面说,一面递眼色与刘姥姥。刘姥姥会意,未语先飞红的脸,欲待不说,今日又所为何来?只得忍耻说道:“论理今儿初次见姑奶奶,却不该说,只是大远的奔了你老这里来,也少不的说了。”刚说到这里,只听二门上小厮们回说:“东府里的小大爷进来了。”凤姐忙止刘姥姥:“不必说了。”一面便问:“你蓉大爷在那里呢?"只听一路靴子脚响,进来了一个十七八岁的少年,面目清秀,身材俊俏,轻裘宝带,美服华冠。刘姥姥此时坐不是,立不是,藏没处藏。凤姐笑道:“你只管坐着,这是我侄儿。”刘姥姥方扭扭捏捏在炕沿上坐了。

贾蓉笑道:“我父亲打发我来求婶子,说上回老舅太太给婶子的那架玻璃炕屏,明日请一个要紧的客,借了略摆一摆就送过来。”凤姐道:-说迟了一日,昨儿已经给了人了。”贾蓉听着,嘻嘻的笑着,在炕沿上半跪道:-婶子若不借,又说我不会说话了,又挨一顿好打呢。婶子只当可怜侄儿罢。”凤姐笑道:“也没见你们,王家的东西都是好的不成?你们那里放着那些好东西,只是看不见,偏我的就是好的。”贾蓉笑道:“那里有这个好呢!只求开恩罢。”凤姐道:“若碰一点儿,你可仔细你的皮!"因命平儿拿了楼房的钥匙,传几个妥当人抬去。贾蓉喜的眉开眼笑,说:“我亲自带了人拿去,别由他们乱碰。”说着便起身出去了。

这里凤姐忽又想起一事来,便向窗外叫:“蓉哥回来。”外面几个人接声说:“蓉大爷快回来。”贾蓉忙复身转来,垂手侍立,听何指示。那凤姐只管慢慢的吃茶,出了半日的神,又笑道:“罢了,你且去罢。晚饭后你来再说罢。这会子有人,我也没精神了。”贾蓉应了一声,方慢慢的退去。

这里刘姥姥心神方定,才又说道:“今日我带了你侄儿来,也不为别的,只因他老子娘在家里,连吃的都没有。如今天又冷了,越想没个派头儿,只得带了你侄儿奔了你老来。”说着又推板儿道:“你那爹在家怎么教你来?打发咱们作煞事来?只顾吃果子咧。”凤姐早已明白了,听他不会说话,因笑止道:“不必说了,我知道了。”因问周瑞家的:“这姥姥不知可用了早饭没有?"刘姥姥忙说道:“一早就往这里赶咧,那里还有吃饭的工夫咧。”凤姐听说,忙命快传饭来。一时周瑞家的传了一桌客饭来,摆在东边屋内,过来带了刘姥姥和板儿过去吃饭。凤姐说道:“周姐姐,好生让着些儿,我不能陪了。”于是过东边房里来。又叫过周瑞家的去,问他才回了太太,说了些什么?周瑞家的道:“太太说,他们家原不是一家子,不过因出一姓,当年又与太老爷在一处作官,偶然连了宗的。这几年来也不大走动。当时他们来一遭,却也没空了他们。今儿既来了瞧瞧我们,是他的好意思,也不可简慢了他。便是有什么说的,叫奶奶裁度着就是了。”凤姐听了说道:“我说呢,既是一家子,我如何连影儿也不知道。”

说话时,刘姥姥已吃毕了饭,拉了板儿过来,м舌咂嘴的道谢。凤姐笑道:“且请坐下,听我告诉你老人家。方才的意思,我已知道了。若论亲戚之间,原该不等上门来就该有照应才是。但如今家内杂事太烦,太太渐上了年纪,一时想不到也是有的。况是我近来接着管些事,都不知道这些亲戚们。二则外头看着虽是烈烈轰轰的,殊不知大有大的艰难去处,说与人也未必信罢。今儿你既老远的来了,又是头一次见我张口,怎好叫你空回去呢。可巧昨儿太太给我的丫头们做衣裳的二十两银子,我还没动呢,你若不嫌少,就暂且先拿了去罢。”

那刘姥姥先听见告艰难,只当是没有,心里便突突的,后来听见给他二十两,喜的又浑身发痒起来,说道:“嗳,我也是知道艰难的。但俗语说的:`瘦死的骆驼比马大-,凭他怎样,你老拔根寒毛比我们的腰还粗呢!"周瑞家的见他说的粗鄙,只管使眼色止他。凤姐看见,笑而不睬,只命平儿把昨儿那包银子拿来,再拿一吊钱来,都送到刘姥姥的跟前。凤姐乃道:“这是二十两银子,暂且给这孩子做件冬衣罢。若不拿着,就真是怪我了。这钱雇车坐罢。改日无事,只管来逛逛,方是亲戚们的意思。天也晚了,也不虚留你们了,到家里该问好的问个好儿罢。”一面说,一面就站了起来。

刘姥姥只管千恩万谢的,拿了银子钱,随了周瑞家的来至外面。周瑞家的道:“我的娘啊!你见了他怎么倒不会说了?开口就是`你侄儿-。我说句不怕你恼的话,便是亲侄儿,也要说和软些。蓉大爷才是他的正经侄儿呢,他怎么又跑出这么一个侄儿来了。”刘姥姥笑道:“我的嫂子,我见了他,心眼儿里爱还爱不过来,那里还说的上话来呢。”二人说着,又到周瑞家坐了片时。刘姥姥便要留下一块银子与周瑞家孩子们买果子吃,周瑞家的如何放在眼里,执意不肯。刘姥姥感谢不尽,仍从后门去了。正是:

得意浓时易接济,受恩深处胜亲朋。

\chapter{送宫花贾琏戏熙凤\ttlbreak 宴宁府宝玉会秦钟}

话说周瑞家的送了刘姥姥去后,便上来回王夫人话。谁知王夫人不在上房,问丫鬟们时,方知往薛姨妈那边闲话去了。周瑞家的听说,便转出东角门至东院,往梨香院来。刚至院门前,只见王夫人的丫鬟名金钏儿者,和一个才留了头的小女孩儿站在台阶坡上顽。见周瑞家的来了,便知有话回,因向内努嘴儿。

周瑞家的轻轻掀帘进去,只见王夫人和薛姨妈长篇大套的说些家务人情等语。周瑞家的不敢惊动,遂进里间来。只见薛宝钗穿着家常衣服,头上只散挽着シ儿,坐在炕里边,伏在小炕桌上同丫鬟莺儿正描花样子呢。见他进来,宝钗才放下笔,转过身来,满面堆笑让:“周姐姐坐。”周瑞家的也忙陪笑问:“姑娘好?"一面炕沿上坐了,因说:“这有两三天也没见姑娘到那边逛逛去,只怕是你宝兄弟冲撞了你不成?"宝钗笑道:“那里的话。只因我那种病又发了,所以这两天没出屋子。”周瑞家的道:“正是呢,姑娘到底有什么病根儿,也该趁早儿请个大夫来,好生开个方子,认真吃几剂,一势儿除了根才是。小小的年纪倒作下个病根儿,也不是顽的。”宝钗听了便笑道:“再不要提吃药。为这病请大夫吃药,也不知白花了多少银子钱呢。凭你什么名医仙药,从不见一点儿效。后来还亏了一个秃头和尚,说专治无名之症,因请他看了。他说我这是从胎里带来的一股热毒,幸而先天壮,还不相干,若吃寻常药,是不中用的。他就说了一个海上方,又给了一包药末子作引子,异香异气的。不知是那里弄了来的。他说发了时吃一丸就好。倒也奇怪,吃他的药倒效验些。”

周瑞家的因问:“不知是个什么海上方儿?姑娘说了,我们也记着,说与人知道,倘遇见这样病,也是行好的事。”宝钗见问,乃笑道:“不用这方儿还好,若用了这方儿,真真把人琐碎死。东西药料一概都有限,只难得`可巧-二字:要春天开的白牡丹花蕊十二两,夏天开的白荷花蕊十二两,秋天的白芙蓉蕊十二两,冬天的白梅花蕊十二两。将这四样花蕊,于次年春分这日晒干,和在药末子一处,一齐研好。又要雨水这日的雨水十二钱,……"周瑞家的忙道:“嗳哟!这么说来,这就得三年的工夫。倘或雨水这日竟不下雨,这却怎处呢?"宝钗笑道:“所以说那里有这样可巧的雨,便没雨也只好再等罢了。白露这日的露水十二钱,霜降这日的霜十二钱,小雪这日的雪十二钱。把这四样水调匀,和了药,再加十二钱蜂蜜,十二钱白糖,丸了龙眼大的丸子,盛在旧磁坛内,埋在花根底下。若发了病时,拿出来吃一丸,用十二分黄柏煎汤送下。”

周瑞家的听了笑道:“阿弥陀佛,真坑死人的事儿!等十年未必都这样巧的呢。”宝钗道:“竟好,自他说了去后,一二年间可巧都得了,好容易配成一料。如今从南带至北,现在就埋在梨花树底下呢。”周瑞家的又问道:“这药可有名子没有呢?"宝钗道:“有。这也是那癞头和尚说下的,叫作`冷香丸。”周瑞家的听了点头儿,因又说:“这病发了时到底觉怎么着?"宝钗道:“也不觉甚怎么着,只不过喘嗽些,吃一丸下去也就好些了。”

周瑞家的还欲说话时,忽听王夫人问:“谁在房里呢?"周瑞家的忙出去答应了,趁便回了刘姥姥之事。略待半刻,见王夫人无语,方欲退出,薛姨妈忽又笑道:“你且站住。我有一宗东西,你带了去罢。”说着便叫香菱。只听帘栊响处,方才和金钏顽的那个小丫头进来了,问:“奶奶叫我作什么?"薛姨妈道:“把匣子里的花儿拿来。”香菱答应了,向那边捧了个小锦匣来。薛姨妈道:“这是宫里头的新鲜样法,拿纱堆的花儿十二支。昨儿我想起来,白放着可惜了儿的,何不给他们姊妹们戴去。昨儿要送去,偏又忘了。你今儿来的巧,就带了去罢。你家的三位姑娘,每人一对,剩下的六枝,送林姑娘两枝,那四枝给了凤哥罢。”王夫人道:“留着给宝丫头戴罢,又想着他们作什么。”薛姨妈道:“姨娘不知道,宝丫头古怪着呢,他从来不爱这些花儿粉儿的。”

说着,周瑞家的拿了匣子,走出房门,见金钏仍在那里晒日阳儿。周瑞家的因问他道:“那香菱小丫头子,可就是常说临上京时买的,为他打人命官司的那个小丫头子么?"金钏道:“可不就是他。”正说着,只见香菱笑嘻嘻的走来。周瑞家的便拉了他的手,细细的看了一会,因向金钏儿笑道:“倒好个模样儿,竟有些象咱们东府里蓉大奶奶的品格儿。”金钏儿笑道:“我也是这们说呢。”周瑞家的又问香菱:“你几岁投身到这里?"又问:“你父母今在何处?今年十几岁了?本处是那里人?"香菱听问,都摇头说:“不记得了。”周瑞家的和金钏儿听了,倒反为叹息伤感一回。

一时间周瑞家的携花至王夫人正房后头来。原来近日贾母说孙女儿们太多了,一处挤着倒不方便,只留宝玉黛玉二人这边解闷,却将迎,探,惜三人移到王夫人这边房后三间小抱厦内居住,令李纨陪伴照管。如今周瑞家的故顺路先往这里来,只见几个小丫头子都在抱厦内听呼唤呢。迎春的丫鬟司棋与探春的丫鬟待书二人正掀帘子出来,手里都捧着茶钟,周瑞家的便知他们姊妹在一处坐着呢,遂进入内房,只见迎春探春二人正在窗下围棋。周瑞家的将花送上,说明缘故。二人忙住了棋,都欠身道谢,命丫鬟们收了。

周瑞家的答应了,因说:“四姑娘不在房里,只怕在老太太那边呢。”丫鬟们道:“那屋里不是四姑娘?"周瑞家的听了,便往这边屋里来。只见惜春正同水月庵的小姑子智能儿一处顽耍呢,见周瑞家的进来,惜春便问他何事。周瑞家的便将花匣打开,说明原故。惜春笑道:“我这里正和智能儿说,我明儿也剃了头同他作姑子去呢,可巧又送了花儿来,若剃了头,可把这花儿戴在那里呢?"说着,大家取笑一回,惜春命丫鬟入画来收了。

周瑞家的因问智能儿:“你是什么时候来的?你师父那秃歪剌往那里去了?"智能儿道:“我们一早就来了。我师父见了太太,就往于老爷府内去了,叫我在这里等他呢。”周瑞家的又道:“十五的月例香供银子可曾得了没有?"智能儿摇头儿说:“我不知道。”惜春听了,便问周瑞家的:“如今各庙月例银子是谁管着?"周瑞家的道:“是余信管着。”惜春听了笑道:“这就是了。他师父一来,余信家的就赶上来,和他师父咕唧了半日,想是就为这事了。”

那周瑞家的又和智能儿劳叨了一会,便往凤姐儿处来。穿夹道从李纨后窗下过,隔着玻璃窗户,见李纨在炕上歪着睡觉呢,遂越过西花墙,出西角门进入凤姐院中。走至堂屋,只见小丫头丰儿坐在凤姐房中门槛上,见周瑞家的来了,连忙摆手儿叫他往东屋里去。周瑞家的会意,忙蹑手蹑足往东边房里来,只见奶子正拍着大姐儿睡觉呢。周瑞家的悄问奶子道:“姐儿睡中觉呢?也该请醒了。”奶子摇头儿。正说着,只听那边一阵笑声,却有贾琏的声音。接着房门响处,平儿拿着大铜盆出来,叫丰儿舀水进去。平儿便到这边来,一见了周瑞家的便问:“你老人家又跑了来作什么?"周瑞家的忙起身,拿匣子与他,说送花儿一事。平儿听了,便打开匣子,拿了四枝,转身去了。半刻工夫,手里拿出两枝来,先叫彩明吩咐道:“送到那边府里给小蓉大奶奶戴去。”次后方命周瑞家的回去道谢。

周瑞家的这才往贾母这边来。穿过了穿堂,抬头忽见他女儿打扮着才从他婆家来。周瑞家的忙问:“你这会跑来作什么?"他女儿笑道:“妈一向身上好?我在家里等了这半日,妈竟不出去,什么事情这样忙的不回家?我等烦了,自己先到了老太太跟前请了安了,这会子请太太的安去。妈还有什么不了的差事,手里是什么东西?"周瑞家的笑道:“嗳!今儿偏偏的来了个刘姥姥,我自己多事,为他跑了半日,这会子又被姨太太看见了,送这几枝花儿与姑娘奶奶们。这会子还没送清楚呢。你这会子跑了来,一定有什么事。”他女儿笑道:“你老人家倒会猜。实对你老人家说,你女婿前儿因多吃了两杯酒,和人分争,不知怎的被人放了一把邪火,说他来历不明,告到衙门里,要递解还乡。所以我来和你老人家商议商议,这个情分,求那一个可了事呢?"周瑞家的听了道:“我就知道呢。这有什么大不了的事!你且家去等我,我给林姑娘送了花儿去就回家去。此时太太二奶奶都不得闲儿,你回去等我。这有什么,忙的如此。”女儿听说,便回去了,又说:“妈,好歹快来。”周瑞家的道:“是了。小人儿家没经过什么事,就急得你这样了。”说着,便到黛玉房中去了。

谁知此时黛玉不在自己房中,却在宝玉房中大家解九连环顽呢。周瑞家的进来笑道:“林姑娘,姨太太着我送花儿与姑娘带来了。”宝玉听说,便先问:“什么花儿?拿来给我。”一面早伸手接过来了。开匣看时,原来是宫制堆纱新巧的假花儿。黛玉只就宝玉手中看了一看,便问道:“还是单送我一人的,还是别的姑娘们都有呢?"周瑞家的道:“各位都有了,这两枝是姑娘的了。”黛玉冷笑道:“我就知道,别人不挑剩下的也不给我。”周瑞家的听了,一声儿不言语。宝玉便问道:“周姐姐,你作什么到那边去了。”周瑞家的因说:“太太在那里,因回话去了,姨太太就顺便叫我带来了。”宝玉道:“宝姐姐在家作什么呢?怎么这几日也不过这边来?"周瑞家的道:“身上不大好呢。”宝玉听了,便和丫头说:“谁去瞧瞧?只说我与林姑娘打发了来请姨太太姐姐安,问姐姐是什么病,现吃什么药。论理我该亲自来的,就说才从学里来,也着了些凉,异日再亲自来看罢。”说着,茜雪便答应去了。周瑞家的自去,无话。原来这周瑞的女婿,便是雨村的好友冷子兴,近因卖古董和人打官司,故教女人来讨情分。周瑞家的仗着主子的势利,把这些事也不放在心上,晚间只求求凤姐儿便完了。至掌灯时分,凤姐已卸了妆,来见王夫人回话:“今儿甄家送了来的东西,我已收了。咱们送他的,趁着他家有年下进鲜的船回去,一并都交给他们带了去罢?"王夫人点头。凤姐又道:“临安伯老太太生日的礼已经打点了,派谁送去呢?"王夫人道:“你瞧谁闲着,就叫他们去四个女人就是了,又来当什么正经事问我。”凤姐又笑道:“今日珍大嫂子来,请我明日过去逛逛,明日倒没有什么事情。”王夫人道:“有事没事都害不着什么。每常他来请,有我们,你自然不便意,他既不请我们,单请你,可知是他诚心叫你散淡散淡,别辜负了他的心,便有事也该过去才是。”凤姐答应了。当下李纨,迎,探等姐妹们亦来定省毕,各自归房无话。

次日凤姐梳洗了,先回王夫人毕,方来辞贾母。宝玉听了,也要跟了逛去。凤姐只得答应,立等着换了衣服,姐儿两个坐了车,一时进入宁府。早有贾珍之妻尤氏与贾蓉之妻秦氏婆媳两个,引了多少姬妾丫鬟媳妇等接出仪门。那尤氏一见了凤姐,必先笑嘲一阵,一手携了宝玉同入上房来归坐。秦氏献茶毕,凤姐因说:“你们请我来作什么?有什么好东西孝敬我,就快献上来,我还有事呢。”尤氏秦氏未及答话,地下几个姬妾先就笑说:“二奶奶今儿不来就罢,既来了就依不得二奶奶了。”正说着,只见贾蓉进来请安。宝玉因问:“大哥哥今日不在家么?"尤氏道:“出城与老爷请安去了。可是你怪闷的,坐在这里作什么?何不也去逛逛?”

秦氏笑道:“今儿巧,上回宝叔立刻要见的我那兄弟,他今儿也在这里,想在书房里呢,宝叔何不去瞧一瞧?"宝玉听了,即便下炕要走。尤氏凤姐都忙说:“好生着,忙什么?"一面便吩咐好生小心跟着,别委曲着他,倒比不得跟了老太太过来就罢了。凤姐说道:“既这么着,何不请进这秦小爷来,我也瞧一瞧。难道我见不得他不成?"尤氏笑道:“罢,罢!可以不必见他,比不得咱们家的孩子们,胡打海摔的惯了。人家的孩子都是斯斯文文的惯了,乍见了你这破落户,还被人笑话死了呢。”凤姐笑道:“普天下的人,我不笑话就罢了,竟叫这小孩子笑话我不成?"贾蓉笑道:“不是这话,他生的腼腆,没见过大阵仗儿,婶子见了,没的生气。”凤姐道:“凭他什么样儿的,我也要见一见!别放你娘的屁了。再不带我看看,给你一顿好嘴巴。”贾蓉笑嘻嘻的说:“我不敢扭着,就带他来。”

说着,果然出去带进一个小后生来,较宝玉略瘦些,眉清目秀,粉面朱唇,身材俊俏,举止风流,似在宝玉之上,只是怯怯羞羞,有女儿之态,腼腆含糊,慢向凤姐作揖问好。凤姐喜的先推宝玉,笑道:“比下去了!"便探身一把携了这孩子的手,就命他身傍坐了,慢慢的问他:几岁了,读什么书,弟兄几个,学名唤什么。秦钟一一答应了。早有凤姐的丫鬟媳妇们见凤姐初会秦钟,并未备得表礼来,遂忙过那边去告诉平儿。平儿知道凤姐与秦氏厚密,虽是小后生家,亦不可太俭,遂自作主意,拿了一匹尺头,两个"状元及第"的小金锞子,交付与来人送过去。凤姐犹笑说太简薄等语。秦氏等谢毕。一时吃过饭,尤氏,凤姐,秦氏等抹骨牌,不在话下。

那宝玉自见了秦钟的人品出众,心中似有所失,痴了半日,自己心中又起了呆意,乃自思道:“天下竟有这等人物!如今看来,我竟成了泥猪癞狗了。可恨我为什么生在这侯门公府之家,若也生在寒门薄宦之家,早得与他交结,也不枉生了一世。我虽如此比他尊贵,可知锦绣纱罗,也不过裹了我这根死木头,美酒羊羔,也不过填了我这粪窟泥沟。`富贵-二字,不料遭我荼毒了!"秦钟自见了宝玉形容出众,举止不凡,更兼金冠绣服,骄婢侈童,秦钟心中亦自思道:“果然这宝玉怨不得人溺爱他。可恨我偏生于清寒之家,不能与他耳鬓交接,可知`贫窭-二字限人,亦世间之大不快事。”二人一样的胡思乱想。忽然宝玉问他读什么书。秦钟见问,因而答以实话。二人你言我语,十来句后,越觉亲密起来。

一时摆上茶果,宝玉便说:“我两个又不吃酒,把果子摆在里间小炕上,我们那里坐去,省得闹你们。”于是二人进里间来吃茶。秦氏一面张罗与凤姐摆酒果,一面忙进来嘱宝玉道:“宝叔,你侄儿倘或言语不防头,你千万看着我,不要理他。他虽腼腆,却性子左强,不大随和此是有的。”宝玉笑道:“你去罢,我知道了。”秦氏又嘱了他兄弟一回,方去陪凤姐。

一时凤姐尤氏又打发人来问宝玉:“要吃什么,外面有,只管要去。”宝玉只答应着,也无心在饮食上,只问秦钟近日家务等事。秦钟因说:“业师于去年病故,家父又年纪老迈,残疾在身,公务繁冗,因此尚未议及再延师一事,目下不过在家温习旧课而已。再读书一事,必须有一二知己为伴,时常大家讨论,才能进益。”宝玉不待说完,便答道:“正是呢,我们却有个家塾,合族中有不能延师的,便可入塾读书,子弟们中亦有亲戚在内可以附读。我因业师上年回家去了,也现荒废着呢。家父之意,亦欲暂送我去温习旧书,待明年业师上来,再各自在家里读。家祖母因说:一则家学里之子弟太多,生恐大家淘气,反不好,二则也因我病了几天,遂暂且耽搁着。如此说来,尊翁如今也为此事悬心。今日回去,何不禀明,就往我们敝塾中来,我亦相伴,彼此有益,岂不是好事?"秦钟笑道:“家父前日在家提起延师一事,也曾提起这里的义学倒好,原要来和这里的亲翁商议引荐。因这里又事忙,不便为这点小事来聒絮的。宝叔果然度小侄或可磨墨涤砚,何不速速的作成,又彼此不致荒废,又可以常相谈聚,又可以慰父母之心,又可以得朋友之乐,岂不是美事?"宝玉道:“放心,放心。咱们回来告诉你姐夫姐姐和琏二嫂子。你今日回家就禀明令尊,我回去再禀明祖母,再无不速成之理。”二人计议一定。那天气已是掌灯时候,出来又看他们顽了一回牌。算帐时,却又是秦氏尤氏二人输了戏酒的东道,言定后日吃这东道。一面就叫送饭。

吃毕晚饭,因天黑了,尤氏说:“先派两个小子送了这秦相公家去。”媳妇们传出去半日,秦钟告辞起身。尤氏问:“派了谁送去?"媳妇们回说:“外头派了焦大,谁知焦大醉了,又骂呢。”尤氏秦氏都说道:“偏又派他作什么!放着。这些小子们,那一个派不得?偏要惹他去。”凤姐道:“我成日家说你太软弱了,纵的家里人这样还了得了。”尤氏叹道:“你难道不知这焦大的?连老爷都不理他的,你珍大哥哥也不理他。只因他从小儿跟着太爷们出过三四回兵,从死人堆里把太爷背了出来,得了命,自己挨着饿,却偷了东西来给主子吃,两日没得水,得了半碗水给主子喝,他自己喝马溺。不过仗着这些功劳情分,有祖宗时都另眼相待,如今谁肯难为他去。他自己又老了,又不顾体面,一味吃酒,吃醉了,无人不骂。我常说给管事的,不要派他差事,全当一个死的就完了。今儿又派了他。”凤姐道:“我何曾不知这焦大。倒是你们没主意,有这样的,何不打发他远远的庄子上去就完了。”说着,因问:“我们的车可齐备了?"地下众人都应道:“伺候齐了。”

凤姐起身告辞,和宝玉携手同行。尤氏等送至大厅,只见灯烛辉煌,众小厮都在丹墀侍立。那焦大又恃贾珍不在家,即在家亦不好怎样他,更可以任意洒落洒落。因趁着酒兴,先骂大总管赖二,说他不公道,欺软怕硬,"有了好差事就派别人,象这等黑更半夜送人的事,就派我。没良心的王八羔子!瞎充管家!你也不想想,焦大太爷跷跷脚,比你的头还高呢。二十年头里的焦大太爷眼里有谁?别说你们这一起杂种王八羔子们!"正骂的兴头上,贾蓉送凤姐的车出去,众人喝他不听,贾蓉忍不得,便骂了他两句,使人捆起来,"等明日酒醒了,问他还寻死不寻死了!"那焦大那里把贾蓉放在眼里,反大叫起来,赶着贾蓉叫:“蓉哥儿,你别在焦大跟前使主子性儿。别说你这样儿的,就是你爹,你爷爷,也不敢和焦大挺腰子!不是焦大一个人,你们就做官儿享荣华受富贵?你祖宗九死一生挣下这家业,到如今了,不报我的恩,反和我充起主子来了。不和我说别的还可,若再说别的,咱们红刀子进去白刀子出来!"凤姐在车上说与贾蓉道:“以后还不早打发了这个没王法的东西!留在这里岂不是祸害?倘或亲友知道了,岂不笑话咱们这样的人家,连个王法规矩都没有。”贾蓉答应"是"。

众小厮见他太撒野了,只得上来几个,揪翻捆倒,拖往马圈里去。焦大越发连贾珍都说出来,乱嚷乱叫说:“我要往祠堂里哭太爷去。那里承望到如今生下这些畜牲来!每日家偷狗戏鸡,爬灰的爬灰,养小叔子的养小叔子,我什么不知道?咱们`胳膊折了往袖子里藏-!"众小厮听他说出这些没天日的话来,唬的魂飞魄散,也不顾别的了,便把他捆起来,用土和马粪满满的填了他一嘴。

凤姐和贾蓉等也遥遥的闻得,便都装作没听见。宝玉在车上见这般醉闹,倒也有趣,因问凤姐道:“姐姐,你听他说`爬灰的爬灰-,什么是`爬灰-?"凤姐听了,连忙立眉嗔目断喝道:“少胡说!那是醉汉嘴里混吣,你是什么样的人,不说没听见,还倒细问!等我回去回了太太,仔细捶你不捶你!"唬的宝玉忙央告道:“好姐姐,我再不敢了。”凤姐道:“这才是呢。等到了家,咱们回了老太太,打发你同秦家侄儿学里念书去要紧。”说着,却自回往荣府而来。正是:

不因俊俏难为友,正为风流始读书。

\chapter{比通灵金莺微露意\ttlbreak 探宝钗黛玉半含酸}

话说凤姐和宝玉回家,见过众人。宝玉先便回明贾母秦钟要上家塾之事,自己也有了个伴读的朋友,正好发奋,又着实的称赞秦钟的人品行事,最使人怜爱。凤姐又在一旁帮着说"过日他还来拜老祖宗"等语,说的贾母喜欢起来。凤姐又趁势请贾母后日过去看戏。贾母虽年老,却极有兴头。至后日,又有尤氏来请,遂携了王夫人林黛玉宝玉等过去看戏。至晌午,贾母便回来歇息了。王夫人本是好清净的,见贾母回来也就回来了。然后凤姐坐了首席,尽欢至晚无话。

却说宝玉因送贾母回来,待贾母歇了中觉,意欲还去看戏取乐,又恐扰的秦氏等人不便,因想起近日薛宝钗在家养病,未去亲候,意欲去望他一望。若从上房后角门过去,又恐遇见别事缠绕,再或可巧遇见他父亲,更为不妥,宁可绕远路罢了。当下众嬷嬷丫鬟伺候他换衣服,见他不换,仍出二门去了,众嬷嬷丫鬟只得跟随出来,还只当他去那府中看戏。谁知到穿堂,便向东向北绕厅后而去。偏顶头遇见了门下清客相公詹光单聘仁二人走来,一见了宝玉,便都笑着赶上来,一个抱住腰,一个携着手,都道:“我的菩萨哥儿,我说作了好梦呢,好容易得遇见了你。”说着,请了安,又问好,劳叨半日,方才走开。老嬷嬷叫住,因问:“二位爷是从老爷跟前来的不是?"二人点头道:“老爷在梦坡斋小书房里歇中觉呢,不妨事的。”一面说,一面走了。说的宝玉也笑了。于是转弯向北奔梨香院来。可巧银库房的总领名唤吴新登与仓上的头目名戴良,还有几个管事的头目,共有七个人,从帐房里出来,一见了宝玉,赶来都一齐垂手站住。独有一个买办名唤钱华,因他多日未见宝玉,忙上来打千儿请安,宝玉忙含笑携他起来。众人都笑说:“前儿在一处看见二爷写的斗方儿,字法越发好了,多早晚儿赏我们几张贴贴。”宝玉笑道:“在那里看见了?"众人道:“好几处都有,都称赞的了不得,还和我们寻呢。”宝玉笑道:“不值什么,你们说与我的小幺儿们就是了。”一面说,一面前走,众人待他过去,方都各自散了。

闲言少述,且说宝玉来至梨香院中,先入薛姨妈室中来,正见薛姨妈打点针黹与丫鬟们呢。宝玉忙请了安,薛姨妈忙一把拉了他,抱入怀内,笑说:“这们冷天,我的儿,难为你想着来,快上炕来坐着罢。”命人倒滚滚的茶来。宝玉因问:“哥哥不在家?"薛姨妈叹道:“他是没笼头的马,天天忙不了,那里肯在家一日。”宝玉道:“姐姐可大安了?"薛姨妈道:“可是呢,你前儿又想着打发人来瞧他。他在里间不是,你去瞧他,里间比这里暖和,那里坐着,我收拾收拾就进去和你说话儿。”宝玉听说,忙下了炕来至里间门前,只见吊着半旧的红н软帘。宝玉掀帘一迈步进去,先就看见薛宝钗坐在炕上作针线,头上挽着漆黑油光的シ儿,蜜合色棉袄,玫瑰紫二色金银鼠比肩褂,葱黄绫棉裙,一色半新不旧,看去不觉奢华。唇不点而红,眉不画而翠,脸若银盆,眼如水杏。罕言寡语,人谓藏愚,安分随时,自云守拙。宝玉一面看,一面问:“姐姐可大愈了?"宝钗抬头只见宝玉进来,连忙起身含笑答说:“已经大好了,倒多谢记挂着。”说着,让他在炕沿上坐了,即命莺儿斟茶来。一面又问老太太姨娘安,别的姐妹们都好。一面看宝玉头上戴着魉壳侗ψ辖鸸冢额上勒着二龙抢珠金抹额,身上穿着秋香色立蟒白狐腋箭袖,系着五色蝴蝶鸾绦,项上挂着长命锁,记名符,另外有一块落草时衔下来的宝玉。宝钗因笑说道:“成日家说你的这玉,究竟未曾细细的赏鉴,我今儿倒要瞧瞧。”说着便挪近前来。宝玉亦凑了上去,从项上摘了下来,递在宝钗手内。宝钗托于掌上,只见大如雀卵,灿若明霞,莹润如酥,五色花纹缠护。这就是大荒山中青埂峰下的那块顽石的幻相。后人曾有诗嘲云:

女娲炼石已荒唐,又向荒唐演大荒。

失去幽灵真境界,幻来亲就臭皮囊。

好知运败金无彩,堪叹时乖玉不光。

白骨如山忘姓氏,无非公子与红妆。那顽石亦曾记下他这幻相并癞僧所镌的篆文,今亦按图画于后。但其真体最小,方能从胎中小儿口内衔下。今若按其体画,恐字迹过于微细,使观者大废眼光,亦非畅事。故今只按其形式,无非略展些规矩,使观者便于灯下醉中可阅。今注明此故,方无胎中之儿口有多大,怎得衔此狼о蠢大之物等语之谤。

通灵宝玉正面图式

通灵宝玉

注云莫失莫忘仙寿恒昌

通灵宝玉反面图式

注云一除邪祟二疗п疾三知祸福

宝钗看毕,又从新翻过正面来细看,口内念道:“莫失莫忘,仙寿恒昌。”念了两遍,乃回头向莺儿笑道:“你不去倒茶,也在这里发呆作什么?"莺儿嘻嘻笑道:“我听这两句话,倒象和姑娘的项圈上的两句话是一对儿。”宝玉听了,忙笑道:“原来姐姐那项圈上也有八个字,我也赏鉴赏鉴。”宝钗道:“你别听他的话,没有什么字。”宝玉笑央:“好姐姐,你怎么瞧我的了呢。”宝钗被缠不过,因说道:“也是个人给了两句吉利话儿,所以錾上了,叫天天带着,不然,沉甸甸的有什么趣儿。”一面说,一面解了排扣,从里面大红袄上将那珠宝晶莹黄金灿烂的璎珞掏将出来。宝玉忙托了锁看时,果然一面有四个篆字,两面八字,共成两句吉谶。亦曾按式画下形相:

音注云不离不弃

音注云芳龄永继宝玉看了,也念了两遍,又念自己的两遍,因笑问:“姐姐这八个字倒真与我的是一对。”莺儿笑道:“是个癞头和尚送的,他说必须錾在金器上-"宝钗不待说完,便嗔他不去倒茶,一面又问宝玉从那里来。

宝玉此时与宝钗就近,只闻一阵阵凉森森甜丝丝的幽香,竟不知系何香气,遂问:“姐姐熏的是什么香?我竟从未闻见过这味儿。”宝钗笑道:“我最怕熏香,好好的衣服,熏的烟燎火气的。”宝玉道:“既如此,这是什么香?"宝钗想了一想,笑道:“是了,是我早起吃了丸药的香气。”宝玉笑道:“什么丸药这么好闻?好姐姐,给我一丸尝尝。”宝钗笑道:“又混闹了,一个药也是混吃的?”

一语未了,忽听外面人说:“林姑娘来了。”话犹未了,林黛玉已摇摇的走了进来,一见了宝玉,便笑道:“嗳哟,我来的不巧了!"宝玉等忙起身笑让坐,宝钗因笑道:“这话怎么说?"黛玉笑道:“早知他来,我就不来了。”宝钗道:“我更不解这意。”黛玉笑道:“要来一群都来,要不来一个也不来,今儿他来了,明儿我再来,如此间错开了来着,岂不天天有人来了?也不至于太冷落,也不至于太热闹了。姐姐如何反不解这意思?”

宝玉因见他外面罩着大红羽缎对衿褂子,因问:“下雪了么?"地下婆娘们道:“下了这半日雪珠儿了。”宝玉道:“取了我的斗篷来不曾?"黛玉便道:“是不是,我来了他就该去了。”宝玉笑道:“我多早晚儿说要去了?不过拿来预备着。”宝玉的奶母李嬷嬷因说道:“天又下雪,也好早晚的了,就在这里同姐姐妹妹一处顽顽罢。姨妈那里摆茶果子呢。我叫丫头去取了斗篷来,说给小幺儿们散了罢。”宝玉应允。李嬷嬷出去,命小厮们都各散去不提。

这里薛姨妈已摆了几样细茶果来留他们吃茶。宝玉因夸前日在那府里珍大嫂子的好鹅掌鸭信。薛姨妈听了,忙也把自己糟的取了些来与他尝。宝玉笑道:“这个须得就酒才好。”薛姨妈便令人去灌了最上等的酒来。李嬷嬷便上来道:“姨太太,酒倒罢了。”宝玉央道:“妈妈,我只喝一钟。”李嬷嬷道:“不中用!当着老太太,太太,那怕你吃一坛呢。想那日我眼错不见一会,不知是那一个没调教的,只图讨你的好儿,不管别人死活,给了你一口酒吃,葬送的我挨了两日骂。姨太太不知道,他性子又可恶,吃了酒更弄性。有一日老太太高兴了,又尽着他吃,什么日子又不许他吃,何苦我白赔在里面。”薛姨妈笑道:“老货,你只放心吃你的去。我也不许他吃多了。便是老太太问,有我呢。”一面令小丫鬟:“来,让你奶奶们去,也吃杯搪搪雪气。”那李嬷嬷听如此说,只得和众人去吃些酒水。这里宝玉又说:“不必温暖了,我只爱吃冷的。”薛姨妈忙道:“这可使不得,吃了冷酒,写字手打р儿。”宝钗笑道:“宝兄弟,亏你每日家杂学旁收的,难道就不知道酒性最热,若热吃下去,发散的就快,若冷吃下去,便凝结在内,以五脏去暖他,岂不受害?从此还不快不要吃那冷的了。”宝玉听这话有情理,便放下冷酒,命人暖来方饮。

黛玉磕着瓜子儿,只抿着嘴笑。可巧黛玉的小丫鬟雪雁走来与黛玉送小手炉,黛玉因含笑问他:“谁叫你送来的?难为他费心,那里就冷死了我!"雪雁道:“紫鹃姐姐怕姑娘冷,使我送来的。”黛玉一面接了,抱在怀中,笑道:“也亏你倒听他的话。我平日和你说的,全当耳旁风,怎么他说了你就依,比圣旨还快些!"宝玉听这话,知是黛玉借此奚落他,也无回复之词,只嘻嘻的笑两阵罢了。宝钗素知黛玉是如此惯了的,也不去睬他。薛姨妈因道:“你素日身子弱,禁不得冷的,他们记挂着你倒不好?"黛玉笑道:“姨妈不知道。幸亏是姨妈这里,倘或在别人家,人家岂不恼?好说就看的人家连个手炉也没有,巴巴的从家里送个来。不说丫鬟们太小心过余,还只当我素日是这等轻狂惯了呢。”薛姨妈道:“你这个多心的,有这样想,我就没这样心。”

说话时,宝玉已是三杯过去。李嬷嬷又上来拦阻。宝玉正在心甜意洽之时,和宝黛姊妹说说笑笑的,那肯不吃。宝玉只得屈意央告:“好妈妈,我再吃两钟就不吃了。”李嬷嬷道:“你可仔细老爷今儿在家,с防问你的书!"宝玉听了这话,便心中大不自在,慢慢的放下酒,垂了头。黛玉先忙的说:“别扫大家的兴!舅舅若叫你,只说姨妈留着呢。这个妈妈,他吃了酒,又拿我们来醒脾了!"一面悄推宝玉,使他赌气,一面悄悄的咕哝说:“别理那老货,咱们只管乐咱们的。”那李嬷嬷不知黛玉的意思,因说道:“林姐儿,你不要助着他了。你倒劝劝他,只怕他还听些。”林黛玉冷笑道:“我为什么助他?我也不犯着劝他。你这妈妈太小心了,往常老太太又给他酒吃,如今在姨妈这里多吃一口,料也不妨事。必定姨妈这里是外人,不当在这里的也未可定。”李嬷嬷听了,又是急,又是笑,说道:“真真这林姐儿,说出一句话来,比刀子还尖。你这算了什么。”宝钗也忍不住笑着,把黛玉腮上一拧,说道:“真真这个颦丫头的一张嘴,叫人恨又不是,喜欢又不是。”薛姨妈一面又说:“别怕,别怕,我的儿!来这里没好的你吃,别把这点子东西唬的存在心里,倒叫我不安。只管放心吃,都有我呢。越发吃了晚饭去,便醉了,就跟着我睡罢。”因命:“再烫热酒来!姨妈陪你吃两杯,可就吃饭罢。”宝玉听了,方又鼓起兴来。

李嬷嬷因吩咐小丫头子们:“你们在这里小心着,我家里换了衣服就来,悄悄的回姨太太,别由着他,多给他吃。”说着便家去了。这里虽还有三两个婆子,都是不关痛痒的,见李嬷嬷走了,也都悄悄去寻方便去了。只剩了两个小丫头子,乐得讨宝玉的欢喜。幸而薛姨妈千哄万哄的,只容他吃了几杯,就忙收过了。作酸笋鸡皮汤,宝玉痛喝了两碗,吃了半碗碧粳粥。一时薛林二人也吃完了饭,又酽酽的沏上茶来大家吃了。薛姨妈方放了心。雪雁等三四个丫头已吃了饭,进来伺候。黛玉因问宝玉道:“你走不走?"宝玉乜斜倦眼道:“你要走,我和你一同走。”黛玉听说,遂起身道:“咱们来了这一日,也该回去了。还不知那边怎么找咱们呢。”说着,二人便告辞。

小丫头忙捧过斗笠来,宝玉便把头略低一低,命他戴上。那丫头便将着大红猩毡斗笠一抖,才往宝玉头上一合,宝玉便说:“罢,罢!好蠢东西,你也轻些儿!难道没见过别人戴过的?让我自己戴罢。”黛玉站在炕沿上道:“罗唆什么,过来,我瞧瞧罢。”宝玉忙就近前来。黛玉用手整理,轻轻笼住束发冠,将笠沿掖在抹额之上,将那一颗核桃大的绛绒簪缨扶起,颤巍巍露于笠外。整理已毕,端相了端相,说道:“好了,披上斗篷罢。”宝玉听了,方接了斗篷披上。薛姨妈忙道:“跟你们的妈妈都还没来呢,且略等等不迟。”宝玉道:“我们倒去等他们,有丫头们跟着也够了。”薛姨妈不放心,到底命两个妇女跟随他兄妹方罢。他二人道了扰,一径回至贾母房中。

贾母尚未用晚饭,知是薛姨妈处来,更加喜欢。因见宝玉吃了酒,遂命他自回房去歇着,不许再出来了。因命人好生看侍着。忽想起跟宝玉的人来,遂问众人:“李奶子怎么不见?"众人不敢直说家去了,只说:“才进来的,想有事才去了。”宝玉踉跄回头道:“他比老太太还受用呢,问他作什么!没有他只怕我还多活两日。”一面说,一面来至自己的卧室。只见笔墨在案,晴雯先接出来,笑说道:“好,好,要我研了那些墨,早起高兴,只写了三个字,丢下笔就走了,哄的我们等了一日。快来与我写完这些墨才罢!"宝玉忽然想起早起的事来,因笑道:“我写的那三个字在那里呢?"晴雯笑道:“这个人可醉了。你头里过那府里去,嘱咐贴在这门斗上,这会子又这么问。我生怕别人贴坏了,我亲自爬高上梯的贴上,这会子还冻的手僵冷的呢。”宝玉听了,笑道:“我忘了。你的手冷,我替你渥着。”说着便伸手携了晴雯的手,同仰首看门斗上新书的三个字。

一时黛玉来了,宝玉笑道:“好妹妹,你别撒谎,你看这三个字那一个好?"黛玉仰头看里间门斗上,新贴了三个字,写着"绛云轩"。黛玉笑道:“个个都好。怎么写的这们好了?明儿也与我写一个匾。”宝玉嘻嘻的笑道:“又哄我呢。”说着又问:“袭人姐姐呢?"晴雯向里间炕上努嘴。宝玉一看,只见袭人和衣睡着在那里。宝玉笑道:“好,太渥早了些。”因又问晴雯道:“今儿我在那府里吃早饭,有一碟子豆腐皮的包子,我想着你爱吃,和珍大奶奶说了,只说我留着晚上吃,叫人送过来的,你可吃了?"晴雯道:“快别提。一送了来,我知道是我的,偏我才吃了饭,就放在那里。后来李奶奶来了看见,说:`宝玉未必吃了,拿了给我孙子吃去罢。-他就叫人拿了家去了。”接着茜雪捧上茶来。宝玉因让"林妹妹吃茶。”众人笑说:“林妹妹早走了,还让呢。”

宝玉吃了半碗茶,忽又想起早起的茶来,因问茜雪道:“早起沏了一碗枫露茶,我说过,那茶是三四次后才出色的,这会子怎么又沏了这个来?"茜雪道:“我原是留着的,那会子李奶奶来了,他要尝尝,就给他吃了。”宝玉听了,将手中的茶杯只顺手往地下一掷,豁啷一声,打了个粉碎,泼了茜雪一裙子的茶。又跳起来问着茜雪道:“他是你那一门子的奶奶,你们这么孝敬他?不过是仗着我小时候吃过他几日奶罢了。如今逞的他比祖宗还大了。如今我又吃不着奶了,白白的养着祖宗作什么!撵了出去,大家干净!"说着便要去立刻回贾母,撵他侞母。原来袭人实未睡着,不过故意装睡,引宝玉来怄他顽耍。先闻得说字问包子等事,也还可不必起来,后来摔了茶钟,动了气,遂连忙起来解释劝阻。早有贾母遣人来问是怎么了。袭人忙道:“我才倒茶来,被雪滑倒了,失手砸了钟子。”一面又安慰宝玉道:“你立意要撵他也好,我们也都愿意出去,不如趁势连我们一齐撵了,我们也好,你也不愁再有好的来伏侍你。”宝玉听了这话,方无了言语,被袭人等扶至炕上,脱换了衣服。不知宝玉口内还说些什么,只觉口齿缠绵,眼眉愈加饧涩,忙伏侍他睡下。袭人伸手从他项上摘下那通灵玉来,用自己的手帕包好,塞在褥下,次日带时便冰不着脖子。那宝玉就枕便睡着了。彼时李嬷嬷等已进来了,听见醉了,不敢前来再加触犯,只悄悄的打听睡了,方放心散去。

次日醒来,就有人回:“那边小蓉大爷带了秦相公来拜。”宝玉忙接了出去,领了拜见贾母。贾母见秦钟形容标致,举止温柔,堪陪宝玉读书,心中十分欢喜,便留茶留饭,又命人带去见王夫人等。众人因素爱秦氏,今见了秦钟是这般人品,也都欢喜,临去时都有表礼。贾母又与了一个荷包并一个金魁星,取"文星和合"之意。又嘱咐他道:“你家住的远,或有一时寒热饥饱不便,只管住在这里,不必限定了。只和你宝叔在一处,别跟着那些不长进的东西们学。”秦钟一一的答应,回去禀知。

他父亲秦业现任营缮郎,年近七十,夫人早亡。因当年无儿女,便向养生堂抱了一个儿子并一个女儿。谁知儿子又死了,只剩女儿,小名唤可儿,长大时,生的形容袅娜,性格风流。因素与贾家有些瓜葛,故结了亲,许与贾蓉为妻。那秦业至五旬之上方得了秦钟。因去岁业师亡故,未暇延请高明之士,只得暂时在家温习旧课。正思要和亲家去商议送往他家塾中,暂且不致荒废,可巧遇见了宝玉这个机会。又知贾家塾中现今司塾的是贾代儒,乃当今之老儒,秦钟此去,学业料必进益,成名可望,因此十分喜悦。只是宦囊羞涩,那贾家上上下下都是一双富贵眼睛,容易拿不出来,为儿子的终身大事,说不得东拼西凑的恭恭敬敬封了二十四两贽见礼,亲自带了秦钟,来代儒家拜见了。然后听宝玉上学之日,好一同入塾。正是:

早知日后闲争气,岂肯今朝错读书。

\chapter{恋风流情友入家塾\ttlbreak 起嫌疑顽童闹学堂}

话说秦业父子专候贾家的人来送上学择日之信。原来宝玉急于要和秦钟相遇,却顾不得别的,遂择了后日一定上学。”后日一早请秦相公到我这里,会齐了,一同前去。”-打发了人送了信。

至是日一早,宝玉起来时,袭人早已把书笔文物包好,收拾的停停妥妥,坐在床沿上发闷。见宝玉醒来,只得伏侍他梳洗。宝玉见他闷闷的,因笑问道:“好姐姐,你怎么又不自在了?难道怪我上学去丢的你们冷清了不成?"袭人笑道:“这是那里话。读书是极好的事,不然就潦倒一辈子,终久怎么样呢。但只一件:只是念书的时节想着书,不念的时节想着家些。别和他们一处顽闹,碰见老爷不是顽的。虽说是奋志要强,那工课宁可少些,一则贪多嚼不烂,二则身子也要保重。这就是我的意思,你可要体谅。”袭人说一句,宝玉应一句。袭人又道:“大毛衣服我也包好了,交出给小子们去了。学里冷,好歹想着添换,比不得家里有人照顾。脚炉手炉的炭也交出去了,你可着他们添。那一起懒贼,你不说,他们乐得不动,白冻坏了你。”宝玉道:“你放心,出外头我自己都会调停的。你们也别闷死在这屋里,长和林妹妹一处去顽笑着才好。”说着,俱已穿戴齐备,袭人催他去见贾母,贾政,王夫人等。宝玉又去嘱咐了晴雯麝月等几句,方出来见贾母。贾母也未免有几句嘱咐的话。然后去见王夫人,又出来书房中见贾政。偏生这日贾政回家早些,正在书房中与相公清客们闲谈。忽见宝玉进来请安,回说上学里去,贾政冷笑道:“你如果再提`上学-两个字,连我也羞死了。依我的话,你竟顽你的去是正理。仔细站脏了我这地,靠脏了我的门!"众清客相公们都早起身笑道:“老世翁何必又如此。今日世兄一去,三二年就可显身成名的了,断不似往年仍作小儿之态了。天也将饭时,世兄竟快请罢。”说着便有两个年老的携了宝玉出去。

贾政因问:“跟宝玉的是谁?"只听外面答应了两声,早进来三四个大汉,打千儿请安。贾政看时,认得是宝玉的奶母之子,名唤李贵。因向他道:“你们成日家跟他上学,他到底念了些什么书!倒念了些流言混语在肚子里,学了些精致的淘气。等我闲一闲,先揭了你的皮,再和那不长进的算帐!"吓的李贵忙双膝跪下,摘了帽子,碰头有声,连连答应"是",又回说:“哥儿已念到第三本《诗经》,什么`呦呦鹿鸣,荷叶浮萍-,小的不敢撒谎。”说的满座哄然大笑起来。贾政也撑不住笑了。因说道:“那怕再念三十本《诗经》,也都是掩耳偷铃,哄人而已。你去请学里太爷的安,就说我说了:什么《诗经》古文,一概不用虚应故事,只是先把《四书》一气讲明背熟,是最要紧的。”李贵忙答应"是",见贾政无话,方退出去。

此时宝玉独站在院外屏声静候,待他们出来,便忙忙的走了。李贵等一面掸衣服,一面说道:“哥儿听见了不曾?可先要揭我们的皮呢!人家的奴才跟主子赚些好体面,我们这等奴才白陪着挨打受骂的。从此后也可怜见些才好。”宝玉笑道:“好哥哥,你别委曲,我明儿请你。”李贵道:“小祖宗,谁敢望你请,只求听一句半句话就有了。”说着,又至贾母这边,秦钟早来候着了,贾母正和他说话儿呢。于是二人见过,辞了贾母。宝玉忽想起未辞黛玉,因又忙至黛玉房中来作辞。彼时黛玉才在窗下对镜理妆,听宝玉说上学去,因笑道:“好,这一去,可定是要`蟾宫折桂-去了。我不能送你了。”宝玉道:“好妹妹,等我下了学再吃饭。和胭脂膏子也等我来再制。”劳叨了半日,方撤身去了。黛玉忙又叫住问道:“你怎么不去辞辞你宝姐姐呢?"宝玉笑而不答,一径同秦钟上学去了。原来这贾家之义学,离此也不甚远,不过一里之遥,原系始祖所立,恐族中子弟有贫穷不能请师者,即入此中肄业。凡族中有官爵之人,皆供给银两,按俸之多寡帮助,为学中之费。特共举年高有德之人为塾掌,专为训课子弟。如今宝秦二人来了,一一的都互相拜见过,读起书来。自此以后,他二人同来同往,同坐同起,愈加亲密。又兼贾母爱惜,也时常的留下秦钟,住上三天五日,与自己的重孙一般疼爱。因见秦钟不甚宽裕,更又助他些衣履等物。不上一月之工,秦钟在荣府便熟了。宝玉终是不安本分之人,竟一味的随心所欲,因此又发了癖性,又特向秦钟悄说道:“咱们俩个人一样的年纪,况又是同窗,以后不必论叔侄,只论弟兄朋友就是了。”先是秦钟不肯,当不得宝玉不依,只叫他"兄弟",或叫他的表字"鲸卿",秦钟也只得混着乱叫起来。

原来这学中虽都是本族人丁与些亲戚的子弟,俗语说的好:“一龙生九种,种种各别。”未免人多了,就有龙蛇混杂,下流人物在内。自宝,秦二人来了,都生的花朵儿一般的模样,又见秦钟腼腆温柔,未语面先红,怯怯羞羞,有女儿之风,宝玉又是天生成惯能作小服低,赔身下气,情性体贴,话语绵缠,因此二人更加亲厚,也怨不得那起同窗人起了疑,背地里你言我语,诟谇谣诼,布满书房内外。原来薛蟠自来王夫人处住后,便知有一家学,学中广有青年子弟,不免偶动了龙阳之兴,因此也假来上学读书,不过是三日打鱼,两日晒网,白送些束ю礼物与贾代儒,却不曾有一些儿进益,只图结交些契弟。谁想这学内就有好几个小学生,图了薛蟠的银钱吃穿,被他哄上手的,也不消多记。更又有两个多情的小学生,亦不知是那一房的亲眷,亦未考真名姓,只因生得妩媚风流,满学中都送了他两个外号,一号"香怜",一号"玉爱"。虽都有窃慕之意,将不利于孺子之心,只是都惧薛蟠的威势,不敢来沾惹。如今宝,秦二人一来,见了他两个,也不免绻缱羡慕,亦因知系薛蟠相知,故未敢轻举妄动。香,玉二人心中,也一般的留情与宝,秦。因此四人心中虽有情意,只未发迹。每日一入学中,四处各坐,却八目勾留,或设言托意,或咏桑寓柳,遥以心照,却外面自为避人眼目。不意偏又有几个滑贼看出形景来,都背后挤眉弄眼,或咳嗽扬声,这也非止一日。可巧这日代儒有事,早已回家去了,只留下一句七言对联,命学生对了,明日再来上书,将学中之事,又命贾瑞暂且管理。妙在薛蟠如今不大来学中应卯了,因此秦钟趁此和香怜挤眉弄眼,递暗号儿,二人假装出小恭,走至后院说梯己话。秦钟先问他:“家里的大人可管你交朋友不管?"一语未了,只听背后咳嗽了一声。二人唬的忙回头看时,原来是窗友名金荣者。香怜有些性急,羞怒相激,问他道:“你咳嗽什么?难道不许我两个说话不成?"金荣笑道:“许你们说话,难道不许我咳嗽不成?我只问你们:有话不明说,许你们这样鬼鬼祟祟的干什么故事?我可也拿住了,还赖什么!先得让我怞个头儿,咱们一声儿不言语,不然大家就奋起来。”秦,香二人急的飞红的脸,便问道:“你拿住什么了?"金荣笑道:“我现拿住了是真的。”说着,又拍着手笑嚷道:“贴的好烧饼!你们都不买一个吃去?"秦钟香怜二人又气又急,忙进去向贾瑞前告金荣,说金荣无故欺负他两个。原来这贾瑞最是个图便宜没行止的人,每在学中以公报私,勒索子弟们请他,后又附助着薛蟠图些银钱酒肉,一任薛蟠横行霸道,他不但不去管约,反助纣为虐讨好儿。偏那薛蟠本是浮萍心性,今日爱东,明日爱西,近来又有了新朋友,把香,玉二人又丢开一边。就连金荣亦是当日的好朋友,自有了香,玉二人,便弃了金荣。近日连香,玉亦已见弃。故贾瑞也无了提携帮衬之人,不说薛蟠得新弃旧,只怨香,玉二人不在薛蟠前提携帮补他,因此贾瑞金荣等一干人,也正在醋妒他两个。今见秦,香二人来告金荣,贾瑞心中便更不自在起来,虽不好呵叱秦钟,却拿着香怜作法,反说他多事,着实抢白了几句。香怜反讨了没趣,连秦钟也讪讪的各归坐位去了。金荣越发得了意,摇头咂嘴的,口内还说许多闲话,玉爱偏又听了不忿,两个人隔座咕咕唧唧的角起口来。金荣只一口咬定说:“方才明明的撞见他两个在后院子里亲嘴摸屁股,一对一у,撅草根儿怞长短,谁长谁先干。”金荣只顾得意乱说,却不防还有别人。谁知早又触怒了一个。你道这个是谁?原来这一个名唤贾蔷,亦系宁府中之正派玄孙,父母早亡,从小儿跟着贾珍过活,如今长了十六岁,比贾蓉生的还风流俊俏。他弟兄二人最相亲厚,常相共处。宁府人多口杂,那些不得志的奴仆们,专能造言诽谤主人,因此不知又有什么小人诟谇谣诼之词。贾珍想亦风闻得些口声不大好,自己也要避些嫌疑,如今竟分与房舍,命贾蔷搬出宁府,自去立门户过活去了。这贾蔷外相既美,内性又聪明,虽然应名来上学,亦不过虚掩眼目而已。仍是斗鸡走狗,赏花玩柳。总恃上有贾珍溺爱,下有贾蓉匡助,因此族人谁敢来触逆于他。他既和贾蓉最好,今见有人欺负秦钟,如何肯依?如今自己要挺身出来报不平,心中却忖度一番,想道:“金荣贾瑞一干人,都是薛大叔的相知,向日我又与薛大叔相好,倘或我一出头,他们告诉了老薛,我们岂不伤和气?待要不管,如此谣言,说的大家没趣。如今何不用计制伏,又止息口声,又伤不了脸面。”想毕,也装作出小恭,走至外面,悄悄的把跟宝玉的书童名唤茗烟者唤到身边,如此这般,调拨他几句。

这茗烟乃是宝玉第一个得用的,且又年轻不谙世事,如今听贾蔷说金荣如此欺负秦钟,连他爷宝玉都干连在内,不给他个利害,下次越发狂纵难制了。这茗烟无故就要欺压人的,如今得了这个信,又有贾蔷助着,便一头进来找金荣,也不叫金相公了,只说"姓金的,你是什么东西!"贾蔷遂跺一跺靴子,故意整整衣服,看看日影儿说:“是时候了。”遂先向贾瑞说有事要早走一步。贾瑞不敢强他,只得随他去了。这里茗烟先一把揪住金荣,问道:“我们у屁股不у屁股,管你фх相干,横竖没у你爹去罢了!你是好小子,出来动一动你茗大爷!"唬的满屋中子弟都怔怔的痴望。贾瑞忙吆喝:“茗烟不得撒野!"金荣气黄了脸,说:“反了!奴才小子都敢如此,我只和你主子说。”便夺手要去抓打宝玉秦钟。尚未去时,从脑后飕的一声,早见一方砚瓦飞来,并不知系何人打来的,幸未打着,却又打在旁人的座上,这座上乃是贾兰贾菌。

这贾菌亦系荣国府近派的重孙,其母亦少寡,独守着贾菌。这贾菌与贾兰最好,所以二人同桌而坐。谁知贾菌年纪虽小,志气最大,极是淘气不怕人的。他在座上冷眼看见金荣的朋友暗助金荣,飞砚来打茗烟,偏没打着茗烟,便落在他桌上,正打在面前,将一个磁砚水壶打了个粉碎,溅了一书黑水。贾菌如何依得,便骂:“好囚攮的们,这不都动了手了么!"骂着,也便抓起砚砖来要打回去。贾兰是个省事的,忙按住砚,极口劝道:“好兄弟,不与咱们相干。”贾菌如何忍得住,便两手抱起书匣子来,照那边抡了去。终是身小力薄,却抡不到那里,刚到宝玉秦钟桌案上就落了下来。只听哗啷啷一声,砸在桌上,书本纸片等至于笔砚之物撒了一桌,又把宝玉的一碗茶也砸得碗碎茶流。贾菌便跳出来,要揪打那一个飞砚的。金荣此时随手抓了一根毛竹大板在手,地狭人多,那里经得舞动长板。茗烟早吃了一下,乱嚷:“你们还不来动手!"宝玉还有三个小厮:一名锄药,一名扫红,一名墨雨。这三个岂有不淘气的,一齐乱嚷:“小妇养的!动了兵器了!"墨雨遂掇起一根门闩,扫红锄药手中都是马鞭子,蜂拥而上。贾瑞急的拦一回这个,劝一回那个,谁听他的话,肆行大闹。众顽童也有趁势帮着打太平拳助乐的,也有胆小藏在一边的,也有直立在桌上拍着手儿乱笑,喝着声儿叫打的。登时间鼎沸起来。

外边李贵等几个大仆人听见里边作起反来,忙都进来一齐喝住。问是何原故,众声不一,这一个如此说,那一个又如彼说。李贵且喝骂了茗烟四个一顿,撵了出去。秦钟的头早撞在金荣的板上,打起一层油皮,宝玉正拿褂襟子替他柔呢,见喝住了众人,便命:“李贵,收书!拉马来,我去回太爷去!我们被人欺负了,不敢说别的,守礼来告诉瑞大爷,瑞大爷反倒派我们的不是,听着人家骂我们,还调唆他们打我们茗烟,连秦钟的头也打破。这还在这里念什么书!茗烟他也是为有人欺侮我的。不如散了罢。”李贵劝道:“哥儿不要性急。太爷既有事回家去了,这会子为这点子事去聒噪他老人家,倒显的咱们没理。依我的主意,那里的事那里了结好,何必去惊动他老人家。这都是瑞大爷的不是,太爷不在这里,你老人家就是这学里的头脑了,众人看着你行事。众人有了不是,该打的打,该罚的罚,如何等闹到这步田地还不管?"贾瑞道:“我吆喝着都不听。”李贵笑道:“不怕你老人家恼我,素日你老人家到底有些不正经,所以这些兄弟才不听。就闹到太爷跟前去,连你老人家也是脱不过的。还不快作主意撕罗开了罢。”宝玉道:“撕罗什么?我必是回去的!"秦钟哭道:“有金荣,我是不在这里念书的。”宝玉道:“这是为什么?难道有人家来的,咱们倒来不得?我必回明白众人,撵了金荣去。”又问李贵:“金荣是那一房的亲戚?"李贵想了一想道:“也不用问了。若问起那一房的亲戚,更伤了兄弟们的和气。”

茗烟在窗外道:“他是东胡同子里璜大奶奶的侄儿。那是什么硬正仗腰子的,也来唬我们。璜大奶奶是他姑娘。你那姑妈只会打旋磨子,给我们琏二奶奶跪着借当头。我眼里就看不起他那样的主子奶奶!"李贵忙断喝不止,说:“偏你这小狗у的知道,有这些蛆嚼!"宝玉冷笑道:“我只当是谁的亲戚,原来是璜嫂子的侄儿,我就去问问他来!"说着便要走。叫茗烟进来包书。茗烟包着书,又得意道:“爷也不用自己去见,等我到他家,就说老太太有说的话问他呢,雇上一辆车拉进去,当着老太太问他,岂不省事。”李贵忙喝道:“你要死!仔细回去我好不好先捶了你,然后再回老爷太太,就说宝玉全是你调唆的。我这里好容易劝哄好了一半了,你又来生个新法子。你闹了学堂,不说变法儿压息了才是,倒要往大里闹!"茗烟方不敢作声儿了。

此时贾瑞也怕闹大了,自己也不干净,只得委曲着来央告秦钟,又央告宝玉。先是他二人不肯。后来宝玉说:“不回去也罢了,只叫金荣赔不是便罢。”金荣先是不肯,后来禁不得贾瑞也来逼他去赔不是,李贵等只得好劝金荣说:“原是你起的端,你不这样,怎得了局?"金荣强不得,只得与秦钟作了揖。宝玉还不依,偏定要磕头。贾瑞只要暂息此事,又悄悄的劝金荣说:“俗语说的好:`杀人不过头点地。-你既惹出事来,少不得下点气儿,磕个头就完事了。”金荣无奈,只得进前来与秦钟磕头。且听下回分解。

\chapter{金寡妇贪利权受辱\ttlbreak 张太医论病细穷源}

话说金荣因人多势众,又兼贾瑞勒令,赔了不是,给秦钟磕了头,宝玉方才不吵闹了。大家散了学,金荣回到家中,越想越气,说:“秦钟不过是贾蓉的小舅子,又不是贾家的子孙,附学读书,也不过和我一样。他因仗着宝玉和他好,他就目中无人。他既是这样,就该行些正经事,人也没的说。他素日又和宝玉鬼鬼祟祟的,只当人都是瞎子,看不见。今日他又去勾搭人,偏偏的撞在我眼睛里。就是闹出事来,我还怕什么不成?”

他母亲胡氏听见他咕咕嘟嘟的说,因问道:“你又要争什么闲气?好容易我望你姑妈说了,你姑妈千方百计的才向他们西府里的琏二奶奶跟前说了,你才得了这个念书的地方。若不是仗着人家,咱们家里还有力量请的起先生?况且人家学里,茶也是现成的,饭也是现成的。你这二年在那里念书,家里也省好大的嚼用呢。省出来的,你又爱穿件鲜明衣服。再者,不是因你在那里念书,你就认得什么薛大爷了?那薛大爷一年不给不给,这二年也帮了咱们有七八十两银子。你如今要闹出了这个学房,再要找这么个地方,我告诉你说罢,比登天还难呢!你给我老老实实的顽一会子睡你的觉去,好多着呢。”于是金荣忍气吞声,不多一时他自去睡了。次日仍旧上学去了。不在话下。

且说他姑娘,原聘给的是贾家玉字辈的嫡派,名唤贾璜。但其族人那里皆能象宁荣二府的富势,原不用细说。这贾璜夫妻守着些小的产业,又时常到宁荣二府里去请请安,又会奉承凤姐儿并尤氏,所以凤姐儿尤氏也时常资助资助他,方能如此度日。今日正遇天气晴明,又值家中无事,遂带了一个婆子,坐上车,来家里走走,瞧瞧寡嫂并侄儿。

闲话之间,金荣的母亲偏提起昨日贾家学房里的那事,从头至尾,一五一十都向他小姑子说了。这璜大奶奶不听则已,听了,一时怒从心上起,说道:“这秦钟小崽子是贾门的亲戚,难道荣儿不是贾门的亲戚?人都别忒势利了,况且都作的是什么有脸的好事!就是宝玉,也犯不上向着他到这个样。等我去到东府瞧瞧我们珍大奶奶,再向秦钟他姐姐说说,叫他评评这个理。”这金荣的母亲听了这话,急的了不得,忙说道:“这都是我的嘴快,告诉了姑奶奶了,求姑奶奶别去,别管他们谁是谁非。倘或闹起来,怎么在那里站得住。若是站不住,家里不但不能请先生,反倒在他身上添出许多嚼用来呢。”璜大奶奶听了,说道:“那里管得许多,你等我说了,看是怎么样!"也不容他嫂子劝,一面叫老婆子瞧了车,就坐上往宁府里来。

到了宁府,进了车门,到了东边小角门前下了车,进去见了贾珍之妻尤氏。也未敢气高,殷殷勤勤叙过寒温,说了些闲话,方问道:“今日怎么没见蓉大奶奶?"尤氏说道:“他这些日子不知怎么着,经期有两个多月没来。叫大夫瞧了,又说并不是喜。那两日,到了下半天就懒待动,话也懒待说,眼神也发眩。我说他:`你且不必拘礼,早晚不必照例上来,你就好生养养罢。就是有亲戚一家儿来,有我呢。就有长辈们怪你,等我替你告诉。-连蓉哥我都嘱咐了,我说:`你不许累ц他,不许招他生气,叫他静静的养养就好了。他要想什么吃,只管到我这里取来。倘或我这里没有,只管望你琏二婶子那里要去。倘或他有个好和歹,你再要娶这么一个媳妇,这么个模样儿,这么个性情的人儿,打着灯笼也没地方找去。-他这为人行事,那个亲戚,那个一家的长辈不喜欢他?所以我这两日好不烦心,焦的我了不得。偏偏今日早晨他兄弟来瞧他,谁知那小孩子家不知好歹,看见他姐姐身上不大爽快,就有事也不当告诉他,别说是这么一点子小事,就是你受了一万分的委曲,也不该向他说才是。谁知他们昨儿学房里打架,不知是那里附学来的一个人欺侮了他了。里头还有些不干不净的话,都告诉了他姐姐。婶子,你是知道那媳妇的:虽则见了人有说有笑,会行事儿,他可心细,心又重,不拘听见个什么话儿,都要度量个三日五夜才罢。这病就是打这个秉性上头思虑出来的。今儿听见有人欺负了他兄弟,又是恼,又是气。恼的是那群混帐狐朋狗友的扯是搬非,调三惑四的那些人,气的是他兄弟不学好,不上心念书,以致如此学里吵闹。他听了这事,今日索性连早饭也没吃。我听见了,我方到他那边安慰了他一会子,又劝解了他兄弟一会子。我叫他兄弟到那边府里找宝玉去了,我才看着他吃了半盏燕窝汤,我才过来了。婶子,你说我心焦不心焦?况且如今又没个好大夫,我想到他这病上,我心里倒象针扎似的。你们知道有什么好大夫没有?”

金氏听了这半日话,把方才在他嫂子家的那一团要向秦氏理论的盛气,早吓的都丢在爪洼国去了。听见尤氏问他有知道好大夫的话,连忙答道:“我们这么听着,实在也没见人说有个好大夫。如今听起大奶奶这个来,定不得还是喜呢。嫂子倒别教人混治。倘或认错了,这可是了不得的。”尤氏道:“可不是呢。”正是说话间,贾珍从外进来,见了金氏,便向尤氏问道:“这不是璜大奶奶么?"金氏向前给贾珍请了安。贾珍向尤氏说道:“让这大妹妹吃了饭去。”贾珍说着话,就过那屋里去了。金氏此来,原要向秦氏说说秦钟欺负了他侄儿的事,听见秦氏有病,不但不能说,亦且不敢提了。况且贾珍尤氏又待的很好,反转怒为喜,又说了一会子话儿,方家去了。

金氏去后,贾珍方过来坐下,问尤氏道:“今日他来,有什么说的事情么?"尤氏答道:“倒没说什么。一进来的时候,脸上倒象有些着了恼的气色似的,及说了半天话,又提起媳妇这病,他倒渐渐的气色平定了。你又叫让他吃饭,他听见媳妇这么病,也不好意思只管坐着,又说了几句闲话儿就去了,倒没求什么事。如今且说媳妇这病,你到那里寻一个好大夫来与他瞧瞧要紧,可别耽误了。现今咱们家走的这群大夫,那里要得,一个个都是听着人的口气儿,人怎么说,他也添几句文话儿说一遍。可倒殷勤的很,三四个人一日轮流着倒有四五遍来看脉。他们大家商量着立个方子,吃了也不见效,倒弄得一日换四五遍衣裳,坐起来见大夫,其实于病人无益。”贾珍说道:“可是。这孩子也糊涂,何必脱脱换换的,倘再着了凉,更添一层病,那还了得。衣裳任凭是什么好的,可又值什么,孩子的身子要紧,就是一天穿一套新的,也不值什么。我正进来要告诉你:方才冯紫英来看我,他见我有些抑郁之色,问我是怎么了。我才告诉他说,媳妇忽然身子有好大的不爽快,因为不得个好太医,断不透是喜是病,又不知有妨碍无妨碍,所以我这两日心里着实着急。冯紫英因说起他有一个幼时从学的先生,姓张名友士,学问最渊博的,更兼医理极深,且能断人的生死。今年是上京给他儿子来捐官,现在他家住着呢。这么看来,竟是合该媳妇的病在他手里除灾亦未可知。我即刻差人拿我的名帖请去了。今日倘或天晚了不能来,明日想必一定来。况且冯紫英又即刻回家亲自去求他,务必叫他来瞧瞧。等这个张先生来瞧了再说罢。”

尤氏听了,心中甚喜,因说道:“后日是太爷的寿日,到底怎么办?"贾珍说道:“我方才到了太爷那里去请安,兼请太爷来家来受一受一家子的礼。太爷因说道:`我是清净惯了的,我不愿意往你们那是非场中去闹去。你们必定说是我的生日,要叫我去受众人些头,莫过你把我从前注的《陰骘文》给我令人好好的写出来刻了,比叫我无故受众人的头还强百倍呢。倘或后日这两日一家子要来,你就在家里好好的款待他们就是了。也不必给我送什么东西来,连你后日也不必来,你要心中不安,你今日就给我磕了头去。倘或后日你要来,又跟随多少人来闹我,我必和你不依。-如此说了又说,后日我是再不敢去的了。且叫来升来,吩咐他预备两日的筵席。”尤氏因叫人叫了贾蓉来:“吩咐来升照旧例预备两日的筵席,要丰丰富富的。你再亲自到西府里去请老太太,大太太,二太太和你琏二婶子来逛逛。你父亲今日又听见一个好大夫,业已打发人请去了,想必明日必来。你可将他这些日子的病症细细的告诉他。”

贾蓉一一的答应着出去了。正遇着方才去冯紫英家请那先生的小子回来了,因回道:“奴才方才到了冯大爷家,拿了老爷的名帖请那先生去。那先生说道:`方才这里大爷也向我说了。但是今日拜了一天的客,才回到家,此时精神实在不能支持,就是去到府上也不能看脉。-他说等调息一夜,明日务必到府。他又说,他`医学浅薄,本不敢当此重荐,因我们冯大爷和府上的大人既已如此说了,又不得不去,你先替我回明大人就是了。大人的名帖实不敢当。-仍叫奴才拿回来了。哥儿替奴才回一声儿罢。”贾蓉转身复进去,回了贾珍尤氏的话,方出来叫了来升来,吩咐他预备两日的筵席的话。来升听毕,自去照例料理。不在话下。

且说次日午间,人回道:“请的那张先生来了。”贾珍遂延入大厅坐下。茶毕,方开言道:“昨承冯大爷示知老先生人品学问,又兼深通医学,小弟不胜钦仰之至。”张先生道:“晚生粗鄙下士,本知见浅陋,昨因冯大爷示知,大人家第谦恭下士,又承呼唤,敢不奉命。但毫无实学,倍增颜汗。”贾珍道:“先生何必过谦。就请先生进去看看儿妇,仰仗高明,以释下怀。”于是,贾蓉同了进去。到了贾蓉居室,见了秦氏,向贾蓉说道:“这就是尊夫人了?"贾蓉道:“正是。请先生坐下,让我把贱内的病说一说再看脉如何?"那先生道:“依小弟的意思,竟先看过脉再说的为是。我是初造尊府的,本也不晓得什么,但是我们冯大爷务必叫小弟过来看看,小弟所以不得不来。如今看了脉息,看小弟说的是不是,再将这些日子的病势讲一讲,大家斟酌一个方儿,可用不可用,那时大爷再定夺。”贾蓉道:“先生实在高明,如今恨相见之晚。就请先生看一看脉息,可治不可治,以便使家父母放心。”于是家下媳妇们捧过大迎枕来,一面给秦氏拉着袖口,露出脉来。先生方伸手按在右手脉上,调息了至数,宁神细诊了有半刻的工夫,方换过左手,亦复如是。诊毕脉息,说道:“我们外边坐罢。”

贾蓉于是同先生到外间房里床上坐下,一个婆子端了茶来。贾蓉道:“先生请茶。”于是陪先生吃了茶,遂问道:“先生看这脉息,还治得治不得?"先生道:“看得尊夫人这脉息:左寸沉数,左关沉伏,右寸细而无力,右关需而无神。其左寸沉数者,乃心气虚而生火,左关沉伏者,乃肝家气滞血亏。右寸细而无力者,乃肺经气分太虚,右关需而无神者,乃脾土被肝木克制。心气虚而生火者,应现经期不调,夜间不寐。肝家血亏气滞者,必然肋下疼胀,月信过期,心中发热。肺经气分太虚者,头目不时眩晕,寅卯间必然自汗,如坐舟中。脾土被肝木克制者,必然不思饮食,精神倦怠,四肢酸软。据我看这脉息,应当有这些症候才对。或以这个脉为喜脉,则小弟不敢从其教也。”旁边一个贴身伏侍的婆子道:“何尝不是这样呢。真正先生说的如神,倒不用我们告诉了。如今我们家里现有好几位太医老爷瞧着呢,都不能的当真切的这么说。有一位说是喜,有一位说是病,这位说不相干,那位说怕冬至,总没有个准话儿。求老爷明白指示指示。”

那先生笑道:“大奶奶这个症候,可是那众位耽搁了。要在初次行经的日期就用药治起来,不但断无今日之患,而且此时已全愈了。如今既是把病耽误到这个地位,也是应有此灾。依我看来,这病尚有三分治得。吃了我的药看,若是夜里睡的着觉,那时又添了二分拿手了。据我看这脉息:大奶奶是个心性高强聪明不过的人,聪明忒过,则不如意事常有,不如意事常有,则思虑太过。此病是忧虑伤脾,肝木忒旺,经血所以不能按时而至。大奶奶从前的行经的日子问一问,断不是常缩,必是常长的。是不是?"这婆子答道:“可不是,从没有缩过,或是长两日三日,以至十日都长过。”先生听了道:“妙啊!这就是病源了。从前若能够以养心调经之药服之,何至于此。这如今明显出一个水亏木旺的症候来。待用药看看。”于是写了方子,递与贾蓉,上写的是:

益气养荣补脾和肝汤

人参二钱白术二钱土炒云苓三钱熟地四钱

归身二钱酒洗白芍二钱炒川芎钱半黄芪三钱

香附米二钱制醋柴胡八分怀山药二钱炒真阿胶二钱蛤粉炒

延胡索钱半酒炒炙甘草八分

引用建莲子七粒去心红枣二枚贾蓉看了,说:“高明的很。还要请教先生,这病与性命终久有妨无妨?"先生笑道:“大爷是最高明的人。人病到这个地位,非一朝一夕的症候,吃了这药也要看医缘了。依小弟看来,今年一冬是不相干的。总是过了春分,就可望全愈了。”贾蓉也是个聪明人,也不往下细问了。于是贾蓉送了先生去了,方将这药方子并脉案都给贾珍看了,说的话也都回了贾珍并尤氏了。尤氏向贾珍说道:“从来大夫不象他说的这么痛快,想必用的药也不错。”贾珍道:“人家原不是混饭吃久惯行医的人。因为冯紫英我们好,他好容易求了他来了。既有这个人,媳妇的病或者就能好了。他那方子上有人参,就用前日买的那一斤好的罢。”贾蓉听毕话,方出来叫人打药去煎给秦氏吃。不知秦氏服了此药病势如何,下回分解。

\chapter{庆寿辰宁府排家宴\ttlbreak 见熙凤贾瑞起淫心}

话说是日贾敬的寿辰,贾珍先将上等可吃的东西,稀奇些的果品,装了十六大捧盒,着贾蓉带领家下人等与贾敬送去,向贾蓉说道:“你留神看太爷喜欢不喜欢,你就行了礼来。你说:`我父亲遵太爷的话未敢来,在家里率领合家都朝上行了礼了。-"贾蓉听罢,即率领家人去了。

这里渐渐的就有人来了。先是贾琏,贾蔷到来,先看了各处的座位,并问:“有什么顽意儿没有?"家人答道:“我们爷原算计请太爷今日来家来,所以未敢预备顽意儿。前日听见太爷又不来了,现叫奴才们找了一班小戏儿并一档子打十番的,都在园子里戏台上预备着呢。”

次后邢夫人,王夫人,凤姐儿,宝玉都来了,贾珍并尤氏接了进去。尤氏的母亲已先在这里呢。大家见过了,彼此让了坐。贾珍尤氏二人亲自递了茶,因说道:“老太太原是老祖宗,我父亲又是侄儿,这样日子,原不敢请他老人家,但是这个时候,天气正凉爽,满园的菊花又盛开,请老祖宗过来散散闷,看着众儿孙热闹热闹,是这个意思。谁知老祖宗又不肯赏脸。”凤姐儿未等王夫人开口,先说道:“老太太昨日还说要来着呢,因为晚上看着宝兄弟他们吃桃儿,老人家又嘴馋,吃了有大半个,五更天的时候就一连起来了两次,今日早晨略觉身子倦些。因叫我回大爷,今日断不能来了,说有好吃的要几样,还要很烂的。”贾珍听了笑道:“我说老祖宗是爱热闹的,今日不来,必定有个原故,若是这么着就是了。”

王夫人道:“前日听见你大妹妹说,蓉哥儿媳妇儿身上有些不大好,到底是怎么样?"尤氏道:“他这个病得的也奇。上月中秋还跟着老太太,太太们顽了半夜,回家来好好的。到了二十后,一日比一日觉懒,也懒待吃东西,这将近有半个多月了。经期又有两个月没来。”邢夫人接着说道:“别是喜罢?"正说着,外头人回道:“大老爷,二老爷并一家子的爷们都来了,在厅上呢。”贾珍连忙出去了。这里尤氏方说道:“从前大夫也有说是喜的。昨日冯紫英荐了他从学过的一个先生,医道很好,瞧了说不是喜,竟是很大的一个症候。昨日开了方子,吃了一剂药,今日头眩的略好些,别的仍不见怎么样大见效。”凤姐儿道:“我说他不是十分支持不住,今日这样的日子,再也不肯不扎挣着上来。”尤氏道:“你是初三日在这里见他的,他强扎挣了半天,也是因你们娘儿两个好的上头,他才恋恋的舍不得去。”凤姐儿听了,眼圈儿红了半天,半日方说道:“真是`天有不测风云,人有旦夕祸福-。这个年纪,倘或就因这个病上怎么样了,人还活着有甚么趣儿!"正说话间,贾蓉进来,给邢夫人,王夫人,凤姐儿前都请了安,方回尤氏道:“方才我去给太爷送吃食去,并回说我父亲在家中伺候老爷们,款待一家子的爷们,遵太爷的话未敢来。太爷听了甚喜欢,说:`这才是-。叫告诉父亲母亲好生伺候太爷太太们,叫我好生伺候叔叔婶子们并哥哥们。还说那《陰骘文》,叫急急的刻出来,印一万张散人。我将此话都回了我父亲了。我这会子得快出去打发太爷们并合家爷们吃饭。”凤姐儿说:“蓉哥儿,你且站住。你媳妇今日到底是怎么着?"贾蓉皱皱眉说道:“不好么!婶子回来瞧瞧去就知道了。”于是贾蓉出去了。

这里尤氏向邢夫人,王夫人道:“太太们在这里吃饭阿,还是在园子里吃去好?小戏儿现预备在园子里呢。”王夫人向邢夫人道:“我们索性吃了饭再过去罢,也省好些事。”邢夫人道:“很好。”于是尤氏就吩咐媳妇婆子们:“快送饭来。”门外一齐答应了一声,都各人端各人的去了。不多一时,摆上了饭。尤氏让邢夫人,王夫人并他母亲都上了坐,他与凤姐儿,宝玉侧席坐了。邢夫人,王夫人道:“我们来原为给大老爷拜寿,这不竟是我们来过生日来了么?"凤姐儿说道:“大老爷原是好养静的,已经修炼成了,也算得是神仙了。太太们这么一说,这就叫作`心到神知-了。”一句话说的满屋里的人都笑起来了。

于是,尤氏的母亲并邢夫人,王夫人,凤姐儿都吃毕饭,漱了口,净了手,才说要往园子里去,贾蓉进来向尤氏说道:“老爷们并众位叔叔哥哥兄弟们也都吃了饭了。大老爷说家里有事,二老爷是不爱听戏又怕人闹的慌,都才去了。别的一家子爷们都被琏二叔并蔷兄弟让过去听戏去了。方才南安郡王,东平郡王,西宁郡王,北静郡王四家王爷,并镇国公牛府等六家,忠靖侯史府等八家,都差人持了名帖送寿礼来,俱回了我父亲,先收在帐房里了,礼单都上上档子了。老爷的领谢的名帖都交给各来人了,各来人也都照旧例赏了,众来人都让吃了饭才去了。母亲该请二位太太,老娘,婶子都过园子里坐着去罢。”尤氏道:“也是才吃完了饭,就要过去了。”

凤姐儿说:“我回太太,我先瞧瞧蓉哥儿媳妇,我再过去。”王夫人道:“很是,我们都要去瞧瞧他,倒怕他嫌闹的慌,说我们问他好罢。”尤氏道:“好妹妹,媳妇听你的话,你去开导开导他,我也放心。你就快些过园子里来。”宝玉也要跟了凤姐儿去瞧秦氏去,王夫人道:“你看看就过去罢,那是侄儿媳妇。”于是尤氏请了邢夫人,王夫人并他母亲都过会芳园去了。

凤姐儿,宝玉方和贾蓉到秦氏这边来了。进了房门,悄悄的走到里间房门口,秦氏见了,就要站起来,凤姐儿说:“快别起来,看起猛了头晕。”于是凤姐儿就紧走了两步,拉住秦氏的手,说道:“我的奶奶!怎么几日不见,就瘦的这么着了!"于是就坐在秦氏坐的褥子上。宝玉也问了好,坐在对面椅子上。贾蓉叫:“快倒茶来,婶子和二叔在上房还未喝茶呢。”

秦氏拉着凤姐儿的手,强笑道:“这都是我没福。这样人家,公公婆婆当自己的女孩儿似的待。婶娘的侄儿虽说年轻,却也是他敬我,我敬他,从来没有红过脸儿。就是一家子的长辈同辈之中,除了婶子倒不用说了,别人也从无不疼我的,也无不和我好的。这如今得了这个病,把我那要强的心一分也没了。公婆跟前未得孝顺一天,就是婶娘这样疼我,我就有十分孝顺的心,如今也不能够了。我自想着,未必熬的过年去呢。”

宝玉正眼瞅着那《海棠春睡图》并那秦太虚写的"嫩寒锁梦因春冷,芳气笼人是酒香"的对联,不觉想起在这里睡晌觉梦到"太虚幻境"的事来。正自出神,听得秦氏说了这些话,如万箭攒心,那眼泪不知不觉就流下来了。凤姐儿心中虽十分难过,但恐怕病人见了众人这个样儿反添心酸,倒不是来开导劝解的意思了。见宝玉这个样子,因说道:“宝兄弟,你忒婆婆妈妈的了。他病人不过是这么说,那里就到得这个田地了?况且能多大年纪的人,略病一病儿就这么想那么想的,这不是自己倒给自己添病了么?"贾蓉道:“他这病也不用别的,只是吃得些饮食就不怕了。”凤姐儿道:“宝兄弟,太太叫你快过去呢。你别在这里只管这么着,倒招的媳妇也心里不好。太太那里又惦着你。”因向贾蓉说道:“你先同你宝叔叔过去罢,我还略坐一坐儿。”贾蓉听说,即同宝玉过会芳园来了。

这里凤姐儿又劝解了秦氏一番,又低低的说了许多衷肠话儿,尤氏打发人请了两三遍,凤姐儿才向秦氏说道:“你好生养着罢,我再来看你。合该你这病要好,所以前日就有人荐了这个好大夫来,再也是不怕的了。”秦氏笑道:“任凭神仙也罢,治得病治不得命。婶子,我知道我这病不过是挨日子。”凤姐儿说道:“你只管这么想着,病那里能好呢?总要想开了才是。况且听得大夫说,若是不治,怕的是春天不好呢。如今才九月半,还有四五个月的工夫,什么病治不好呢?咱们若是不能吃人参的人家,这也难说了,你公公婆婆听见治得好你,别说一日二钱人参,就是二斤也能够吃的起。好生养着罢,我过园子里去了。”秦氏又道:“婶子,恕我不能跟过去了。闲了时候还求婶子常过来瞧瞧我,咱们娘儿们坐坐,多说几遭话儿。”凤姐儿听了,不觉得又眼圈儿一红,遂说道:“我得了闲儿必常来看你。”于是凤姐儿带领跟来的婆子丫头并宁府的媳妇婆子们,从里头绕进园子的便门来。但只见:

黄花满地,白柳横坡。小桥通若耶之溪,曲径接天台之

路。石中清流激湍,篱落飘香,树头红叶翩翻,疏林如画。

西风乍紧,初罢莺啼,暖日当暄,又添蛩语。遥望东南,

建几处依山之榭,纵观西北,结三间临水之轩。笙簧盈

耳。别有幽情,罗绮穿林,倍添韵致。凤姐儿正自看园中的景致,一步步行来赞赏。猛然从假山石后走过一个人来,向前对凤姐儿说道:“请嫂子安。”凤姐儿猛然见了,将身子望后一退,说道:“这是瑞大爷不是?"贾瑞说道:“嫂子连我也不认得了?不是我是谁!"凤姐儿道:“不是不认得,猛然一见,不想到是大爷到这里来。”贾瑞道:“也是合该我与嫂子有缘。我方才偷出了席,在这个清净地方略散一散,不想就遇见嫂子也从这里来。这不是有缘么?"一面说着,一面拿眼睛不住的觑着凤姐儿。

凤姐儿是个聪明人,见他这个光景,如何不猜透八九分呢,因向贾瑞假意含笑道:“怨不得你哥哥时常提你,说你很好。今日见了,听你说这几句话儿,就知道你是个聪明和气的人了。这会子我要到太太们那里去,不得和你说话儿,等闲了咱们再说话儿罢。”贾瑞道:“我要到嫂子家里去请安,又恐怕嫂子年轻,不肯轻易见人。”凤姐儿假意笑道:“一家子骨肉,说什么年轻不年轻的话。”贾瑞听了这话,再不想到今日得这个奇遇,那神情光景亦发不堪难看了。凤姐儿说道:“你快入席去罢,仔细他们拿住罚你酒。”贾瑞听了,身上已木了半边,慢慢的一面走着,一面回过头来看。凤姐儿故意的把脚步放迟了些儿,见他去远了,心里暗忖道:“这才是知人知面不知心呢,那里有这样禽兽的人呢。他如果如此,几时叫他死在我的手里,他才知道我的手段!"于是凤姐儿方移步前来。将转过了一重山坡,见两三个婆子慌慌张张的走来,见了凤姐儿,笑说道:“我们奶奶见二奶奶只是不来,急的了不得,叫奴才们又来请奶奶来了。”凤姐儿说道:“你们奶奶就是这么急脚鬼似的。”凤姐儿慢慢的走着,问:“戏唱了几出了?"那婆子回道:“有八九出了。”说话之间,已来到了天香楼的后门,见宝玉和一群丫头们在那里玩呢。凤姐儿说道:“宝兄弟,别忒淘气了。”有一个丫头说道:“太太们都在楼上坐着呢,请奶奶就从这边上去罢。”

凤姐儿听了,款步提衣上了楼,见尤氏已在楼梯口等着呢。尤氏笑说道:“你们娘儿两个忒好了,见了面总舍不得来了。你明日搬来和他住着罢。你坐下,我先敬你一钟。”于是凤姐儿在邢王二夫人前告了坐,又在尤氏的母亲前周旋了一遍,仍同尤氏坐在一桌上吃酒听戏。尤氏叫拿戏单来,让凤姐儿点戏,凤姐儿说道:“亲家太太和太太们在这里,我如何敢点。”邢夫人王夫人说道:“我们和亲家太太都点了好几出了,你点两出好的我们听。”凤姐儿立起身来答应了一声,方接过戏单,从头一看,点了一出《还魂》,一出《弹词》,递过戏单去说:“现在唱的这《双官诰》,唱完了,再唱这两出,也就是时候了。”王夫人道:“可不是呢,也该趁早叫你哥哥嫂子歇歇,他们又心里不静。”尤氏说道:“太太们又不常过来,娘儿们多坐一会子去,才有趣儿,天还早呢。”凤姐儿立起身来望楼下一看,说:“爷们都往那里去了?"旁边一个婆子道:“爷们才到凝曦轩,带了打十番的那里吃酒去了。”凤姐儿说道:“在这里不便宜,背地里又不知干什么去了!"尤氏笑道:“那里都象你这么正经人呢。”于是说说笑笑,点的戏都唱完了,方才撤下酒席,摆上饭来。吃毕,大家才出园子来,到上房坐下,吃了茶,方才叫预备车,向尤氏的母亲告了辞。尤氏率同众姬妾并家下婆子媳妇们方送出来,贾珍率领众子侄都在车旁侍立,等候着呢,见了邢夫人,王夫人道:“二位婶子明日还过来逛逛。”王夫人道:“罢了,我们今日整坐了一日,也乏了,明日歇歇罢。”于是都上车去了。贾瑞犹不时拿眼睛觑着凤姐儿。贾珍等进去后,李贵才拉过马来,宝玉骑上,随了王夫人去了。这里贾珍同一家子的弟兄子侄吃过了晚饭,方大家散了。

次日,仍是众族人等闹了一日,不必细说。此后凤姐儿不时亲自来看秦氏。秦氏也有几日好些,也有几日仍是那样。贾珍,尤氏,贾蓉好不焦心。

且说贾瑞到荣府来了几次,偏都遇见凤姐儿往宁府那边去了。这年正是十一月三十日冬至。到交节的那几日,贾母,王夫人,凤姐儿日日差人去看秦氏,回来的人都说:“这几日也没见添病,也不见甚好。”王夫人向贾母说:“这个症候,遇着这样大节不添病,就有好大的指望了。”贾母说:“可是呢,好个孩子,要是有些原故,可不叫人疼死。”说着,一阵心酸,叫凤姐儿说道:“你们娘儿两个也好了一场,明日大初一,过了明日,你后日再去看一看他去。你细细的瞧瞧他那光景,倘或好些儿,你回来告诉我,我也喜欢喜欢。那孩子素日爱吃的,你也常叫人做些给他送过去。”凤姐儿一一的答应了。

到了初二日,吃了早饭,来到宁府,看见秦氏的光景,虽未甚添病,但是那脸上身上的肉全瘦干了。于是和秦氏坐了半日,说了些闲话儿,又将这病无妨的话开导了一遍。秦氏说道:“好不好,春天就知道了。如今现过了冬至,又没怎么样,或者好的了也未可知。婶子回老太太,太太放心罢。昨日老太太赏的那枣泥馅的山药糕,我倒吃了两块,倒象克化的动似的。”凤姐儿说道:“明日再给你送来。我到你婆婆那里瞧瞧,就要赶着回去回老太太的话去。”秦氏道:“婶子替我请老太太,太太安罢。”

凤姐儿答应着就出来了,到了尤氏上房坐下。尤氏道:“你冷眼瞧媳妇是怎么样?"凤姐儿低了半日头,说道:“这实在没法儿了。你也该将一应的后事用的东西给他料理料理,冲一冲也好。”尤氏道:“我也叫人暗暗的预备了。就是那件东西不得好木头,暂且慢慢的办罢。”于是凤姐儿吃了茶,说了一会子话儿,说道:“我要快回去回老太太的话去呢。”尤氏道:“你可缓缓的说,别吓着老太太。”凤姐儿道:“我知道。”于是凤姐儿就回来了。到了家中,见了贾母,说:“蓉哥儿媳妇请老太太安,给老太太磕头,说他好些了,求老祖宗放心罢。他再略好些,还要给老祖宗磕头请安来呢。”贾母道:“你看他是怎么样?"凤姐儿说:“暂且无妨,精神还好呢。”贾母听了,沉吟了半日,因向凤姐儿说:“你换换衣服歇歇去罢。”

凤姐儿答应着出来,见过了王夫人,到了家中,平儿将烘的家常的衣服给凤姐儿换了。凤姐儿方坐下,问道:“家里没有什么事么?"平儿方端了茶来,递了过去,说道:“没有什么事。就是那三百银子的利银,旺儿媳妇送进来,我收了。再有瑞大爷使人来打听奶奶在家没有,他要来请安说话。”凤姐儿听了,哼了一声,说道:“这畜生合该作死,看他来了怎么样!"平儿因问道:“这瑞大爷是因什么只管来?"凤姐儿遂将九月里宁府园子里遇见他的光景,他说的话,都告诉了平儿。平儿说道:“癞蛤蟆想天鹅肉吃,没人轮的混帐东西,起这个念头,叫他不得好死!"凤姐儿道:“等他来了,我自有道理。”不知贾瑞来时作何光景,且听下回分解。

\chapter{王熙凤毒设相思局\ttlbreak 贾天祥正照风月鉴}

话说凤姐正与平儿说话,只见有人回说:“瑞大爷来了。”凤姐急命"快请进来。”贾瑞见往里让,心中喜出望外,急忙进来,见了凤姐,满面陪笑,连连问好。凤姐儿也假意殷勤,让茶让坐。

贾瑞见凤姐如此打扮,亦发酥倒,因饧了眼问道:“二哥哥怎么还不回来?"凤姐道:“不知什么原故。”贾瑞笑道:“别是路上有人绊住了脚了,舍不得回来也未可知?"凤姐道:“也未可知。男人家见一个爱一个也是有的。”贾瑞笑道:“嫂子这话说错了,我就不这样。”凤姐笑道:“象你这样的人能有几个呢,十个里也挑不出一个来。”贾瑞听了喜的抓耳挠腮,又道:“嫂子天天也闷的很。”凤姐道:“正是呢,只盼个人来说话解解闷儿。”贾瑞笑道:“我倒天天闲着,天天过来替嫂子解解闲闷可好不好?"凤姐笑道:“你哄我呢,你那里肯往我这里来。”贾瑞道:“我在嫂子跟前,若有一点谎话,天打雷劈!只因素日闻得人说,嫂子是个利害人,在你跟前一点也错不得,所以唬住了我。如今见嫂子最是个有说有笑极疼人的,我怎么不来,-死了也愿意!"凤姐笑道:“果然你是个明白人,比贾蓉两个强远了。我看他那样清秀,只当他们心里明白,谁知竟是两个胡涂虫,一点不知人心。”

贾瑞听了这话,越发撞在心坎儿上,由不得又往前凑了一凑,觑着眼看凤姐带的荷包,然后又问带着什么戒指。凤姐悄悄道:“放尊重着,别叫丫头们看了笑话。”贾瑞如听纶音佛语一般,忙往后退。凤姐笑道:“你该走了。”贾瑞说:“我再坐一坐儿。”-好狠心的嫂子。”凤姐又悄悄的道:“大天白日,人来人往,你就在这里也不方便。你且去,等着晚上起了更你来,悄悄的在西边穿堂儿等我。”贾瑞听了,如得珍宝,忙问道:“你别哄我。但只那里人过的多,怎么好躲的?"凤姐道:“你只放心。我把上夜的小厮们都放了假,两边门一关,再没别人了。”贾瑞听了,喜之不尽,忙忙的告辞而去,心内以为得手。

盼到晚上,果然黑地里摸入荣府,趁掩门时,钻入穿堂。果见漆黑无一人,往贾母那边去的门户已倒锁,只有向东的门未关。贾瑞侧耳听着,半日不见人来,忽听咯噔一声,东边的门也倒关了。贾瑞急的也不敢则声,只得悄悄的出来,将门撼了撼,关的铁桶一般。此时要求出去亦不能够,南北皆是大房墙,要跳亦无攀援。这屋内又是过门风,空落落,现是腊月天气,夜又长,朔风凛凛,侵肌裂骨,一夜几乎不曾冻死。好容易盼到早晨,只见一个老婆子先将东门开了,进去叫西门。贾瑞瞅他背着脸,一溜烟抱着肩跑了出来,幸而天气尚早,人都未起,从后门一径跑回家去。原来贾瑞父母早亡,只有他祖父代儒教养。那代儒素日教训最严,不许贾瑞多走一步,生怕他在外吃酒赌钱,有误学业。今忽见他一夜不归,只料定他在外非饮即赌,嫖娼宿妓,那里想到这段公案,因此气了一夜。贾瑞也捻着一把汗,少不得回来撒谎,只说:“往舅舅家去了,天黑了,留我住了一夜。”代儒道:“自来出门,非禀我不敢擅出,如何昨日私自去了?据此亦该打,何况是撒谎。”因此,发狠到底打了三四十扳,不许吃饭,令他跪在院内读文章,定要补出十天的工课来方罢。贾瑞直冻了一夜,今又遭了苦打,且饿着肚子,跪着在风地里读文章,其苦万状。

此时贾瑞前心犹是未改,再想不到是凤姐捉弄他。过后两日,得了空,便仍来找凤姐。凤姐故意抱怨他失信,贾瑞急的赌身发誓。凤姐因见他自投罗网,少不得再寻别计令他知改,故又约他道:“今日晚上,你别在那里了。你在我这房后小过道子里那间空屋里等我,可别冒撞了。”贾瑞道:“果真?"凤姐道:“谁可哄你,你不信就别来。”贾瑞道:“来,来,来。死也要来!"凤姐道:“这会子你先去罢。”贾瑞料定晚间必妥,此时先去了。凤姐在这里便点兵派将,设下圈套。

那贾瑞只盼不到晚上,偏生家里亲戚又来了,直等吃了晚饭才去,那天已有掌灯时候。又等他祖父安歇了,方溜进荣府,直往那夹道中屋子里来等着,热锅上的蚂蚁一般,只是干转。左等不见人影,右听也没声响,心下自思:“别是又不来了,又冻我一夜不成?"正自胡猜,只见黑аа的来了一个人,贾瑞便意定是凤姐,不管皂白,饿虎一般,等那人刚至门前,便如猫捕鼠的一般,抱住叫道:“亲嫂子,等死我了。”说着,抱到屋里炕上就亲嘴扯裤子,满口里"亲娘”“亲爹"的乱叫起来。那人只不作声。贾瑞拉了自己裤子,硬帮帮的就想顶入。忽见灯光一闪,只见贾蔷举着个捻子照道:“谁在屋里?"只见炕上那人笑道:“瑞大叔要臊我呢。”贾瑞一见,却是贾蓉,真臊的无地可入,不知要怎么样才好,回身就要跑,被贾蔷一把揪住道:“别走!如今琏二嫂已经告到太太跟前,说你无故调戏他。他暂用了个脱身计,哄你在这边等着,太太气死过去,因此叫我来拿你。刚才你又拦住他,没的说,跟我去见太太!”

贾瑞听了,魂不附体,只说:“好侄儿,只说没有见我,明日我重重的谢你。”贾蔷道:“你若谢我,放你不值什么,只不知你谢我多少?况且口说无凭,写一文契来。”贾瑞道:“这如何落纸呢?"贾蔷道:“这也不妨,写一个赌钱输了外人帐目,借头家银若干两便罢。”贾瑞道:“这也容易。只是此时无纸笔。”贾蔷道:“这也容易。”说罢翻身出来,纸笔现成,拿来命贾瑞写。他两作好作歹,只写了五十两,然后画了押,贾蔷收起来。然后撕逻贾蓉。贾蓉先咬定牙不依,只说:“明日告诉族中的人评评理。”贾瑞急的至于叩头。贾蔷作好作歹的,也写了一张五十两欠契才罢。贾蔷又道:“如今要放你,我就担着不是。老太太那边的门早已关了,老爷正在厅上看南京的东西,那一条路定难过去,如今只好走后门。若这一走,倘或遇见了人,连我也完了。等我们先去哨探哨探,再来领你。这屋你还藏不得,少时就来堆东西。等我寻个地方。”说毕,拉着贾瑞,仍熄了灯,出至院外,摸着大台矶底下,说道:“这窝儿里好,你只蹲着,别哼一声,等我们来再动。”说毕,二人去了。

贾瑞此时身不由己,只得蹲在那里。心下正盘算,只听头顶上一声响,б拉拉一净桶尿粪从上面直泼下来,可巧浇了他一身一头。贾瑞掌不住嗳哟了一声,忙又掩住口,不敢声张,满头满脸浑身皆是尿屎,冰冷打战。只见贾蔷跑来叫:“快走,快走!"贾瑞如得了命,三步两步从后门跑到家里,天已三更,只得叫门。开门人见他这般景况,问是怎的。少不得扯谎说:“黑了,失脚掉在茅厕里了。”一面到了自己房中更衣洗濯,心下方想到是凤姐顽他,因此发一回恨,再想想凤姐的模样儿,又恨不得一时搂在怀内,一夜竟不曾合眼。

自此满心想凤姐,只不敢往荣府去了。贾蓉两个又常常的来索银子,他又怕祖父知道,正是相思尚且难禁,更又添了债务,日间工课又紧,他二十来岁人,尚未娶亲,迩来想着凤姐,未免有那指头告了消乏等事,更兼两回冻恼奔波,因此三五下里夹攻,不觉就得了一病:心内发膨胀,口中无滋味,脚下如绵,眼中似醋,黑夜作烧,白昼常倦,下溺连精,嗽痰带血。诸如此症,不上一年都添全了。于是不能支持,一头睡倒,合上眼还只梦魂颠倒,满口乱说胡话,惊怖异常。百般请医疗治,诸如肉桂,附子,鳖甲,麦冬,玉竹等药,吃了有几十斤下去,也不见个动静。倏又腊尽春回,这病更又沉重。代儒也着了忙,各处请医疗治,皆不见效。因后来吃"独参汤",代儒如何有这力量,只得往荣府来寻。王夫人命凤姐秤二两给他,凤姐回说:“前儿新近都替老太太配了药,那整的太太又说留着送杨提督的太太配药,偏生昨儿我已送了去了。”王夫人道:“就是咱们这边没了,你打发个人往你婆婆那边问问,或是你珍大哥哥那府里再寻些来,凑着给人家。吃好了,救人一命,也是你的好处。”凤姐听了,也不遣人去寻,只得将些渣末泡须凑了几钱,命人送去,只说:“太太送来的,再也没了。”然后回王夫人,只说:“都寻了来,共凑了有二两送去。”

那贾瑞此时要命心甚切,无药不吃,只是白花钱,不见效。忽然这日有个跛足道人来化斋,口称专治冤业之症。贾瑞偏生在内就听见了,直着声叫喊说:“快请进那位菩萨来救我!"一面叫,一面在枕上叩首。众人只得带了那道士进来。贾瑞一把拉住,连叫"菩萨救我!"那道士叹道:“你这病非药可医。我有个宝贝与你,你天天看时,此命可保矣。”说毕,从褡裢中取出一面镜子来-两面皆可照人,镜把上面錾着"风月宝鉴"四字-递与贾瑞道:“这物出自太虚幻境空灵殿上,警幻仙子所制,专治邪思妄动之症,有济世保生之功。所以带他到世上,单与那些聪明杰俊,风雅王孙等看照。千万不可照正面,只照他的背面,要紧,要紧!三日后吾来收取,管叫你好了。”说毕,佯常而去,众人苦留不住。

贾瑞收了镜子,想道:“这道士倒有意思,我何不照一照试试。”想毕,拿起"风月鉴"来,向反面一照,只见一个骷髅立在里面,唬得贾瑞连忙掩了,骂:“道士混帐,如何吓我!-我倒再照照正面是什么。”想着,又将正面一照,只见凤姐站在里面招手叫他。贾瑞心中一喜,荡悠悠的觉得进了镜子,与凤姐云雨一番,凤姐仍送他出来。到了床上,哎哟了一声,一睁眼,镜子从手里掉过来,仍是反面立着一个骷髅。贾瑞自觉汗津津的,底下已遗了一滩精。心中到底不足,又翻过正面来,只见凤姐还招手叫他,他又进去。如此三四次。到了这次,刚要出镜子来,只见两个人走来,拿铁锁把他套住,拉了就走。贾瑞叫道:“让我拿了镜子再走。”-只说了这句,就再不能说话了。

旁边伏侍贾瑞的众人,只见他先还拿着镜子照,落下来,仍睁开眼拾在手内,末后镜子落下来便不动了。众人上来看看,已没了气。身子底下冰凉渍湿一大滩精,这才忙着穿衣抬床。代儒夫妇哭的死去活来,大骂道士,"是何妖镜!若不早毁此物,遗害于世不小。”遂命架火来烧,只听镜内哭道:“谁叫你们瞧正面了!你们自己以假为真,何苦来烧我?"正哭着,只见那跛足道人从外面跑来,喊道:“谁毁`风月鉴-,吾来救也!"说着,直入中堂,抢入手内,飘然去了。

当下,代儒料理丧事,各处去报丧。三日起经,七日发引,寄灵于铁槛寺,日后带回原籍。当下贾家众人齐来吊问,荣国府贾赦赠银二十两,贾政亦是二十两,宁国府贾珍亦有二十两,别者族中贫富不等,或三两五两,不可胜数。另有各同窗家分资,也凑了二三十两。代儒家道虽然淡薄,倒也丰丰富富完了此事。

谁知这年冬底,林如海的书信寄来,却为身染重疾,写书特来接林黛玉回去。贾母听了,未免又加忧闷,只得忙忙的打点黛玉起身。宝玉大不自在,争奈父女之情,也不好拦劝。于是贾母定要贾琏送他去,仍叫带回来。一应土仪盘缠,不消烦说,自然要妥贴。作速择了日期,贾琏与林黛玉辞别了贾母等,带领仆从,登舟往扬州去了。要知端的,且听下回分解。

\chapter{秦可卿死封龙禁尉\ttlbreak 王熙凤协理宁国府}

话说凤姐儿自贾琏送黛玉往扬州去后,心中实在无趣,每到晚间,不过和平儿说笑一回,就胡乱睡了。

这日夜间,正和平儿灯下拥炉倦绣,早命浓薰绣被,二人睡下,屈指算行程该到何处,不知不觉已交三鼓。平儿已睡熟了。凤姐方觉星眼微朦,恍惚只见秦氏从外走来,含笑说道:“婶子好睡!我今日回去,你也不送我一程。因娘儿们素日相好,我舍不得婶子,故来别你一别。还有一件心愿未了,非告诉婶子,别人未必中用。”

凤姐听了,恍惚问道:“有何心愿?你只管托我就是了。”秦氏道:“婶婶,你是个脂粉队里的英雄,连那些束带顶冠的男子也不能过你,你如何连两句俗语也不晓得?常言`月满则亏,水满则溢-,又道是`登高必跌重-。如今我们家赫赫扬扬,已将百载,一日倘或乐极悲生,若应了那句`树倒猢狲散-的俗语,岂不虚称了一世的诗书旧族了!"凤姐听了此话,心胸大快,十分敬畏,忙问道:“这话虑的极是,但有何法可以永保无虞?"秦氏冷笑道:“婶子好痴也。否极泰来,荣辱自古周而复始,岂人力能可保常的。但如今能于荣时筹画下将来衰时的世业,亦可谓常保永全了。即如今日诸事都妥,只有两件未妥,若把此事如此一行,则后日可保永全了。”

凤姐便问何事。秦氏道:“目今祖茔虽四时祭祀,只是无一定的钱粮,第二,家塾虽立,无一定的供给。依我想来,如今盛时固不缺祭祀供给,但将来败落之时,此二项有何出处?莫若依我定见,趁今日富贵,将祖茔附近多置田庄房舍地亩,以备祭祀供给之费皆出自此处,将家塾亦设于此。合同族中长幼,大家定了则例,日后按房掌管这一年的地亩,钱粮,祭祀,供给之事。如此周流,又无争竞,亦不有典卖诸弊。便是有了罪,凡物可入官,这祭祀产业连官也不入的。便败落下来,子孙回家读书务农,也有个退步,祭祀又可永继。若目今以为荣华不绝,不思后日,终非长策。眼见不日又有一件非常喜事,真是烈火烹油,鲜花着锦之盛。要知道,也不过是瞬间的繁华,一时的欢乐,万不可忘了那`盛筵必散-的俗语。此时若不早为后虑,临期只恐后悔无益了。”凤姐忙问:“有何喜事?"秦氏道:“天机不可泄漏。只是我与婶子好了一场,临别赠你两句话,须要记着。”因念道:

三春过后诸芳尽,各自须寻各自门。凤姐还欲问时,只听二门上传事云板连叩四下,将凤姐惊醒。人回:“东府蓉大奶奶没了。”凤姐闻听,吓了一身冷汗,出了一回神,只得忙忙的穿衣,往王夫人处来。

彼时合家皆知,无不纳罕,都有些疑心。那长一辈的想他素日孝顺,平一辈的想他素日和睦亲密,下一辈的想他素日慈爱,以及家中仆从老小想他素日怜贫惜贱,慈老爱幼之恩,莫不悲嚎痛哭者。

闲言少叙,却说宝玉因近日林黛玉回去,剩得自己孤в,也不和人顽耍,每到晚间便索然睡了。如今从梦中听见说秦氏死了,连忙翻身爬起来,只觉心中似戳了一刀的不忍,哇的一声,直奔出一口血来。袭人等慌慌忙忙上来д扶,问是怎么样,又要回贾母来请大夫。宝玉笑道:“不用忙,不相干,这是急火攻心,血不归经。”说着便爬起来,要衣服换了,来见贾母,即时要过去。袭人见他如此,心中虽放不下,又不敢拦,只是由他罢了。贾母见他要去,因说:“才г气的人,那里不干净,二则夜里风大,等明早再去不迟。”宝玉那里肯依。贾母命人备车,多派跟随人役,拥护前来。一直到了宁国府前,只见府门洞开,两边灯笼照如白昼,乱烘烘人来人往,里面哭声摇山振岳。宝玉下了车,忙忙奔至停灵之室,痛哭一番。然后见过尤氏。谁知尤氏正犯了胃疼旧疾,睡在床上。然后又出来见贾珍。彼时贾代儒,代修,贾敕,贾效,贾敦,贾赦,贾政,贾琮,贾е,贾珩,贾ё,贾琛,贾琼,贾ж,贾蔷,贾菖,贾菱,贾芸,贾芹,贾蓁,贾萍,贾藻,贾蘅,贾芬,贾芳,贾兰,贾菌,贾芝等都来了。贾珍哭的泪人一般,正和贾代儒等说道:“合家大小,远近亲友,谁不知我这媳妇比儿子还强十倍。如今伸腿去了,可见这长房内绝灭无人了。”说着又哭起来。众人忙劝:“人已辞世,哭也无益,且商议如何料理要紧。”贾珍拍手道:“如何料理,不过尽我所有罢了!"正说着,只见秦业,秦钟并尤氏的几个眷属尤氏姊妹也都来了。贾珍便命贾琼,贾琛,贾ж,贾蔷四个人去陪客,一面吩咐去请钦天监陰阳司来择日,择准停灵七七四十九日,三日后开丧送讣闻。这四十九日,单请一百单八众禅僧在大厅上拜大悲忏,超度前亡后化诸魂,以免亡者之罪,另设一坛于天香楼上,是九十九位全真道士,打四十九日解冤洗业醮。然后停灵于会芳园中,灵前另外五十众高僧,五十众高道,对坛按七作好事。那贾敬闻得长孙媳死了,因自为早晚就要飞升,如何肯又回家染了红尘,将前功尽弃呢,因此并不在意,只凭贾珍料理。

贾珍见父亲不管,亦发恣意奢华。看板时,几副杉木板皆不中用。可巧薛蟠来吊问,因见贾珍寻好板,便说道:“我们木店里有一副板,叫作什么樯木,出在潢海铁网山上,作了棺材,万年不坏。这还是当年先父带来,原系义忠亲王老千岁要的,因他坏了事,就不曾拿去。现在还封在店内,也没有人出价敢买。你若要,就抬来使罢。”贾珍听说,喜之不尽,即命人抬来。大家看时,只见帮底皆厚八寸,纹若槟榔,味若檀麝,以手扣之,玎з如金玉。大家都奇异称赞。贾珍笑问:“价值几何?"薛蟠笑道:“拿一千两银子来,只怕也没处买去。什么价不价,赏他们几两工钱就是了。”贾珍听说,忙谢不尽,即命解锯糊漆。贾政因劝道:“此物恐非常人可享者,殓以上等杉木也就是了。”此时贾珍恨不能代秦氏之死,这话如何肯听。因忽又听得秦氏之丫鬟名唤瑞珠者,见秦氏死了,他也触柱而亡。此事可罕,合族人也都称叹。贾珍遂以孙女之礼敛殡,一并停灵于会芳园中之登仙阁。小丫鬟名宝珠者,因见秦氏身无所出,乃甘心愿为义女,誓任摔丧驾灵之任。贾珍喜之不尽,即时传下,从此皆呼宝珠为小姐。那宝珠按未嫁女之丧,在灵前哀哀欲绝。于是,合族人丁并家下诸人,都各遵旧制行事,自不得紊乱。

贾珍因想着贾蓉不过是个黉门监,灵幡经榜上写时不好看,便是执事也不多,因此心下甚不自在。可巧这日正是首七第四日,早有大明宫掌宫内相戴权,先备了祭礼遣人来,次后坐了大轿,打伞鸣锣,亲来上祭。贾珍忙接着,让至逗蜂轩献茶。贾珍心中打算定了主意,因而趁便就说要与贾蓉捐个前程的话。戴权会意,因笑道:“想是为丧礼上风光些。”贾珍忙笑道:“老内相所见不差。”戴权道:“事倒凑巧,正有个美缺,如今三百员龙禁尉短了两员,昨儿襄阳侯的兄弟老三来求我,现拿了一千五百两银子,送到我家里。你知道,咱们都是老相与,不拘怎么样,看着他爷爷的分上,胡乱应了。还剩了一个缺,谁知永兴节度使冯胖子来求,要与他孩子捐,我就没工夫应他。既是咱们的孩子要捐,快写个履历来。”贾珍听说,忙吩咐:“快命书房里人恭敬写了大爷的履历来。”小厮不敢怠慢,去了一刻,便拿了一张红纸来与贾珍。贾珍看了,忙送与戴权。看时,上面写道:

江南江宁府江宁县监生贾蓉,年二十岁。曾祖,原

任京营节度使世袭一等神威将军贾代化,祖,乙卯科进士贾

敬,父,世袭三品爵威烈将军贾珍。戴权看了,回手便递与一个贴身的小厮收了,说道:“回来送与户部堂官老赵,说我拜上他,起一张五品龙禁尉的票,再给个执照,就把这履历填上,明儿我来兑银子送去。”小厮答应了,戴权也就告辞了。贾珍十分款留不住,只得送出府门。临上轿,贾珍因问:“银子还是我到部兑,还是一并送入老内相府中?"戴权道:“若到部里,你又吃亏了。不如平准一千二百两银子,送到我家就完了。”贾珍感谢不尽,只说:“待服满后,亲带小犬到府叩谢。”于是作别。

接着,便又听喝道之声,原来是忠靖侯史鼎的夫人来了。王夫人,邢夫人,凤姐等刚迎入上房,又见锦乡侯,川宁侯,寿山伯三家祭礼摆在灵前。少时,三人下轿,贾政等忙接上大厅。如此亲朋你来我去,也不能胜数。只这四十九日,宁国府街上一条白漫漫人来人往,花簇簇官去官来。

贾珍命贾蓉次日换了吉服,领凭回来。灵前供用执事等物俱按五品职例。灵牌疏上皆写"天朝诰授贾门秦氏恭人之灵位"。会芳园临街大门洞开,旋在两边起了鼓乐厅,两班青衣按时奏乐,一对对执事摆的刀斩斧齐。更有两面朱红销金大字牌对竖在门外,上面大书:“防护内廷紫禁道御前侍卫龙禁尉"。对面高起着宣坛,僧道对坛榜文,榜上大书:“世袭宁国公冢孙妇,防护内廷御前侍卫龙禁尉贾门秦氏恭人之丧。四大部州至中之地,奉天承运太平之国,总理虚无寂静教门僧录司正堂万虚,总理元始三一教门道录司正堂叶生等,敬谨修斋,朝天叩佛",以及"恭请诸伽蓝,揭谛,功曹等神,圣恩普锡,神威远镇,四十九日消灾洗业平安水陆道场"等语,亦不消烦记。

只是贾珍虽然此时心意满足,但里面尤氏又犯了旧疾,不能料理事务,惟恐各诰命来往,亏了礼数,怕人笑话,因此心中不自在。当下正忧虑时,因宝玉在侧问道:“事事都算安贴了,大哥哥还愁什么?"贾珍见问,便将里面无人的话说了出来。宝玉听说笑道:“这有何难,我荐一个人与你权理这一个月的事,管必妥当。”贾珍忙问:“是谁?"宝玉见座间还有许多亲友,不便明言,走至贾珍耳边说了两句。贾珍听了喜不自禁,连忙起身笑道:“果然安贴,如今就去。”说着拉了宝玉,辞了众人,便往上房里来。

可巧这日非正经日期,亲友来的少,里面不过几位近亲堂客,邢夫人,王夫人,凤姐并合族中的内眷陪坐。闻人报:“大爷进来了。”唬的众婆娘唿的一声,往后藏之不迭,独凤姐款款站了起来。贾珍此时也有些病症在身,二则过于悲痛了,因拄个拐踱了进来。邢夫人等因说道:“你身上不好,又连日事多,该歇歇才是,又进来做什么?"贾珍一面扶拐,扎挣着要蹲身跪下请安道乏。邢夫人等忙叫宝玉搀住,命人挪椅子来与他坐。贾珍断不肯坐,因勉强陪笑道:“侄儿进来有一件事要求二位婶子并大妹妹。”邢夫人等忙问:“什么事?"贾珍忙笑道:“婶子自然知道,如今孙子媳妇没了,侄儿媳妇偏又病倒,我看里头着实不成个体统。怎么屈尊大妹妹一个月,在这里料理料理,我就放心了。”邢夫人笑道:“原来为这个。你大妹妹现在你二婶子家,只和你二婶子说就是了。”王夫人忙道:“他一个小孩子家,何曾经过这样事,倘或料理不清,反叫人笑话,倒是再烦别人好。”贾珍笑道:“婶子的意思侄儿猜着了,是怕大妹妹劳苦了。若说料理不开,我包管必料理的开,便是错一点儿,别人看着还是不错的。从小儿大妹妹顽笑着就有杀伐决断,如今出了阁,又在那府里办事,越发历练老成了。我想了这几日,除了大妹妹再无人了。婶子不看侄儿,侄儿媳妇的分上,只看死了的分上罢!"说着滚下泪来。

王夫人心中怕的是凤姐儿未经过丧事,怕他料理不清,惹人耻笑。今见贾珍苦苦的说到这步田地,心中已活了几分,却又眼看着凤姐出神。那凤姐素日最喜揽事办,好卖弄才干,虽然当家妥当,也因未办过婚丧大事,恐人还不伏,巴不得遇见这事。今见贾珍如此一来,他心中早已欢喜。先见王夫人不允,后见贾珍说的情真,王夫人有活动之意,便向王夫人道:“大哥哥说的这么恳切,太太就依了罢。”王夫人悄悄的道:“你可能么?"凤姐道:“有什么不能的。外面的大事已经大哥哥料理清了,不过是里头照管照管,便是我有不知道的,问问太太就是了。”王夫人见说的有理,便不作声。贾珍见凤姐允了,又陪笑道:“也管不得许多了,横竖要求大妹妹辛苦辛苦。我这里先与妹妹行礼,等事完了,我再到那府里去谢。”说着就作揖下去,凤姐儿还礼不迭。

贾珍便忙向袖中取了宁国府对牌出来,命宝玉送与凤姐,又说:“妹妹爱怎样就怎样,要什么只管拿这个取去,也不必问我。只求别存心替我省钱,只要好看为上,二则也要同那府里一样待人才好,不要存心怕人抱怨。只这两件外,我再没不放心的了。”凤姐不敢就接牌,只看着王夫人。王夫人道:“你哥哥既这么说,你就照看照看罢了。只是别自作主意,有了事,打发人问你哥哥,嫂子要紧。”宝玉早向贾珍手里接过对牌来,强递与凤姐了。又问:“妹妹住在这里,还是天天来呢?若是天天来,越发辛苦了。不如我这里赶着收拾出一个院落来,妹妹住过这几日倒安稳。”凤姐笑道:“不用。那边也离不得我,倒是天天来的好。”贾珍听说,只得罢了。然后又说了一回闲话,方才出去。

一时女眷散后,王夫人因问凤姐:“你今儿怎么样?"凤姐儿道:“太太只管请回去,我须得先理出一个头绪来,才回去得呢。”王夫人听说,便先同邢夫人等回去,不在话下。

这里凤姐儿来至三间一所抱厦内坐了,因想:头一件是人口混杂,遗失东西,第二件,事无专执,临期推委,第三件,需用过费,滥支冒领,第四件,任无大小,苦乐不均,第五件,家人豪纵,有脸者不服钤束,无脸者不能上进。此五件实是宁国府中风俗,不知凤姐如何处治,且听下回分解。正是:

金紫万千谁治国,裙钗一二可齐家。

\chapter{林如海捐馆扬州城\ttlbreak 贾宝玉路谒北静王}

话说宁国府中都总管来升闻得里面委请了凤姐,因传齐同事人等说道:“如今请了西府里琏二奶奶管理内事,倘或他来支取东西,或是说话,我们须要比往日小心些。每日大家早来晚散,宁可辛苦这一个月,过后再歇着,不要把老脸丢了。那是个有名的烈货,脸酸心硬,一时恼了,不认人的。”众人都道:“有理。”又有一个笑道:“论理,我们里面也须得他来整治整治,都忒不像了。”正说着,只见来旺媳妇拿了对牌来领取呈文京榜纸札,票上批着数目。众人连忙让坐倒茶,一面命人按数取纸来抱着,同来旺媳妇一路来至仪门口,方交与来旺媳妇自己抱进去了。

凤姐即命彩明钉造簿册。即时传来升媳妇,兼要家口花名册来查看,又限于明日一早传齐家人媳妇进来听差等语。大概点了一点数目单册,问了来升媳妇几句话,便坐车回家。一宿无话。至次日,卯正二刻便过来了。那宁国府中婆娘媳妇闻得到齐,只见凤姐正与来升媳妇分派,众人不敢擅入,只在窗外听觑。只听凤姐与来升媳妇道:“既托了我,我就说不得要讨你们嫌了。我可比不得你们奶奶好性儿,由着你们去。再不要说你们`这府里原是这样-的话,如今可要依着我行,错我半点儿,管不得谁是有脸的,谁是没脸的,一例现清白处理。”说着,便吩咐彩明念花名册,按名一个一个的唤进来看视。

一时看完,便又吩咐道:“这二十个分作两班,一班十个,每日在里头单管人客来往倒茶,别的事不用他们管。这二十个也分作两班,每日单管本家亲戚茶饭,别的事也不用他们管。这四十个人也分作两班,单在灵前上香添油,挂幔守灵,供饭供茶,随起举哀,别的事也不与他们相干。这四个人单在内茶房收管杯碟茶器,若少一件,便叫他四个描赔。这四个人单管酒饭器皿,少一件,也是他四个描赔。这八个单管监收祭礼。这八个单管各处灯油,蜡烛,纸札,我总支了来,交与你八个,然后按我的定数再往各处去分派。这三十个每日轮流各处上夜,照管门户,监察火烛,打扫地方。这下剩的按着房屋分开,某人守某处,某处所有桌椅古董起,至于痰盒掸帚,一草一苗,或丢或坏,就和守这处的人算帐描赔。来升家的每日揽总查看,或有偷懒的,赌钱吃酒的,打架拌嘴的,立刻来回我,你有徇情,经我查出,三四辈子的老脸就顾不成了。如今都有定规,以后那一行乱了,只和那一行说话。素日跟我的人,随身自有钟表,不论大小事,我是皆有一定的时辰。横竖你们上房里也有时辰钟。卯正二刻我来点卯,巳正吃早饭,凡有领牌回事的,只在午初刻。戌初烧过黄昏纸,我亲到各处查一遍,回来上夜的交明钥匙。第二日仍是卯正二刻过来。说不得咱们大家辛苦这几日罢,事完了,你们家大爷自然赏你们。”

说罢,又吩咐按数发与茶叶,油烛,鸡毛掸子,笤帚等物。一面又搬取家伙:桌围,椅搭,坐褥,毡席,痰盒,脚踏之类。一面交发,一面提笔登记,某人管某处,某人领某物,开得十分清楚。众人领了去,也都有了投奔,不似先时只拣便宜的做,剩下的苦差没个招揽。各房中也不能趁乱失迷东西。便是人来客往,也都安静了,不比先前一个正摆茶,又去端饭,正陪举哀,又顾接客。如这些无头绪,荒乱,推托,偷闲,窃取等弊,次日一概都Ь了。

凤姐儿见自己威重令行,心中十分得意。因见尤氏犯病,贾珍又过于悲哀,不大进饮食,自己每日从那府中煎了各样细粥,精致小菜,命人送来劝食。贾珍也另外吩咐每日送上等菜到抱厦内,单与凤姐。那凤姐不畏勤劳,天天于卯正二刻就过来点卯理事,独在抱厦内起坐,不与众妯娌合群,便有堂客来往,也不迎会。

这日乃五七正五日上,那应佛僧正开方破狱,传灯照亡,参阎君,拘都鬼,筵请地藏王,开金桥,引幢幡,那道士们正伏章申表,朝三清,叩玉帝,禅僧们行香,放焰口,拜水忏,又有十三众尼僧,搭绣衣,и红鞋,在灵前默诵接引诸咒,十分热闹。那凤姐必知今日人客不少,在家中歇宿一夜,至寅正,平儿便请起来梳洗。及收拾完备,更衣プ手,吃了两口奶子糖粳米粥,漱口已毕,已是卯正二刻了。来旺媳妇率领诸人伺候已久。凤姐出至厅前,上了车,前面打了一对明角灯,大书"荣国府"三个大字,款款来至宁府。大门上门灯朗挂,两边一色戳灯,照如白昼,白汪汪穿孝仆从两边侍立。请车至正门上,小厮等退去,众媳妇上来揭起车帘。凤姐下了车,一手扶着丰儿,两个媳妇执着手把灯罩,簇拥着凤姐进来。宁府诸媳妇迎来请安接待。凤姐缓缓走入会芳园中登仙阁灵前,一见了棺材,那眼泪恰似断线之珠,滚将下来。院中许多小厮垂手伺候烧纸。凤姐吩咐得一声:“供茶烧纸。”只听一棒锣鸣,诸乐齐奏,早有人端过一张大圈椅来,放在灵前,凤姐坐了,放声大哭。于是里外男女上下,见凤姐出声,都忙忙接声嚎哭。

一时贾珍尤氏遣人来劝,凤姐方才止住。来旺媳妇献茶漱口毕,凤姐方起身,别过族中诸人,自入抱厦内来。按名查点,各项人数都已到齐,只有迎送亲客上的一人未到。即命传到,那人已张惶愧惧。凤姐冷笑道:“我说是谁误了,原来是你!你原比他们有体面,所以才不听我的话。”那人道:“小的天天都来的早,只有今儿,醒了觉得早些,因又睡迷了,来迟了一步,求奶奶饶过这次。”正说着,只见荣国府中的王兴媳妇来了,在前探头。

凤姐且不发放这人,却先问:“王兴媳妇作什么?"王兴媳妇巴不得先问他完了事,连忙进去说:“领牌取线,打车轿网络。”说着,将个帖儿递上去。凤姐命彩明念道:“大轿两顶,小轿四顶,车四辆,共用大小络子若干根,用珠儿线若干斤。”凤姐听了,数目相合,便命彩明登记,取荣国府对牌掷下。王兴家的去了。

凤姐方欲说话时,见荣国府的四个执事人进来,都是要支取东西领牌来的。凤姐命彩明要了帖念过,听了一共四件,指两件说道:“这两件开销错了,再算清了来取。”说着掷下帖子来。那二人扫兴而去。

凤姐因见张材家的在旁,因问:“你有什么事?"张材家的忙取帖儿回说:“就是方才车轿围作成,领取裁缝工银若干两。”凤姐听了,便收了帖子,命彩明登记。待王兴家的交过牌,得了买办的回押相符,然后方与张材家的去领。一面又命念那一个,是为宝玉外书房完竣,支买纸料糊裱。凤姐听了,即命收帖儿登记,待张材家的缴清,又发与这人去了。

凤姐便说道:“明儿他也睡迷了,后儿我也睡迷了,将来都没了人了。本来要饶你,只是我头一次宽了,下次人就难管,不如现开发的好。”登时放下脸来,喝命:“带出去,打二十板子!"一面又掷下宁国府对牌:“出去说与来升,革他一月银米!"众人听说,又见凤姐眉立,知是恼了,不敢怠慢,拖人的出去拖人,执牌传谕的忙去传谕。那人身不由己,已拖出去挨了二十大板,还要进来叩谢。凤姐道:“明日再有误的,打四十,后日的六十,有要挨打的,只管误!"说着,吩咐:“散了罢。”窗外众人听说,方各自执事去了。彼时宁府荣府两处执事领牌交牌的,人来人往不绝,那抱愧被打之人含羞去了,这才知道凤姐利害。众人不敢偷闲,自此兢兢业业,执事保全。不在话下。

如今且说宝玉因见今日人众,恐秦钟受了委曲,因默与他商议,要同他往凤姐处来坐。秦钟道:“他的事多,况且不喜人去,咱们去了,他岂不烦腻。”宝玉道:“他怎好腻我们,不相干,只管跟我来。”说着,便拉了秦钟,直至抱厦。凤姐才吃饭,见他们来了,便笑道:“好长腿子,快上来罢。”宝玉道:“我们偏了。”凤姐道:“在这边外头吃的,还是那边吃的?"宝玉道:“这边同那些浑人吃什么!原是那边,我们两个同老太太吃了来的。”一面归坐。

凤姐吃毕饭,就有宁国府中的一个媳妇来领牌,为支取香灯事。凤姐笑道:“我算着你们今儿该来支取,总不见来,想是忘了。这会子到底来取,要忘了,自然是你们包出来,都便宜了我。”那媳妇笑道:“何尝不是忘了,方才想起来,再迟一步,也领不成了。”说罢,领牌而去。

一时登记交牌。秦钟因笑道:“你们两府里都是这牌,倘或别人私弄一个,支了银子跑了,怎样?"凤姐笑道:“依你说,都没王法了。”宝玉因道:“怎么咱们家没人领牌子做东西?"凤姐道:“人家来领的时候,你还做梦呢。我且问你,你们这夜书多早晚才念呢?"宝玉道:“巴不得这如今就念才好,他们只是不快收拾出书房来,这也无法。”凤姐笑道:“你请我一请,包管就快了。”宝玉道:“你要快也不中用,他们该作到那里的,自然就有了。”凤姐笑道:“便是他们作,也得要东西,搁不住我不给对牌是难的。”宝玉听说,便猴向凤姐身上立刻要牌,说:“好姐姐,给出牌子来,叫他们要东西去。”凤姐道:“我乏的身子上生疼,还搁的住柔搓。你放心罢,今儿才领了纸裱糊去了,他们该要的还等叫去呢,可不傻了?"宝玉不信,凤姐便叫彩明查册子与宝玉看了。正闹着,人回:“苏州去的人昭儿来了。”凤姐急命唤进来。昭儿打千儿请安。凤姐便问:“回来做什么的?"昭儿道:“二爷打发回来的。林姑老爷是九月初三日巳时没的。”二爷带了林姑娘同送林姑老爷灵到苏州,大约赶年底就回来。二爷打发小的来报个信请安,讨老太太示下,还瞧瞧奶奶家里好,叫把大毛衣服带几件去。”凤姐道:“你见过别人了没有?"昭儿道:“都见过了。”说毕,连忙退去。凤姐向宝玉笑道:“你林妹妹可在咱们家住长了。”宝玉道:“了不得,想来这几日他不知哭的怎样呢。”说着,蹙眉长叹。

凤姐见昭儿回来,因当着人未及细问贾琏,心中自是记挂,待要回去,争奈事情繁杂,一时去了,恐有延迟失误,惹人笑话。少不得耐到晚上回来,复令昭儿进来,细问一路平安信息。连夜打点大毛衣服,和平儿亲自检点包裹,再细细追想所需何物,一并包藏交付昭儿。又细细吩咐昭儿:“在外好生小心伏侍,不要惹你二爷生气,时时劝他少吃酒,别勾引他认得混帐老婆,-回来打折你的腿"等语。赶乱完了,天已四更将尽,总睡下又走了困,不觉天明鸡唱,忙梳洗过宁府中来。

那贾珍因见发引日近。亲自坐车,带了陰阳司吏,往铁槛寺来踏看寄灵所在。又一一嘱咐住持色空,好生预备新鲜陈设,多请名僧,以备接灵使用。色空忙看晚斋。贾珍也无心茶饭,因天晚不得进城,就在净室胡乱歇了一夜。次日早,便进城来料理出殡之事,一面又派人先往铁槛寺,连夜另外修饰停灵之处,并厨茶等项接灵人口坐落。

里面凤姐见日期有限,也预先逐细分派料理,一面又派荣府中车轿人从跟王夫人送殡,又顾自己送殡去占下处。目今正值缮国公诰命亡故,王邢二夫人又去打祭送殡,西安郡王妃华诞,送寿礼,镇国公诰命生了长男,预备贺礼,又有胞兄王仁连家眷回南,一面写家信禀叩父母并带往之物,又有迎春染病,每日请医服药,看医生启帖,症源,药案等事,亦难尽述。又兼发引在迩,因此忙的凤姐茶饭也没工夫吃得,坐卧不能清净。刚到了宁府,荣府的人又跟到宁府,既回到荣府,宁府的人又找到荣府。凤姐见如此,心中倒十分欢喜,并不偷安推托,恐落人褒贬,因此日夜不暇,筹划得十分的整肃。于是合族上下无不称叹者。

这日伴宿之夕,里面两班小戏并耍百戏的与亲朋堂客伴宿,尤氏犹卧于内室,一应张罗款待,独是凤姐一人周全承应。合族中虽有许多妯娌,但或有羞口的,或有羞脚的,或有不惯见人的,或有惧贵怯官的,种种之类,俱不及凤姐举止舒徐,言语慷慨,珍贵宽大,因此也不把众人放在眼里,挥霍指示,任其所为,目若无人。一夜中灯明火彩,客送官迎,那百般热闹,自不用说的。至天明,吉时已到,一般六十四名青衣请灵,前面铭旌上大书:“奉天洪建兆年不易之朝诰封一等宁国公冢孙妇防护内廷紫禁道御前侍卫龙禁尉享强寿贾门秦氏恭人之灵柩"。一应执事陈设,皆系现赶着新做出来的,一色光艳夺目。宝珠自行未嫁女之礼外,摔丧驾灵,十分哀苦。

那时官客送殡的,有镇国公牛清之孙现袭一等伯牛继宗,理国公柳彪之孙现袭一等子柳芳,齐国公陈翼之孙世袭三品威镇将军陈瑞文,治国公马魁之孙世袭三品威远将军马尚,修国公侯晓明之孙世袭一等子侯孝康,缮国公诰命亡故,故其孙石光珠守孝不曾来得。这六家与宁荣二家,当日所称"八公"的便是。余者更有南安郡王之孙,西宁郡王之孙,忠靖侯史鼎,平原侯之孙世袭二等男蒋子宁,定城侯之孙世袭二等男兼京营游击谢鲸,襄阳侯之孙世袭二等男戚建辉,景田侯之孙五城兵马司裘良。余者锦乡伯公子韩奇,神武将军公子冯紫英,陈也俊,卫若兰等诸王孙公子,不可枚数。堂客算来亦有十来顶大轿,三四十小轿,连家下大小轿车辆,不下百余十乘。连前面各色执事,陈设,百耍,浩浩荡荡,一带摆三四里远。

走不多时,路旁彩棚高搭。设席张筵,和音奏乐,俱是各家路祭:第一座是东平王府祭棚,第二座是南安郡王祭棚,第三座是西宁郡王,第四座是北静郡王的。原来这四王,当日惟北静王功高,及今子孙犹袭王爵。现今北静王水溶年未弱冠,生得形容秀美,情性谦和。近闻宁国公冢孙妇告殂,因想当日彼此祖父相与之情,同难同荣,未以异姓相视,因此不以王位自居,上日也曾探丧上祭,如今又设路奠,命麾下各官在此伺候。自己五更入朝,公事一毕,便换了素服,坐大轿鸣锣张伞而来,至棚前落轿。手下各官两旁拥侍,军民人众不得往还。

一时只见宁府大殡浩浩荡荡,压地银山一般从北而至。早有宁府开路传事人看见,连忙回去报与贾珍。贾珍急命前面驻扎,同贾赦贾政三人连忙迎来,以国礼相见。水溶在轿内欠身含笑答礼,仍以世交称呼接待,并不妄自尊大。贾珍道:“犬妇之丧,累蒙郡驾下临,荫生辈何以克当。”水溶笑道:“世交之谊,何出此言。”遂回头命长府官主祭代奠。贾赦等一旁还礼毕,复身又来谢恩。

水溶十分谦逊,因问贾政道:“那一位是衔宝而诞者?几次要见一见,都为杂冗所阻,想今日是来的,何不请来一会。”贾政听说,忙回去,急命宝玉脱去孝服,领他前来。那宝玉素日就曾听得父兄亲友人等说闲话时,赞水溶是个贤王,且生得才貌双全,风流潇洒,每不以官俗国体所缚。每思相会,只是父亲拘束严密,无由得会,今见反来叫他,自是欢喜。一面走,一面早瞥见那水溶坐在轿内,好个仪表人材。不知近看时又是怎样,且听下回分解。

\chapter{王凤姐弄权铁槛寺\ttlbreak 秦鲸卿得趣馒头庵}

话说宝玉举目见北静王水溶头上戴着洁白簪缨银翅王帽,穿着江牙海水五爪坐龙白蟒袍,系着碧玉红ネ带,面如美玉,目似明星,真好秀丽人物。宝玉忙抢上来参见,水溶连忙从轿内伸出手来挽住。见宝玉戴着束发银冠,勒着双龙出海抹额,穿着白蟒箭袖,围着攒珠银带,面若春花,目如点漆。水溶笑道:“名不虚传,果然如`宝-似`玉。”因问:“衔的那宝贝在那里?"宝玉见问,连忙从衣内取了递与过去。水溶细细的看了,又念了那上头的字,因问:“果灵验否?"贾政忙道:“虽如此说,只是未曾试过。”水溶一面极口称奇道异,一面理好彩绦,亲自与宝玉带上,又携手问宝玉几岁,读何书。宝玉一一的答应。

水溶见他语言清楚,谈吐有致,一面又向贾政笑道:“令郎真乃龙驹凤雏,非小王在世翁前唐突,将来`雏凤清于老凤声-,未可量也。”贾政忙陪笑道:“犬子岂敢谬承金奖。赖蕃郡余祯,果如是言,亦荫生辈之幸矣。”水溶又道:“只是一件,令郎如是资质,想老太夫人,夫人辈自然钟爱极矣,但吾辈后生,甚不宜钟溺,钟溺则未免荒失学业。昔小王曾蹈此辙,想令郎亦未必不如是也。若令郎在家难以用功,不妨常到寒第。小王虽不才,却多蒙海上众名士凡至都者,未有不另垂青目。是以寒第高人颇聚。令郎常去谈会谈会,则学问可以日进矣。”贾政忙躬身答应。

水溶又将腕上一串念珠卸了下来,递与宝玉道:“今日初会,仓促竟无敬贺之物,此是前日圣上亲赐йк香念珠一串,权为贺敬之礼。”宝玉连忙接了,回身奉与贾政。贾政与宝玉一齐谢过。于是贾赦,贾珍等一齐上来请回舆,水溶道:“逝者已登仙界,非碌碌你我尘寰中之人也。小王虽上叨天恩,虚邀郡袭,岂V可越仙味进也?"贾赦等见执意不从,只得告辞谢恩回来,命手下掩乐停音,滔滔然将殡过完,方让水溶回舆去了。不在话下。

且说宁府送殡,一路热闹非常。刚至城门前,又有贾赦,贾政,贾珍等诸同僚属下各家祭棚接祭,一一的谢过,然后出城,竟奔铁槛寺大路行来。彼时贾珍带贾蓉来到诸长辈前,让坐轿上马,因而贾赦一辈的各自上了车轿,贾珍一辈的也将要上马。凤姐儿因记挂着宝玉,怕他在郊外纵性逞强,不服家人的话,贾政管不着这些小事,惟恐有个失闪,难见贾母,因此便命小厮来唤他。宝玉只得来到他车前。凤姐笑道:“好兄弟,你是个尊贵人,女孩儿一样的人品,别学他们猴在马上。下来,咱们姐儿两个坐车,岂不好?"宝玉听说,忙下了马,爬入凤姐车上,二人说笑前来。不一时,只见从那边两骑马压地飞来,离凤姐车不远,一齐蹿下来,扶车回说:“这里有下处,奶奶请歇更衣。”凤姐急命请邢夫人王夫人的示下,那人回来说:“太太们说不用歇了,叫奶奶自便罢。”凤姐听了,便命歇了再走。众小厮听了,一带辕马,岔出人群,往北飞走。宝玉在车内急命请秦相公。那时秦钟正骑马随着他父亲的轿,忽见宝玉的小厮跑来,请他去打尖。秦钟看时,只见凤姐儿的车往北而去,后面拉着宝玉的马,搭着鞍笼,便知宝玉同凤姐坐车,自己也便带马赶上去,同入一庄门内。早有家人将众庄汉撵尽。那庄农人家无多房舍,婆娘们无处回避,只得由他们去了。那些村姑庄妇见了凤姐,宝玉,秦钟的人品衣服,礼数款段,岂有不爱看的?

一时凤姐进入茅堂,因命宝玉等先出去顽顽。宝玉等会意,因同秦钟出来,带着小厮们各处游顽。凡庄农动用之物,皆不曾见过。宝玉一见了锹,镢,锄,犁等物,皆以为奇,不知何项所使,其名为何。小厮在旁一一的告诉了名色,说明原委。宝玉听了,因点头叹道:“怪道古人诗上说,`谁知盘中餐,粒粒皆辛苦-,正为此也。”一面说,一面又至一间房前,只见炕上有个纺车,宝玉又问小厮们:“这又是什么?"小厮们又告诉他原委。宝玉听说,便上来拧转作耍,自为有趣。只见一个约有十七八岁的村庄丫头跑了来乱嚷:“别动坏了!"众小厮忙断喝拦阻。宝玉忙丢开手,陪笑说道:“我因为没见过这个,所以试他一试。”那丫头道:“你们那里会弄这个,站开了,我纺与你瞧。”秦钟暗拉宝玉笑道:“此卿大有意趣。”宝玉一把推开,笑道:“该死的!再胡说,我就打了。”说着,只见那丫头纺起线来。宝玉正要说话时,只听那边老婆子叫道:“二丫头,快过来!"那丫头听见,丢下纺车,一径去了。

宝玉怅然无趣。只见凤姐儿打发人来叫他两个进去。凤姐洗了手,换衣服抖灰,问他们换不换。宝玉不换,只得罢了。家下仆妇们将带着行路的茶壶茶杯,十锦屉盒,各样小食端来,凤姐等吃过茶,待他们收拾完毕,便起身上车。外面旺儿预备下赏封,赏了本村主人。庄妇等来叩赏。凤姐并不在意,宝玉却留心看时,内中并无二丫头。一时上了车,出来走不多远,只见迎头二丫头怀里抱着他小兄弟,同着几个小女孩子说笑而来。宝玉恨不得下车跟了他去,料是众人不依的,少不得以目相送,争奈车轻马快,一时展眼无踪。

走不多时,仍又跟上大殡了。早有前面法鼓金铙,幢幡宝盖:铁槛寺接灵众僧齐至。少时到入寺中,另演佛事,重设香坛。安灵于内殿偏室之中,宝珠安于里寝室相伴。外面贾珍款待一应亲友,也有扰饭的,也有不吃饭而辞的,一应谢过乏,从公侯伯子男一起一起的散去,至未末时分方才散尽了。里面的堂客皆是凤姐张罗接待,先从显官诰命散起,也到晌午大错时方散尽了。只有几个亲戚是至近的,等做过三日安灵道场方去。那时邢,王二夫人知凤姐必不能来家,也便就要进城。王夫人要带宝玉去,宝玉乍到郊外,那里肯回去,只要跟凤姐住着。王夫人无法,只得交与凤姐便回来了。原来这铁槛寺原是宁荣二公当日修造,现今还是有香火地亩布施,以备京中老了人口,在此便宜寄放。其中陰阳两宅俱已预备妥贴,好为送灵人口寄居。不想如今后辈人口繁盛,其中贫富不一,或性情参商:有那家业艰难安分的,便住在这里了,有那尚排场有钱势的,只说这里不方便,一定另外或村庄或尼庵寻个下处,为事毕宴退之所。即今秦氏之丧,族中诸人皆权在铁槛寺下榻,独有凤姐嫌不方便,因而早遣人来和馒头庵的姑子净虚说了,腾出两间房子来作下处。原来这馒头庵就是水月庵,因他庙里做的馒头好,就起了这个浑号,离铁槛寺不远。当下和尚工课已完,奠过茶饭,贾珍便命贾蓉请凤姐歇息。凤姐见还有几个妯娌陪着女亲,自己便辞了众人,带了宝玉,秦钟往水月庵来。原来秦业年迈多病,不能在此,只命秦钟等待安灵罢了。那秦钟便只跟着凤姐,宝玉,一时到了水月庵,净虚带领智善,智能两个徒弟出来迎接,大家见过。凤姐等来至净室更衣净手毕,因见智能儿越发长高了,模样儿越发出息了,因说道:“你们师徒怎么这些日子也不往我们那里去?"净虚道:“可是这几天都没工夫,因胡老爷府里产了公子,太太送了十两银子来这里,叫请几位师父念三日《血盆经》,忙的没个空儿,就没来请奶奶的安。”不言老尼陪着凤姐。且说秦钟,宝玉二人正在殿上顽耍,因见智能过来,宝玉笑道:“能儿来了。”秦钟道:“理那东西作什么?"宝玉笑道:“你别弄鬼,那一日在老太太屋里,一个人没有,你搂着他作什么?这会子还哄我。”秦钟笑道:“这可是没有的话。”宝玉笑道:“有没有也不管你,你只叫住他倒碗茶来我吃,就丢开手。”秦钟笑道:“这又奇了,你叫他倒去,还怕他不倒?何必要我说呢。”宝玉道:“我叫他倒的是无情意的,不及你叫他倒的是有情意的。”秦钟只得说道:“能儿,倒碗茶来给我。”那智能儿自幼在荣府走动,无人不识,因常与宝玉秦钟顽笑。他如今大了,渐知风月,便看上了秦钟人物风流,那秦钟也极爱他妍媚,二人虽未上手,却已情投意合了。今智能见了秦钟,心眼俱开,走去倒了茶来。秦钟笑道:“给我。”宝玉叫:“给我!"智能儿抿嘴笑道:“一碗茶也争,我难道手里有蜜!"宝玉先抢得了,吃着,方要问话,只见智善来叫智能去摆茶碟子,一时来请他两个去吃茶果点心。他两个那里吃这些东西,坐一坐仍出来顽耍。

凤姐也略坐片时,便回至净室歇息,老尼相送。此时众婆娘媳妇见无事,都陆续散了,自去歇息,跟前不过几个心腹常侍小婢,老尼便趁机说道:“我正有一事,要到府里求太太,先请奶奶一个示下。”凤姐因问何事。老尼道:“阿弥陀佛!只因当日我先在长安县内善才庵内出家的时节,那时有个施主姓张,是大财主。他有个女儿小名金哥,那年都往我庙里来进香,不想遇见了长安府府太爷的小舅子李衙内。那李衙内一心看上,要娶金哥,打发人来求亲,不想金哥已受了原任长安守备的公子的聘定。张家若退亲,又怕守备不依,因此说已有了人家。谁知李公子执意不依,定要娶他女儿,张家正无计策,两处为难。不想守备家听了此言,也不管青红皂白,便来作践辱骂,说一个女儿许几家,偏不许退定礼,就打官司告状起来。那张家急了,只得着人上京来寻门路,赌气偏要退定礼。我想如今长安节度云老爷与府上最契,可以求太太与老爷说声,打发一封书去,求云老爷和那守备说一声,不怕那守备不依。若是肯行,张家连倾家孝顺也都情愿。”

凤姐听了笑道:“这事倒不大,只是太太再不管这样的事。”老尼道:“太太不管,奶奶也可以主张了。”凤姐听说笑道:“我也不等银子使,也不做这样的事。”净虚听了,打去妄想,半晌叹道:“虽如此说,张家已知我来求府里,如今不管这事,张家不知道没工夫管这事,不希罕他的谢礼,倒象府里连这点子手段也没有的一般。”

凤姐听了这话,便发了兴头,说道:“你是素日知道我的,从来不信什么是陰司地狱报应的,凭是什么事,我说要行就行。你叫他拿三千银子来,我就替他出这口气。”老尼听说,喜不自禁,忙说:“有,有!这个不难。”凤姐又道:“我比不得他们扯篷拉牵的图银子。这三千银子,不过是给打发说去的小厮作盘缠,使他赚几个辛苦钱,我一个钱也不要他的。便是三万两,我此刻也拿的出来。”老尼连忙答应,又说道:“既如此,奶奶明日就开恩也罢了。”凤姐道:“你瞧瞧我忙的,那一处少了我?既应了你,自然快快的了结。”老尼道:“这点子事,在别人的跟前就忙的不知怎么样,若是奶奶的跟前,再添上些也不够奶奶一发挥的。只是俗语说的,`能者多劳-,太太因大小事见奶奶妥贴,越性都推给奶奶了,奶奶也要保重金体才是。”一路话奉承的凤姐越发受用,也不顾劳乏,更攀谈起来。

谁想秦钟趁黑无人,来寻智能。刚至后面房中,只见智能独在房中洗茶碗,秦钟跑来便搂着亲嘴。智能急的跺脚说:“这算什么!再这么我就叫唤。”秦钟求道:“好人,我已急死了。你今儿再不依,我就死在这里。”智能道:“你想怎样?除非等我出了这牢坑,离了这些人,才依你。”秦钟道:“这也容易,只是远水救不得近渴。”说着,一口吹了灯,满屋漆黑,将智能抱到炕上,就云雨起来。那智能百般的挣挫不起,又不好叫的,少不得依他了。正在得趣,只见一人进来,将他二人按住,也不则声。二人不知是谁,唬的不敢动一动。只听那人嗤的一声,掌不住笑了,二人听声方知是宝玉。秦钟连忙起来,抱怨道:“这算什么?"宝玉笑道:“你倒不依,咱们就叫喊起来。”羞的智能趁黑地跑了。宝玉拉了秦钟出来道:“你可还和我强?"秦钟笑道:“好人,你只别嚷的众人知道,你要怎样我都依你。”宝玉笑道:“这会子也不用说,等一会睡下,再细细的算帐。”一时宽衣安歇的时节,凤姐在里间,秦钟宝玉在外间,满地下皆是家下婆子,打铺坐更。凤姐因怕通灵玉失落,便等宝玉睡下,命人拿来显谧约赫肀撸宝玉不知与秦钟算何帐目,未见真切,未曾记得,此是疑案,不敢纂创。

一宿无话。至次日一早,便有贾母王夫人打发了人来看宝玉,又命多穿两件衣服,无事宁可回去。宝玉那里肯回去,又有秦钟恋着智能,调唆宝玉求凤姐再住一天。凤姐想了一想:凡丧仪大事虽妥,还有一半点小事未曾安插,可以指此再住一日,岂不又在贾珍跟前送了满情,二则又可以完净虚那事,三则顺了宝玉的心,贾母听见,岂不欢喜?因有此三益,便向宝玉道:“我的事都完了,你要在这里逛,少不得越性辛苦一日罢了,明儿可是定要走的了。”宝玉听说,千姐姐万姐姐的央求:“只住一日,明儿必回去的。”于是又住了一夜。

凤姐便命悄悄将昨日老尼之事,说与来旺儿。来旺儿心中俱已明白,急忙进城找着主文的相公,假托贾琏所嘱,修书一封,连夜往长安县来,不过百里路程,两日工夫俱已妥协。那节度使名唤云光,久见贾府之情,这点小事,岂有不允之理,给了回书,旺儿回来。且不在话下。

却说凤姐等又过一日,次日方别了老尼,着他三日后往府里去讨信。那秦钟与智能百般不忍分离,背地里多少幽期密约,俱不用细述,只得含恨而别。凤姐又到铁槛寺中照望一番。宝珠执意不肯回家,贾珍只得派妇女相伴。后回再见。

\chapter{贾元春才选凤藻宫\ttlbreak 秦鲸卿夭逝黄泉路}

话说宝玉见收拾了外书房,约定与秦钟读夜书。偏那秦钟秉赋最弱,因在郊外受了些风霜,又与智能儿偷期绻缱,未免失于调养,回来时便咳嗽伤风,懒进饮食,大有不胜之状,遂不敢出门,只在家中养息。宝玉便扫了兴头,只得付于无可奈何,且自静候大愈时再约。

那凤姐儿已是得了云光的回信,俱已妥协。老尼达知张家,果然那守备忍气吞声的受了前聘之物。谁知那张家父母如此爱势贪财,却养了一个知义多情的女儿,闻得父母退了前夫,他便一条麻绳悄悄的自缢了。那守备之子闻得金哥自缢,他也是个极多情的,遂也投河而死,不负妻义。张李两家没趣,真是人财两空。这里凤姐却坐享了三千两,王夫人等连一点消息也不知道。自此凤姐胆识愈壮,以后有了这样的事,便恣意的作为起来。也不消多记。

一日正是贾政的生辰,宁荣二处人丁都齐集庆贺,闹热非常。忽有门吏忙忙进来,至席前报说:“有六宫都太监夏老爷来降旨。”唬的贾赦贾政等一干人不知是何消息,忙止了戏文,撤去酒席,摆了香案,启中门跪接。早见六宫都太监夏守忠乘马而至,前后左右又有许多内监跟从。那夏守忠也并不曾负诏捧敕,至檐前下马,满面笑容,走至厅上,南面而立,口内说:“特旨:立刻宣贾政入朝,在临敬殿陛见。”说毕,也不及吃茶,便乘马去了。贾赦等不知是何兆头。只得急忙更衣入朝。

贾母等合家人等心中皆惶惶不定,不住的使人飞马来往报信。有两个时辰工夫,忽见赖大等三四个管家喘吁吁跑进仪门报喜,又说"奉老爷命,速请老太太带领太太等进朝谢恩"等语。那时贾母正心神不定,在大堂廊下伫立,那邢夫人,王夫人,尤氏,李纨,凤姐,迎春姊妹以及薛姨妈等皆在一处,听如此信至,贾母便唤进赖大来细问端的。赖大禀道:“小的们只在临敬门外伺候,里头的信息一概不能得知。后来还是夏太监出来道喜,说咱们家大小姐晋封为凤藻宫尚书,加封贤德妃。后来老爷出来亦如此吩咐小的。如今老爷又往东宫去了,速请老太太领着太太们去谢恩。”贾母等听了方心神安定,不免又都洋洋喜气盈腮。于是都按品大妆起来。贾母带领邢夫人,王夫人,尤氏,一共四乘大轿入朝。贾赦,贾珍亦换了朝服,带领贾蓉,贾蔷奉侍贾母大轿前往。于是宁荣两处上下里外,莫不欣然踊跃,个个面上皆有得意之状,言笑鼎沸不绝。

谁知近日水月庵的智能私逃进城,找至秦钟家下看视秦钟,不意被秦业知觉,将智能逐出,将秦钟打了一顿,自己气的老病发作,三五日光景呜呼死了。秦钟本自怯弱,又带病未愈,受了笞杖,今见老父气死,此时悔痛无及,更又添了许多症候。因此宝玉心中怅然如有所失。虽闻得元春晋封之事,亦未解得愁闷。贾母等如何谢恩,如何回家,亲朋如何来庆贺,宁荣两处近日如何热闹,众人如何得意,独他一个皆视有如无,毫不曾介意。因此众人嘲他越发呆了。且喜贾琏与黛玉回来,先遣人来报信,明日就可到家,宝玉听了,方略有些喜意。细问原由,方知贾雨村亦进京陛见,皆由王子腾累上保本,此来后补京缺,与贾琏是同宗弟兄,又与黛玉有师从之谊,故同路作伴而来。林如海已葬入祖坟了,诸事停妥,贾琏方进京的。本该出月到家,因闻得元春喜信,遂昼夜兼程而进,一路俱各平安。宝玉只问得黛玉"平安"二字,余者也就不在意了。

好容易盼至明日午错,果报:“琏二爷和林姑娘进府了。”见面时彼此悲喜交接,未免又大哭一阵,后又致喜庆之词。宝玉心中品度黛玉,越发出落的超逸了。黛玉又带了许多书籍来,忙着打扫卧室,安插器具,又将些纸笔等物分送宝钗,迎春,宝玉等人。宝玉又将北静王所赠йк香串珍重取出来,转赠黛玉。黛玉说:“什么臭男人拿过的!我不要他。”遂掷而不取。宝玉只得收回,暂且无话。

且说贾琏自回家参见过众人,回至房中。正值凤姐近日多事之时,无片刻闲暇之工,见贾琏远路归来,少不得拨冗接待,房内无外人,便笑道:“国舅老爷大喜!国舅老爷一路风尘辛苦。小的听见昨日的头起报马来报,说今日大驾归府,略预备了一杯水酒掸尘,不知赐光谬领否?"贾琏笑道:“岂敢岂敢,多承多承。”一面平儿与众丫鬟参拜毕,献茶。贾琏遂问别后家中的诸事,又谢凤姐的躁持劳碌。凤姐道:“我那里照管得这些事!见识又浅,口角又笨,心肠又直率,人家给个棒槌,我就认作`针-。脸又软,搁不住人给两句好话,心里就慈悲了。况且又没经历过大事,胆子又小,太太略有些不自在,就吓的我连觉也睡不着了。我苦辞了几回,太太又不容辞,倒反说我图受用,不肯习学了。殊不知我是捻着一把汗儿呢。一句也不敢多说,一步也不敢多走。你是知道的,咱们家所有的这些管家奶奶们,那一位是好缠的?错一点儿他们就笑话打趣,偏一点儿他们就指桑说槐的报怨。`坐山观虎斗-,`借剑杀人-,`引风吹火-,`站干岸儿-,`推倒油瓶不扶-,都是全挂子的武艺。况且我年纪轻,头等不压众,怨不得不放我在眼里。更可笑那府里忽然蓉儿媳妇死了,珍大哥又再三再四的在太太跟前跪着讨情,只要请我帮他几日,我是再四推辞,太太断不依,只得从命。依旧被我闹了个马仰人翻,更不成个体统,至今珍大哥哥还抱怨后悔呢。你这一来了,明儿你见了他,好歹描补描补,就说我年纪小,原没见过世面,谁叫大爷错委他的。”正说着,只听外间有人说话,凤姐便问:“是谁?"平儿进来回道:“姨太太打发了香菱妹子来问我一句话,我已经说了,打发他回去了。”贾琏笑道:“正是呢,方才我见姨妈去,不防和一个年轻的小媳妇子撞了个对面,生的好齐整模样。我疑惑咱家并无此人,说话时因问姨妈,谁知就是上京来买的那小丫头,名叫香菱的,竟与薛大傻子作了房里人,开了脸,越发出挑的标致了。那薛大傻子真玷辱了他。”凤姐道:“嗳!往苏杭走了一趟回来,也该见些世面了,还是这么眼馋肚饱的。你要爱他,不值什么,我去拿平儿换了他来如何?那薛老大也是`吃着碗里看着锅里-的,这一年来的光景,他为要香菱不能到手,和姨妈打了多少饥荒。也因姨妈看着香菱模样儿好还是末则,其为人行事,却又比别的女孩子不同,温柔安静,差不多的主子姑娘也跟他不上呢,故此摆酒请客的费事,明堂正道的与他作了妾。过了没半月,也看的马棚风一般了,我倒心里可惜了的。”一语未了,二门上小厮传报:“老爷在大书房等二爷呢。”贾琏听了,忙忙整衣出去。

这里凤姐乃问平儿:“方才姨妈有什么事,巴巴打发了香菱来?"平儿笑道:“那里来的香菱,是我借他暂撒个谎。奶奶说说,旺儿嫂子越发连个承算也没了。”说着,又走至凤姐身边,悄悄的说道:“奶奶的那利钱银子,迟不送来,早不送来,这会子二爷在家,他且送这个来了。幸亏我在堂屋里撞见,不然时走了来回奶奶,二爷倘或问奶奶是什么利钱,奶奶自然不肯瞒二爷的,少不得照实告诉二爷。我们二爷那脾气,油锅里的钱还要找出来花呢,听见奶奶有了这个梯己,他还不放心的花了呢。所以我赶着接了过来,叫我说了他两句,谁知奶奶偏听见了问,我就撒谎说香菱来了。”凤姐听了笑道:“我说呢,姨妈知道你二爷来了,忽喇巴的反打发个房里人来了?原来你这蹄子у鬼。”

说话时贾琏已进来,凤姐便命摆上酒馔来,夫妻对坐。凤姐虽善饮,却不敢任兴,只陪侍着贾琏。一时贾琏的侞母赵嬷嬷走来,贾琏凤姐忙让吃酒,令其上炕去。赵嬷嬷执意不肯。平儿等早于炕沿下设下一杌,又有一小脚踏,赵嬷嬷在脚踏上坐了。贾琏向桌上拣两盘肴馔与他放在杌上自吃。凤姐又道:“妈妈很嚼不动那个,倒没的辛怂的牙。”因向平儿道:“早起我说那一碗火腿炖肘子很烂,正好给妈妈吃,你怎么不拿了去赶着叫他们热来?"又道:“妈妈,你尝一尝你儿子带来的惠泉酒。”赵嬷嬷道:“我喝呢,奶奶也喝一盅,怕什么?只不要过多了就是了。我这会子跑了来,倒也不为饮酒,倒有一件正经事,奶奶好歹记在心里,疼顾我些罢。我们这爷,只是嘴里说的好,到了跟前就忘了我们。幸亏我从小儿奶了你这么大。我也老了,有的是那两个儿子,你就另眼照看他们些,别人也不敢呲牙儿的。我还再四的求了你几遍,你答应的倒好,到如今还是燥屎。这如今又从天上跑出这一件大喜事来,那里用不着人?所以倒是来和奶奶来说是正经,靠着我们爷,只怕我还饿死了呢。”

凤姐笑道:“妈妈你放心,两个奶哥哥都交给我。你从小儿奶的儿子,你还有什么不知他那脾气的?拿着皮肉倒往那不相干的外人身上贴。可是现放着奶哥哥,那一个不比人强?你疼顾照看他们,谁敢说个`不-字儿?没的白便宜了外人。-我这话也说错了,我们看着是`外人-,你却看着`内人-一样呢。”说的满屋里人都笑了。赵嬷嬷也笑个不住,又念佛道:“可是屋子里跑出青天来了。若说`内人-`外人-这些混帐原故,我们爷是没有,不过是脸软心慈,搁不住人求两句罢了。”凤姐笑道:“可不是呢,有`内人-的他才慈软呢,他在咱们娘儿们跟前才是刚硬呢!"赵嬷嬷笑道:“奶奶说的太尽情了,我也乐了,再吃一杯好酒。从此我们奶奶作了主,我就没的愁了。”

贾琏此时没好意思,只是讪笑吃酒,说`胡说-二字,-"快盛饭来,吃碗子还要往珍大爷那边去商议事呢。”凤姐道:“可是别误了正事。才刚老爷叫你作什么?"贾琏道:“就为省亲。”凤姐忙问道:“省亲的事竟准了不成?"贾琏笑道:“虽不十分准,也有八分准了。”凤姐笑道:“可见当今的隆恩。历来听书看戏,古时从未有的。”赵嬷嬷又接口道:“可是呢,我也老糊涂了。我听见上上下下吵嚷了这些日子,什么省亲不省亲,我也不理论他去,如今又说省亲,到底是怎么个原故?"贾琏道:“如今当今贴体万人之心,世上至大莫如`孝-字,想来父母儿女之性,皆是一理,不是贵贱上分别的。当今自为日夜侍奉太上皇,皇太后,尚不能略尽孝意,因见宫里嫔妃才人等皆是入宫多年,抛离父母音容,岂有不思想之理?在儿女思想父母,是分所应当。想父母在家,若只管思念女儿,竟不能见,倘因此成疾致病,甚至死亡,皆由朕躬禁锢,不能使其遂天轮之愿,亦大伤天和之事。故启奏太上皇,皇太后,每月逢二六日期,准其椒房眷属入宫请候看视。于是太上皇,皇太后大喜,深赞当今至孝纯仁,体天格物。因此二位老圣人又下旨意,说椒房眷属入宫,未免有国体仪制,母女尚不能惬怀。竟大开方便之恩,特降谕诸椒房贵戚,除二六日入宫之恩外,凡有重宇别院之家,可以驻跸关防之外,不妨启请内廷鸾舆入其私第,庶可略尽骨肉私情,天轮中之至性。此旨一下,谁不踊跃感戴?现今周贵人的父亲已在家里动了工了,修盖省亲别院呢。又有吴贵妃的父亲吴天Щ家,也往城外踏看地方去了。这岂不有八九分了?”

赵嬷嬷道:“阿弥陀佛!原来如此。这样说,咱们家也要预备接咱们大小姐了?"贾琏道:“这何用说呢!不然,这会子忙的是什么?"凤姐笑道:“若果如此,我可也见个大世面了。可恨我小几岁年纪,若早生二三十年,如今这些老人家也不薄我没见世面了。说起当年太祖皇帝仿舜巡的故事,比一部书还热闹,我偏没造化赶上。”赵嬷嬷道:“唉哟哟,那可是千载希逢的!那时候我才记事儿,咱们贾府正在姑苏扬州一带监造海舫,修理海塘,只预备接驾一次,把银子都花的淌海水似的!说起来……"凤姐忙接道:“我们王府也预备过一次。那时我爷爷单管各国进贡朝贺的事,凡有的外国人来,都是我们家养活。粤,闽,滇,浙所有的洋船货物都是我们家的。”

赵嬷嬷道:“那是谁不知道的?如今还有个口号儿呢,说`东海少了白玉床,龙王来请江南王-,这说的就是奶奶府上了。还有如今现在江南的甄家,嗳哟哟,好势派!独他家接驾四次,若不是我们亲眼看见,告诉谁谁也不信的。别讲银子成了土泥,凭是世上所有的,没有不是堆山塞海的,`罪过可惜-四个字竟顾不得了。”凤姐道:“常听见我们太爷们也这样说,岂有不信的。只纳罕他家怎么就这么富贵呢?"赵嬷嬷道:“告诉奶奶一句话,也不过是拿着皇帝家的银子往皇帝身上使罢了!谁家有那些钱买这个虚热闹去?"正说的热闹,王夫人又打发人来瞧凤姐吃了饭不曾。凤姐便知有事等他,忙忙的吃了半碗饭,漱口要走,又有二门上小厮们回:“东府里蓉,蔷二位哥儿来了。”贾琏才漱了口,平儿捧着盆盥手,见他二人来了,便问:“什么话?快说。”凤姐且止步稍候,听他二人回些什么。贾蓉先回说:“我父亲打发我来回叔叔:老爷们已经议定了,从东边一带,借着东府里花园起,转至北边,一共丈量准了,三里半大,可以盖造省亲别院了。已经传人画图样去了,明日就得。叔叔才回家,未免劳乏,不用过我们那边去,有话明日一早再请过去面议。”贾琏笑着忙说:“多谢大爷费心体谅,我就不过去了。正经是这个主意才省事,盖造也容易,若采置别处地方去,那更费事,且倒不成体统。你回去说这样很好,若老爷们再要改时,全仗大爷谏阻,万不可另寻地方。明日一早我给大爷去请安去,再议细话。”贾蓉忙应几个"是"。

贾蔷又近前回说:“下姑苏聘请教习,采买女孩子,置办乐器行头等事,大爷派了侄儿,带领着来管家两个儿子,还有单聘仁,卜固修两个清客相公,一同前往,所以命我来见叔叔。”贾琏听了,将贾蔷打谅了打谅,笑道:“你能在这一行么?这个事虽不算甚大,里头大有藏掖的。”贾蔷笑道:“只好学习着办罢了。”

贾蓉在身旁灯影下悄拉凤姐的衣襟,凤姐会意,因笑道:“你也太躁心了,难道大爷比咱们还不会用人?偏你又怕他不在行了。谁都是在行的?孩子们已长的这么大了,`没吃过猪肉,也看见过猪跑-。大爷派他去,原不过是个坐纛旗儿,难道认真的叫他去讲价钱会经纪去呢!依我说就很好。”贾琏道:“自然是这样。并不是我驳回,少不得替他算计算计。”因问:“这一项银子动那一处的?"贾蔷道:“才也议到这里。赖爷爷说,不用从京里带下去,江南甄家还收着我们五万银子。明日写一封书信会票我们带去,先支三万,下剩二万存着,等置办花烛彩灯并各色帘栊帐缦的使费。”贾琏点头道:“这个主意好。”

凤姐忙向贾蔷道:“既这样,我有两个在行妥当人,你就带他们去办,这个便宜了你呢。”贾蔷忙陪笑说:“正要和婶婶讨两个人呢,这可巧了。”因问名字。凤姐便问赵嬷嬷。彼时赵嬷嬷已听呆了话,平儿忙笑推他,他才醒悟过来,忙说:“一个叫赵天梁,一个叫赵天栋。”凤姐道:“可别忘了,我可干我的去了。”说着便出去了。贾蓉忙送出来,又悄悄的向凤姐道:“婶子要什么东西,吩咐我开个帐给蔷兄弟带了去,叫他按帐置办了来。”凤姐笑道:“别放你娘的屁!我的东西还没处撂呢,希罕你们鬼鬼祟祟的?"说着一径去了。

这里贾蔷也悄问贾琏:“要什么东西?顺便织来孝敬。”贾琏笑道:“你别兴头。才学着办事,倒先学会了这把戏。我短了什么,少不得写信来告诉你,且不要论到这里。”说毕,打发他二人去了。接着回事的人来,不止三四次,贾琏害乏,便传与二门上,一应不许传报,俱等明日料理。凤姐至三更时分方下来安歇,一宿无话。

次早贾琏起来,见过贾赦贾政,便往宁府中来,合同老管事的人等,并几位世交门下清客相公,审察两府地方,缮画省亲殿宇,一面察度办理人丁。自此后,各行匠役齐集,金银铜锡以及土木砖瓦之物,搬运移送不歇。先令匠人拆宁府会芳园墙垣楼阁,直接入荣府东大院中。荣府东边所有下人一带群房尽已拆去。当日宁荣二宅,虽有一小巷界断不通,然这小巷亦系私地,并非官道,故可以连属。会芳园本是从北拐角墙下引来一股活水,今亦无烦再引。其山石树木虽不敷用,贾赦住的乃是荣府旧园,其中竹树山石以及亭榭栏杆等物,皆可挪就前来。如此两处又甚近,凑来一处,省得许多财力,纵亦不敷,所添亦有限。全亏一个老明公号山子野者,一一筹画起造。

贾政不惯于俗务,只凭贾赦,贾珍,贾琏,赖大,来升,林之孝,吴新登,詹光,程日兴等几人安插摆布。凡堆山凿池,起楼竖阁,种竹栽花,一应点景等事,又有山子野制度。下朝闲暇,不过各处看望看望,最要紧处和贾赦等商议商议便罢了。贾赦只在家高卧,有芥豆之事,贾珍等或自去回明,或写略节,或有话说,便传呼贾琏,赖大等领命。贾蓉单管打造金银器皿。贾蔷已起身往姑苏去了。贾珍,赖大等又点人丁,开册籍,监工等事,一笔不能写到,不过是喧阗热闹非常而已。暂且无话。

且说宝玉近因家中有这等大事,贾政不来问他的书,心中是件畅事,无奈秦钟之病日重一日,也着实悬心,不能乐业。这日一早起来才梳洗完毕,意欲回了贾母去望候秦钟,忽见茗烟在二门照壁前探头缩脑,宝玉忙出来问他:“作什么?"茗烟道:“秦相公不中用了!"宝玉听说,吓了一跳,忙问道:“我昨儿才瞧了他来,还明明白白,怎么就不中用了?"茗烟道:“我也不知道,才刚是他家的老头子来特告诉我的。”宝玉听了,忙转身回明贾母。贾母吩咐:“好生派妥当人跟去,到那里尽一尽同窗之情就回来,不许多耽搁了。”宝玉听了,忙忙的更衣出来,车犹未备,急的满厅乱转。一时催促的车到,忙上了车,李贵,茗烟等跟随。来至秦钟门首,悄无一人,遂蜂拥至内室,唬的秦钟的两个远房婶母并几个弟兄都藏之不迭。

此时秦钟已发过两三次昏了,移床易箦多时矣。宝玉一见,便不禁失声。李贵忙劝道:“不可不可,秦相公是弱症,未免炕上挺扛的骨头不受用,所以暂且挪下来松散些。哥儿如此,岂不反添了他的病?"宝玉听了,方忍住近前,见秦钟面如白蜡,合目呼吸于枕上。宝玉忙叫道:“鲸兄!宝玉来了。”连叫两三声,秦钟不睬。宝玉又道:“宝玉来了。”

那秦钟早已魂魄离身,只剩得一口悠悠余气在胸,正见许多鬼判持牌提索来捉他。那秦钟魂魄那里肯就去,又记念着家中无人掌管家务,又记挂着父亲还有留积下的三四千两银子,又记挂着智能尚无下落,因此百般求告鬼判。无奈这些鬼判都不肯徇私,反叱咤秦钟道:“亏你还是读过书的人,岂不知俗语说的:`阎王叫你三更死,谁敢留人到五更。-我们陰间上下都是铁面无私的,不比你们阳间瞻情顾意,有许多的关碍处。”正闹着,那秦钟魂魄忽听见"宝玉来了"四字,便忙又央求道:“列位神差,略发慈悲,让我回去,和这一个好朋友说一句话就来的。”众鬼道:“又是什么好朋友?"秦钟道:“不瞒列位,就是荣国公的孙子,小名宝玉。”都判官听了,先就唬慌起来,忙喝骂鬼使道:“我说你们放了他回去走走罢,你们断不依我的话,如今只等他请出个运旺时盛的人来才罢。”众鬼见都判如此,也都忙了手脚,一面又抱怨道:“你老人家先是那等雷霆电雹,原来见不得`宝玉-二字。依我们愚见,他是阳,我们是陰,怕他们也无益于我们。”都判道:“放屁!俗语说的好,`天下官管天下事-,自古人鬼之道却是一般,陰阳并无二理。别管他陰也罢,阳也罢,还是把他放回没有错了的。”众鬼听说,只得将秦魂放回,哼了一声,微开双目,见宝玉在侧,乃勉强叹道:“怎么不肯早来?再迟一步也不能见了。”宝玉忙携手垂泪道:“有什么话留下两句。”秦钟道:“并无别话。以前你我见识自为高过世人,我今日才知自误了。以后还该立志功名,以荣耀显达为是。”说毕,便长叹一声,萧然长逝了。萧然长逝了。
